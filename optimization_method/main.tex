\documentclass{article}
\usepackage[UTF8, scheme=plain,heading=true]{ctex}
\usepackage{amsthm}
\usepackage{amsmath}
\usepackage{amssymb}
\usepackage{breqn}
% Use verb inside other environments
\usepackage{fancyvrb}
\everymath{\displaystyle}
% Some may argue that the use of displaystyle in inline formulas may
% break line spacing and make the document look ugly. However, I
% prefer to follow the choice of the textbook setting
\usepackage{hyperref}
\usepackage{mathrsfs}
\title{无痛入门单纯形法}
\begin{document}
\maketitle

2025年3月26日, 笔者入门单纯形法未遂, 深感苦闷, 遂作此篇.

\section{这玩意TM有用吗?}
哥们, 高中写多少题了你也没问, 现在问有没有用有必要吗?
% TODO: Add delete line

\section{从微积分开始}

说实话, 我们已经在微积分里干过不知道多少次求最大最小值的问题了. 区区求线性函数的极值, 我们上来就是一个求导(梯度):
\[
    c = \nabla f =\left( \frac{\partial f}{\partial x_1},
    \frac{\partial f}{\partial x_2} \right) = (2,1)
\]

我们发现, 非常函数线性函数的梯度总是常数值, 这意味着求梯度使其等于零的做法是行不通的,
函数在梯度方向上始终保持严格单调递增. 所以我们要求最大值, 实际上是求函数可行域在线性函数\(f\) 梯度方向最远能延伸到哪里.

用一点投影的语言, 可行域内

\end{document}