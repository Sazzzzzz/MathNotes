\chapter{随机变量与分布函数}
随机变量是一个事件到实数(轴上的Borel集)的映射, 目的在于数值化样本点. 之所以掷骰子点数的期望看起来不自然,
是因为他依赖于随机变量对事件的映射.
随机变量还允许我们对变量进行数学运算.
\[
    \xi \sim N(0,1) \iff \xi+1 \sim N(1,1)
\]
\section{泊松分布的意义}

考虑在一小时内把时间划分为\(n\) 份进行伯努利试验, 并期望得到\(\lambda\)次试验结果. 那么随着\(n
\to \infty\), 得到结果次数\(k\) 服从的二项分布将逐渐逼近泊松分布. 这就是泊松分布的意义.
