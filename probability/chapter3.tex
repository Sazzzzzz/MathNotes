\chapter{随机变量与分布函数}
\section{随机变量及其分布
}
随机变量是一个事件到实数(轴上的Borel集)的映射,目的在于数值化样本点。之所以掷骰子点数的期望看起来不自然,
是因为他依赖于随机变量对事件的映射。随机变量还允许我们对变量进行数学运算。\[
    \xi \sim N(0,1) \iff \xi+1 \sim N(1,1)
\]

分布是随机变量的概率性质。

分布函数是随机变量取值小于等于某个特定值的概率。在离散型随机变量里,分布列是随机变量取特定值的结果。在连续型随机变量里,
概率密度函数是分布函数的导数。二者都把概率刻画成一定的直观测度,但截然不同。不可能通过趋于无穷的分布列逼近概率密度函数,
因为此时分布列必然所有元素都趋于0。但分布函数是可以由离散向连续逼近的。

% TODO: 增加二项分布逼近超几何分布

\subsection{离散型随机变量}
% 伯努利试验
% 二项分布的简单介绍
% 几何分布的无记忆性
% 由几何分布推导帕斯卡分布
% TODO:负二项分布?有什么用?
考虑在一小时内把时间划分为\(n\) 份进行伯努利试验,并期望得到\(\lambda\)次试验结果。那么随着\(n
\to \infty\),得到结果次数\(k\) 服从的二项分布将逐渐逼近泊松分布。这就是泊松分布的意义。

\subsection{连续型随机变量}
% 伯努利试验推导泊松分布(直观+公式)
% 指数分布的直观推导与无记忆性
% Erlang分布的推导和指数分布的推导过来
% TODO: 正态分布切片还是正态分布?
% TODO: P130计算机字长造成的误差是什么分布?不是对数分布?
% TODO: P137当时间紧迫我们应该求稳还是冒险
