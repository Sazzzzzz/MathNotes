% !TeX root = main.tex
\chapter{数字特征与特征函数}
\section{数学期望}
数学期望本质是一种加权平均。
% TODO: 定义(一般 离散 高维连续)
%todo:线性性
% 期望的另一个计算方法

\begin{theorem}{佚名统计学家公式}
    若\(g(x)\) 是一元博雷尔函数,\(\eta=g(\xi)\) ,则\[
        (E(\eta)) = \int_{-\infty}^{\infty} y \d{F_{\eta}}(y) =
        \int_{-\infty}^{\infty} g(x) \d{F_{\xi}}(x)
    \]
\end{theorem}
% todo:简单证明
% TODO:例子:保险行业 s核算检验德军坦克 cha最大值的期望
\subsubsection{二元正态分布最大值的期望}
期望和方差的很多题目用性质\underline{注意力}可以优雅干脆的解决,如果注意力不集中,力大砖飞的计算也没有问题。

\begin{problem}{二元正态分布最大值的期望}
    设\(X\)和\(Y\)是标准正态分布随机变量,相关系数为\(\rho\),
    求证:\[E\max(X,Y)= \sqrt{\frac{1-\rho}{\pi}}\]。
\end{problem}

\begin{quote}
    准备好了吗孩子们?
\end{quote}

\begin{proof}
    联合分布密度函数为:\[
        p(x,y) = \frac{1}{2\pi\sqrt{1-\rho^2}}
        \exp\left(-\frac{1}{2(1-\rho^2)}(x^2 - 2\rho xy + y^2)\right)
    \]
    \begin{align*}
        &\placeholder{=} E\max(X,Y)\\
        &= \iint\limits_{x>y} x\cdot p(x,y)
        \d{x}\d{y} + \iint\limits_{y\geq x} y\cdot p(x,y) \d{x}\d{y}\\
        &= \int_{-\infty}^{\infty } \int_{y}^{\infty } x
        p(x,y) \d{x} \d{y} + \int_{-\infty}^{\infty }
        \int_{x}^{\infty } y p(x,y) \d{y} \d{x}\\
        &= 2\int_{-\infty}^{\infty } xp(x,y) \d{x} \\
        &= \frac{1}{\pi \sqrt{1-\rho^{2}}}
        \underbrace{\int_{-\infty}^{+\infty} \int_{y}^{+\infty} x
            \exp\left( -\frac{1}{2(1-\rho^{2})}(x^{2} - 2\rho
        xy + y^{2}) \right) \d{x} \d{y}}_{I}
    \end{align*}
    记\(c = \sqrt{2(1-\rho^{2})}\)
    \begin{align*}
        I &= \e^{-\frac{y^{2}}{2}}\int_{y}^{+\infty} x \exp\left(
        -\frac{1}{2(1-\rho^{2})}(x -\rho y)^{2} \right) \d{x}\\
        & \xlongequal{ct = x-\rho y} \e^{-\frac{y^{2}}{2}}
        \underbrace{\int_{\frac{(1-\rho) y}{c}}^{\infty } t
        \e^{-t^{2}}  \d{t}}_{I_{1}} + \e^{-\frac{y^{2}}{2}} \rho y
        c \underbrace{\int_{\frac{(1-\rho)y}{c}}^{\infty}
        \e^{-t}  \d{t}}_{I_{2}}
    \end{align*}
    \begin{align*}
        I_{1} &= \frac{1}{2}
        \int_{\frac{(1-\rho)y}{c}}^{\infty } \e^{-t^{2}}
        \d{t^{2}} = \frac{1}{2} \e^{-\frac{1-\rho}{2(1+\rho)}y^{2}}\\
        I_{2} &= \frac{\sqrt{\pi}}{2} \erfc\left(
        \sqrt{\frac{1-\rho}{2(1+\rho)}}y \right)\\
        I &= \left( 1-\rho^{2}
        \right)\e^{-\frac{y^{2}}{1+\rho}} + \rho y
        \e^{-\frac{y^{2}}{2}}
        \sqrt{\frac{\pi}{2}\left( 1-\rho^{2} \right)}
        \erfc\left( \sqrt{\frac{1-\rho}{2(1+\rho)}}y \right)
    \end{align*}
    \begin{align*}
        &\placeholder{=} E\max(X,Y) \\
        &= \left( 1-\rho^{2} \right)
        \underbrace{\int_{-\infty }^{\infty}
        \e^{-\frac{y^{2}}{1+\rho}} \d{y}}_{J_{1}} + \rho
        \sqrt{\frac{\pi}{2}\left( 1-\rho^{2} \right)}
        \underbrace{\int_{-\infty }^{\infty } y\e^{-\frac{y^{2}}{2}}
            \erfc\left( \sqrt{\frac{1-\rho}{2(1+\rho)}}y
        \right) \d{y}}_{J_{2}}
    \end{align*}
    \[
        J_{1} \xlongequal{t=\frac{y}{\sqrt{1+\rho}}}
        (1+\rho)\underbrace{\int_{-\infty }^{\infty }
        \e^{-t^{2}}\d{t}}_{\text{Gaussian Integral}}=
        \sqrt{\pi(1+\rho)}
    \]
    \begin{align*}
        J_{2} &= \int_{-\infty}^{\infty }
        y\e^{-\frac{y^{2}}{2}} \left( 1-\erf\left(
        \sqrt{\frac{1-\rho}{2(1+\rho)}}y \right) \right) \d{y}\\
        &= 0+ 2\int_{0}^{\infty } \erf\left(
        \sqrt{\frac{1-\rho}{2(1+\rho)}}y \right)
        \d{\e^{-\frac{y^{2}}{2}} }\\
        &= \left. 2\e^{-\frac{y^{2}}{2}} \erf\left(
        \sqrt{\frac{1-\rho}{2(1+\rho)}}y \right)
        \right|_{0}^{\infty} - \frac{2}{\sqrt{\pi}} \int_{0}^{\infty }
        \e^{-\frac{y^{2}}{2}}
        \sqrt{\frac{1-\rho}{2(1+\rho)}}
        \e^{-\frac{1-\rho}{2(1+\rho)}y^{2}} \d{y}\\
        &= -\frac{4}{\sqrt{\pi}}
        \sqrt{\frac{1-\rho}{2(1+\rho)}} \int_{0}^{\infty}
        \e^{-\frac{y^{2}}{1+\rho}}  \d{y}\\
        &= -\sqrt{2(1-\rho)}
    \end{align*}
    \[
        E\max(X,Y) = \frac{1}{\pi\left( 1-\rho^{2} \right)}
        \left[ \left( 1-\rho^{2} \right)J_{1} +
            \rho\sqrt{\frac{\pi}{2}\left( 1-\rho^{2}
        \right)}J_{2} \right] = \sqrt{\frac{1-\rho}{\pi}}
    \]
\end{proof}
% todo:报童问题与不完全伽玛函数
\section{方差、相关系数、矩}
\subsection{方差}
为了描述随机变量的离散程度,一个符合直觉的度量应该是平均绝对偏差:
\[
    \frac{1}{n} \sum_{i=1}^{n} \abs{x_i - \mu}
\]
但其在数学上并不方便处理。于是引入方差:
\[
    \sigma^2 = \frac{1}{n} \sum_{i=1}^{n} (x_i - \mu)^2
\]
% 定义
% 性质:
% TODO: This looks ugly
% \begin{table}[H]
%     \centering
%     \begin{tabular}{c|c|c}
%         \hline
%         分布 & 数学期望 & 方差\\
%         \hline
%         二项分布\(B(n,p)\) & \(np\) & \(np(1-p)\) \\
%         泊松分布\(P(\lambda)\) & \(\lambda\) & \(\lambda\) \\
%         几何分布\(G(p)\) & \(\frac{1}{p}\) & \(\frac{1-p}{p^2}\) \\
%         超几何分布\(H(N,K,n)\) & \(\frac{nK}{N}\) &
%         \(\frac{nK(N-K)(N-n)}{N^2(N-1)}\) \\
%         均匀分布\(U(a,b)\) & \(\frac{a+b}{2}\) &
% \(\frac{(b-a)^2}{12}\) \\
%         指数分布\(E(\lambda)\) & \(\frac{1}{\lambda}\) &
%         \(\frac{1}{\lambda^2}\) \\
%         正态分布\(N(\mu,\sigma^2)\) & \(\mu\) & \(\sigma^2\) \\
%         卡方分布\(C(n)\) & \(n\) & \(2n\) \\
%         均匀分布\(U(0,1)\) & \(\frac{1}{2}\) &
%         \(\frac{1}{12}\) \\
%         \hline
%     \end{tabular}
% \end{table}
% 归一化和归一公式
\subsection{相关系数}
相关系数描述两个随机变量之间的线性关系。
% 各种性质
\subsubsection{相关与独立}
% 随机向量空间
\subsubsection{计算例子}
% 超几何分布与错排问题的期望与方差计算(和直接计算)
% 实用例子:噪声信号模型,投资模型
\subsection{矩}
% 杆的转动惯量与均匀分布的方差的关系
% 为什么要定义矩
% 性质
% 分位数的定义
% P276 35
\section{母函数}
母函数是概率分布的又一种刻画方法。一个母函数完全确定一个概率分布。
\begin{definition}{母函数}
    若随机变量\(\xi\) 取非负整数值,且相应的分布列为\(\{p_k\}_{k=0}^{\infty}\),则称
    \[
        P(s) = \sum_{k=0}^{\infty} p_k s^{k}
    \]
    为\(\xi\)的母函数。
\end{definition}

母函数只是一个形式幂级数,它所能解决的问题实在和幂级数本身收敛与否关系不大。既然如此,回想复变函数中学过的洛朗级数,
我们为何不能大胆一点,直接使用双边的幂级数\(\sum_{n=-\infty}^{\infty} p_{k}s^{k}
\) ,对所有离散随机变量给出定义呢?

\begin{definition}{双边Z变换}
    双边Z变换把离散时域信号\(x[n]\)转为形式幂级数\(X(z)\):
    \[
        X(z) = \mathcal{Z}\{x[n]\} =\sum_{n=-\infty}^{\infty}
        x[n] z^{-n}
    \]
\end{definition}

\begin{note}{生成函数与Z变换的关系}
    他俩其实是一个东西。

    单边Z变换是双边Z变换的一个特例,与母函数相比完全只相差\(z\) 的幂的系数。但相较而言,
    Z变换常出现于信号与系统,作为Laplace变换的离散时域对应,而生成函数常用于概率论和组合数学里,作为随机变量分布的刻画工具。

    因此,组合数学中用到的很多奇技淫巧例题实际上都可以用Z变换的性质来给出母函数版本的对应。
\end{note}

\subsection{生成函数的性质}
% (骰子、二项分布逼近波松)

\begin{description}
    \item[线性性] \(P\left\{ a x_{n}+b y_{n} \right\} = a
        P\{x_{n}\} + b P\{y_{n}\}\)
    \item[平移] \(P\{x_{n+k}\} = \frac{1}{s^{k}}
            \left( X(s) -
        (x_{0} +\dots +x_{k-1}s^{k-1} ) \right)\)
    \item[前向差分] \(P\{\Delta x_{n}\} = \frac{1}{s} \left(
        X(s)-x_{0} \right)- X(s)\)
    \item[缩放] \(P\{a^{n} x_{n}\} = X(as)\)
    \item[微分] \(P\{n x_{n}\} = s(X'(s))\)
    \item[卷积] \(P\{x_{n}y_{n}\} = X(s)Y(s)\)
\end{description}
% \footnote{均对应自\href{https://zh.wikipedia.org/wiki/Z%E8%BD%89%E6%8F%9B#%E5%B8%B8%E8%A7%81%E7%9A%84Z%E5%8F%98%E6%8D%A2%E5%AF%B9%E8%A1%A8}{Z变换的性质}}
% TODO: 降采样怎么对应?
\subsection{生成函数表}

% \begin{table}[H]
%     \centering
%     \begin{tabular}{|c|c|c|}
%         \hline
%         分布 & 生成函数 \\
%         \hline
%         二项分布\(B(n,p)\) & \((1-p + ps)^{n}\) \\
%         泊松分布\(P(\lambda)\) & \(\exp(\lambda(s-1))\) \\
%         几何分布\(G(p)\) & \(\frac{ps}{1-(1-p)s}\) \\
%         超几何分布\(H(N,K,n)\) &
% \(\frac{(1-s)^{N-K}(1-s)^{n}}{(1-s)^{N}}\) \\
%         指数分布\(E(\lambda)\) &
% \(\frac{\lambda}{\lambda - s}\) \\
%         正态分布\(N(\mu,\sigma^2)\) & \(\exp\left(\mu s +
%         \frac{\sigma^2 s^{2}}{2}\right)\) \\
%         \hline
%     \end{tabular}
% \end{table}