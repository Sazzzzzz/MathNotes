\chapter{数项级数}
\section{级数收敛性的概念和基本性质}

\begin{enumerate}
    \item \(\sum\limits_{n=1}^{\infty}
        \dfrac{n}{(n+1)(n+2)(n+3)}\)
        % TODO: Implement Gosper method
        \begin{align*}
            & \sum\limits_{n=1}^{\infty} \frac{n}{(n+1)(n+2)(n+3)} \\
            & = \lim_{k \to \infty} \sum\limits_{n=1}^{k}
            \left(\frac{n+\tfrac{1}{2}}{(n+1)(n+2)} -
            \frac{n+\tfrac{3}{2}}{(n+2)(n+3)}\right)                \\
            & = \lim_{k \to \infty} \left( \frac{1}{4} - \frac{k +
            \tfrac{1}{2}}{(k+1)(k+2)} \right)                       \\
            & = \frac{1}{4}
        \end{align*}
    \item \(\sum\limits_{n=1}^{\infty} \dfrac{2n-1}{2^{n}}\)
        % TODO: 裂项、Abel求和、求导、错位相减
        \begin{align*}
            & \sum\limits_{n=1}^{\infty}
            \dfrac{2n-1}{2^{n}}          \\
            & = \lim_{k \to \infty} \sum_{n=1}^{k}
            \frac{2k-1}{2^{k}} \\
            & = \lim_{k \to \infty} \sum_{n=1}^{k}
        \end{align*}
    \item \(\sum\limits_{n=1}^{\infty} \left(
        \sqrt{n+2}-2\sqrt{n+1}+\sqrt{n} \right) \)
        \begin{align*}
            & \sum\limits_{n=1}^{\infty} \left(
            \sqrt{n+2}-2\sqrt{n+1}+\sqrt{n} \right)
            \\
            & = \lim_{k \to \infty} \sum_{n=1}^{k} \left(
            \sqrt{n+2}-\sqrt{n+1}\right) -
            \left(\sqrt{n+1}-\sqrt{n} \right) \\
            & = \lim_{k \to \infty}  \sum_{n=1}^{k} \left(
                \frac{1}{\sqrt{n+2}+\sqrt{n+1}} -
            \frac{1}{\sqrt{n+1}+\sqrt{n}} \right)
            \\
            & = \lim_{k \to \infty} \left(
                \frac{1}{\sqrt{k+2}+\sqrt{k+1}} -
            \frac{1}{\sqrt{2}+\sqrt{1}} \right)
            \\
            & = 1 - \sqrt{2}
        \end{align*}

\end{enumerate}

\begin{problem}
    设 \(a_n \leq c_n \leq b_n,n = 1,2,\dots\)
    且级数\(\sum_{n=1}^{\infty} a_n\) 和 \(\sum_{n=1}^{\infty}
    b_n\) 都收敛,求证级数
    \(\sum_{n=1}^{\infty} c_n\) 也收敛。
\end{problem}
% TODO: The solution here is incomplete

\begin{solution}
    由于\(\sum_{n=1}^{\infty} a_{n}\) 和 \(\sum_{n=1}^{\infty} b_{n}\)
    均收敛,由Cauchy收敛原理知: \(\forall \varepsilon > 0,\exists N \in
    \N\)使得\(\forall n > N,p \in \N\),有:
    \begin{align*}
        \left| a_{n} + a_{n+1} + \dots + a_{n+p} \right| <
        \varepsilon \\
        \left| b_{n} + b_{n+1} + \dots + b_{n+p} \right| < \varepsilon
    \end{align*}
    由于
    \(a_{n} \leq c_{n} \leq b_{n}\),故:
    \[
        \left| c_{n} + c_{n+1} + \dots + c_{n+p} \right| < \varepsilon
    \]
    故级数\(\sum_{n=1}^{\infty} c_{n}\) 收敛.
\end{solution}

\begin{problem}

\end{problem}

\section{正项级数}
\begin{problem}
    判断下列级数的收敛性:
    \begin{enumerate}
        \item \(\sum_{n=1}^{\infty} \frac{1}{n^2}\)
        \item \(\sum_{n=1}^{\infty} \frac{1}{n(n+1)}\)
        \item \(\sum_{n=1}^{\infty} \frac{1}{n(n+1)(n+2)}\)
        \item \(\sum_{n=1}^{\infty} \frac{1}{n(n+1)(n+2)(n+3)}\)
        \item \(\sum_{n=1}^{\infty} \frac{1}{n(n+1)(n+2)(n+3)(n+4)}\)
    \end{enumerate}
\end{problem}
\[
    \sum_{n=1}^{\infty} \frac{2 \cdot 5 \cdot 8 \cdot \dots
    (3n-1)}{1 \cdot 5 \cdot 9 \cdot \dots (4n-3)} \\
\]
由比值审敛法:
\[
    D_{n} = \tfrac{\tfrac{3n+2}{4n+1}}{\tfrac{3n-1}{4n-3}} =
    \frac{12n^{2}-n-6}{12n^{2}-n-3}
\]
当\(\forall n > 0\),有\(D_{n} < 1\),故由D'Alembert判别法知级数收敛.

\[
    \sum_{n=1}^{\infty} \frac{3^n}{n!}
\]
由比值审敛法:
\[
    D_{n} = \frac{3}{n+1}
\]
当\(n>3\)时,有\(D_{n} < 1\),故级数收敛.

\[
    \sum_{n=1}^{\infty} \frac{3^{n}n!}{(2n)!}
\]
由比值审敛法:
\[
    D_{n} = \frac{3^{n+1}(n+1)!}{(2n+2)!} \cdot \frac{(2n)!}{3^{n}n!}
    = \frac{3}{2(n+1)}
\]

\section{正项级数收敛性的进一步讨论}

\begin{problem}
    设 \(\alpha > 0\),证明级数 \(\sum_{n=1}^{\infty} \left|
        \frac{\alpha(\alpha - 1) \dots (\alpha - n + 1)}{n!}
    \right| \) 收敛。
\end{problem}

\begin{solution}
    易知: \[
        \frac{u_{n}}{u_{n+1}}= \left| \frac{n+1}{\alpha-n} \right| =
        \frac{n+1}{n-\alpha}
    \]
    故当\(n\)足够大时,有\(\frac{u_{n}}{u_{n+1}}\to 1\),\(n \to +\infty\)
    且:
    \[
        \lim_{n \to \infty} R_{n} = \lim_{n \to \infty} n \left(
        \frac{n+1}{n-\alpha} \right)
        =\lim_{n \to \infty} \alpha +1 +
        \frac{\alpha(\alpha+1)}{n-\alpha} = 1+\alpha > 1
    \]
    故级数收敛
\end{solution}

\begin{problem}
    考虑以下级数的收敛性:
    \begin{itemize}
        \item \(\sum_{n=1}^{\infty}
                \frac{\sqrt{n!}}{(a+1)(a+\sqrt{2}) \dots
            (a+\sqrt{n})},\ a > 0\)
        \item \(\sum_{n=1}^{\infty} \frac{n!n^{-p}}{q(q+1)
                \dots (q+n)},
            \quad p > 0,\quad q > 0\)
    \end{itemize}
\end{problem}

\begin{solution}
    \begin{itemize}
        \item 易知: \[
                \lim_{n \to \infty} R_{n}=\lim_{n \to \infty} n\left(
                    \frac{\alpha+\sqrt{n+1}}{\sqrt{n+1}-1}
                \right)= \infty
            \]
            故级数发散.
        \item 易知: \[
                \frac{u_{n}}{u_{n+1}}= \left(
                1+\frac{q}{n+1} \right)\left(
                1+\frac{1}{n} \right)^{p}
            \]
            且有:
            % TODO: A lot of errors here.
            \begin{align}
                \left( 1+\frac{1}{n} \right)^{p} & \sim 1+
                \frac{p}{n}+
                \frac{p(p-1)}{2n^{2}} + o(\frac{1}{n^{2}})
                \\
                1 + \frac{q}{n+1}                & \sim
                1+\frac{q}{n}-\frac{q}{n^{2}}+o(\frac{1}{n^{2}})
            \end{align}
            % TODO: Should I use \sim?
            故:
            \begin{align*}
                \frac{u_{n}}{u_{n+1}} & \sim  1+(p+q)\cdot
                \frac{1}{n}+\left(
                \frac{p}{2}-q \right)\left( 1-p\right) \cdot
                \frac{1}{n^{2}}+o(\frac{1}{n^{2}})
                \\
                & \sim \lambda + \frac{\mu}{n} + \frac{\nu}{n^{2}} +
                o(\frac{1}{n^{2}})
            \end{align*}
            易知: \(\lambda = 1\),\(\mu = p+q\),\(\theta_{n} =
                \left(\frac{p}{2}-q
                \right)\left(
            1-p\right)\cdot\frac{1}{n^{2}}+o(\frac{1}{n^{2}})\),
            \begin{itemize}
                \item 当\(p+q > 1\)时,级数收敛.
                \item 当\(p+q \le 1\)时,有\(\theta_n\)有界,级数发散.
            \end{itemize}
    \end{itemize}

\end{solution}
\section{任意项级数}
\[
    1 - \frac{1}{2} + \frac{1}{3!} - \frac{1}{4} + \frac{1}{5!} -
    \frac{1}{6} + \dots
\]

\begin{align*}
    & 1 - \frac{1}{2} + \frac{1}{3!} - \frac{1}{4} + \frac{1}{5!} -
    \frac{1}{6} + \dots
    \\
    & = \sum_{n=1}^{\infty} \frac{1}{(2n-1)!} + \sum_{n=1}^{\infty}
    \frac{1}{2n}
    \\ &= \sin 1 + \frac{1}{2} \sum_{n=1}^{\infty} \frac{1}{n}
\end{align*}
故级数发散

\[
    \sum_{n=1}^{\infty} (-1)^{n+1} \frac{\ln n}{\sqrt{n}};
\]

\[
    \sum_{n=1}^{\infty} \left| (-1)^{n+1} \frac{\ln
    n}{\sqrt{n}} \right|
    > \sum_{n=1}^{\infty} \frac{1}{\sqrt{n}} ;
\]
故级数不绝对收敛.
且当\(n \to + \infty\),\(\frac{\ln n}{\sqrt{n}}\)单调趋于0.
由莱布尼茨判别法知级数条件收敛.

\[
    \sum_{n=1}^{\infty} (-1)^n \sin \frac{x}{n} \arctan n (x \neq 0);
\]

易知,\(\sin \frac{x}{n}\)单调趋于0,故由莱布尼茨判别法知\(\sum_{n=1}^{\infty} \sin
\frac{x}{n}\)收敛。同时\(\arctan n\)单调有界,故由迪利克雷判别法知\(\sum_{n=1}^{\infty}
(-1)^{n} \sin \frac{x}{n} \arctan n\)收敛.

但\[
    \left| (-1)^{n}\sin \frac{x}{n} \arctan n \right| < \frac{\pi}{2}
    \sin \frac{x}{n} \sim \frac{x}{n} + o(\frac{x}{n})
\]
故级数不绝对收敛。综上,原级数条件收敛.

\section{无穷乘积}
\section{组合级数与重排级数}
\section{无穷乘积}
\section{级数的乘积、累次级数与二重级数}
\begin{problem}
    设\(\left\vert q \right\vert < 1\),求证明:
    \[
        \left( \sum_{n=0}^{\infty} q^{n} \right)^2
        =\sum_{n=0}^{\infty} (n + 1) q^{n}
    \]
\end{problem}
\begin{solution}
    由于\(\left\vert q \right\vert < 1\),故\(\sum_{n=0}^{\infty}
    q^{n}\)绝对收敛。故级数乘积\(\left( \sum_{n=0}^{\infty} q^{n}
    \right)^2\)的任意重排收敛.
    考虑\(\sum_{n=0}^{\infty} q^{n}\)与自身的Cauchy形式的乘积:
    \[
        \left( \sum_{n=0}^{\infty} q^{n} \right)^2
        =\sum_{n=0}^{\infty} \left( \sum_{k=0}^{n} q^{k}
        q^{n-k} \right)
        =\sum_{n=0}^{\infty} (n + 1) q^{n}
    \]
\end{solution}

\begin{problem}
    设\(S(x) = \sum_{n=0}^{\infty} (-1)^n \frac{x^{2n+1}}{(2n+1)!}\),
    \(C(x) = \sum_{n=0}^{\infty} (-1)^n \frac{x^{2n}}{(2n)!}\),
    求证对于所有的\(x\),有\(S(2x)=2S(x)C(x)\),\([S(x)]^2+[C(x)]^2=1\)
\end{problem}

% TODO: Add a figure here.

\begin{solution}
    记\(S(x) = \sum_{n=0}^{\infty} (-1)^n \frac{x^{2n+1}}{(2n+1)!} =
    \sum_{n=0}^{\infty} s(x)\),\(C(x) = \sum_{n=0}^{\infty} (-1)^n
    \frac{x^{2n}}{(2n)!} = \sum_{n=0}^{\infty} c(x)\),则:
    \[
        \left\vert \frac{s_{n+1}}{s_{n}} \right\vert =
        \frac{x^2}{(2n+2)(2n+3)} \to 0,\quad n \to \infty
    \]

    \[
        \left\vert \frac{c_{n+1}}{c_{n}} \right\vert =
        \frac{x^2}{(2n+2)(2n+1)} \to 0,\quad n \to \infty
    \]

    故\(S(x)\)与\(C(x)\)都绝对收敛.

    \begin{align*}
        2S(x)C(x) & = 2\left( \sum_{n=0}^{\infty} (-1)^n
        \frac{x^{2n+1}}{(2n+1)!} \right)  \left(
            \sum_{n=0}^{\infty} (-1)^n
        \frac{x^{2n}}{(2n)!} \right)
        \\
        & = 2 \sum_{n=0}^{\infty} \sum_{i=0}^{n} (-1)^{i}
        \frac{x^{2i+1}}{(2i+1)!} (-1)^{n-i}
        \frac{x^{2n-2i}}{(2(n-i))!}           \\
        & = 2 \sum_{n=0}^{\infty} (-1)^n
        \frac{x^{2n+1}}{(2n+1)!} \cdot
        \sum_{i=0}^{n} C^{i}_{2n+1}
        \\
    \end{align*}
    而:
    \[
        2 \sum_{i=0}^{n} C^{i}_{2n+1} = \sum_{i=0}^{n} C^{i}_{2n+1} +
        \sum_{i=0}^{n} C^{2n+1-i}_{2n+1}  = \sum_{i=0}^{2n+1}
        C^{i}_{2n+1} = 2^{2n+1}
    \]
    故:
    \[
        2S(x)C(x) = \sum_{n=0}^{\infty} (-1)^n
        \frac{x^{2n+1}}{(2n+1)!}
        \cdot 2^{2n+1} = \sum_{n=0}^{\infty} (-1)^n
        \frac{(2x)^{2n+1}}{(2n+1)!} = S(2x)
    \]

    \begin{align*}
        &\mathrel{\phantom{=}} [S(x)]^2+[C(x)]^2\\
        & = \sum_{n=0}^{\infty} \sum_{i=0}^{n} (-1)^{i}
        \frac{x^{2i+1}}{(2i+1)!} \cdot (-1)^{n-i}
        \frac{x^{2n-2i+1}}{(2(n-i)+1)!}
        + \sum_{n=0}^{\infty} \sum_{i=0}^{n} (-1)^{i}
        \frac{x^{2i}}{(2i)!} \cdot (-1)^{n-i} \frac{x^{2n -
        2i}}{(2(n-i))!}      \\
        & = \sum_{n=0}^{\infty} (-1)^n x^{2n+2} \cdot \sum_{i=0}^{n}
        \left( \frac{1}{(2i+1)!} \cdot \frac{1}{(2(n-i)+1)!} \right) +
        \sum_{n=0}^{\infty} (-1)^n x^{2n} \cdot \sum_{i=0}^{n}
        \left( \frac{1}{(2i)!} \cdot \frac{1}{(2(n-i))!} \right)
        \\
        & = 1 + \sum_{n=0}^{\infty} (-1)^n x^{2n+2}
        \cdot
        % TODO:添加一个上括号
        \left[ \sum_{i=0}^{n} \left( \frac{1}{(2i+1)!} \cdot
            \frac{1}{(2(n-i)+1)!} \right)
            - \sum_{i=0}^{n+1} \left( \frac{1}{(2i)!} \cdot
        \frac{1}{(2(n-i))!} \right) \right]
    \end{align*}

    故只需证明\(A=0\).
    而:
    \[
        A=\frac{1}{n!}\left[ \sum_{i=1}^{n} C^{2i+1}_{2n} -
        \sum_{i=0}^{n} C^{2i}_{2n} \right] = - \frac{1}{n!}
        \sum_{i=0}^{n} (-1)^i C^{i}_{2n} =
        - \frac{1}{n!} \left( 1 - 1 \right)^2
        =0
    \]
    故\([S(x)]^2+[C(x)]^2=1\).
\end{solution}
\section{习题}
\begin{problem}
    求证: 级数\(\sum_{n=1}^{\infty} \frac{(-1)^n}{n}\)
    与自身的Cauchy形式的乘积是发散级数.
\end{problem}

\begin{solution}
    记级数\(\sum_{n=1}^{\infty} \frac{(-1)^n}{n}\)与自身的Cauchy形式的乘积为:
    \[
        \sum_{n=1}^{\infty} a_n =
        \sum_{n=1}^{\infty} \sum_{i=1}^{n+1} \frac{(-1)^i}{\sqrt{i}}
        \cdot \frac{(-1)^{n+1-i}}{\sqrt{n-i}} =
        \sum_{n=1}^{\infty} (-1)^n \sum_{i=1}^{n+1}
        \frac{1}{\sqrt{i(n-i+1)}}
    \]
    则:
    \[
        \lim_{n \to \infty} \left\vert a_n \right\vert
        = \lim_{n \to \infty} \sum_{i=1}^{n+1}
        \frac{1}{\sqrt{i(n-i+1)}}
        \ge \lim_{n \to \infty} \sum_{i=1}^{n+1} \frac{2}{n+1} = 2
        > 0
    \]
    故级数\(\sum_{n=1}^{\infty} \frac{(-1)^n}{n}\) 与自身的Cauchy形式的乘积是发散级数.
\end{solution}

\setcounter{problems}{4}
\begin{problem}
    设 \(\{a_n\}\) 是递增的正数数列,求证:级数 \(\sum_{n=1}^{\infty} \left(1 -
    \frac{a_n}{a_{n+1}}\right)\) 在 \(\{a_n\}\) 有界时收敛,而在
    \(\{a_n\}\) 无界时发散。
\end{problem}

\begin{solution}
    先证明\(\{a_{n}\}\)有界时, 级数\(\sum_{n=1}^{\infty} \left(1 -
    \frac{a_n}{a_{n+1}}\right)\) 收敛.

    由\(\{a_{n}\}\) 是递增有界的正数数列知,\(\{a_{n}\}\) 收敛.
    \(\sum_{n=1}^{\infty} (a_{n+1} - a_{n})\)
    与\(\sum_{n=1}^{\infty} \left(1 -
    \frac{a_n}{a_{n+1}}\right)\)均为正项级数。由\(\sum_{n=1}^{\infty}
    (a_{n+1} - a_{n})\) 的部分和数列为\(U_{n} = a_{n+1} - a_{1}\),
    且\(\{a_{n}\}\) 收敛,故 \(\sum_{n=1}^{\infty} (a_{n+1} - a_{n})\) 收敛.

    记\(u_{n} = a_{n+1} - a_{n}\),\(v_{n} = 1 -
    \frac{a_{n}}{a_{n+1}}\),则: \(\frac{v_{n}}{u_{n}} =
    \frac{1}{a_{n+1}} > 0\) 有界,
    由比较判别法知级数\(\sum_{n=1}^{\infty} \left(1 -
    \frac{a_n}{a_{n+1}}\right)\) 收敛.

    再证明\(\{a_{n}\}\)无界时,级数\(\sum_{n=1}^{\infty} \left(1 -
    \frac{a_n}{a_{n+1}}\right)\) 发散.

    下证: \[
        \left(1 - \frac{a_{n}}{a_{n+1}}\right) + \left(1 -
        \frac{a_{n+1}}{a_{n+2}}\right) + \dots +
        \left(1 - \frac{a_{n+p}}{a_{n+p+1}}\right) \geq 1 -
        \frac{a_{n}}{a_{n+p+1}},\forall n,p \in \N
    \]

    用数学归纳法证明:

    当\(p = 1\) 时,显然成立.
    设当\(p = k\) 时成立,则当\(p = k+1\) 时:
    \begin{align*}
        &\mathrel{\phantom{=}} \left(1 - \frac{a_{n}}{a_{n+1}}\right)
        + \left(1 - \frac{a_{n+1}}{a_{n+2}}\right) + \dots + \left(1 -
        \frac{a_{n+k+1}}{a_{n+k+2}}\right) - \left(1 -
        \frac{a_{n}}{a_{n+k+2}}\right)\\
        & \geq \left(1 - \frac{a_{n}}{a_{n+k+1}}\right) + \left(1 -
        \frac{a_{n+k+1}}{a_{n+k+2}}\right) - \left(1 -
        \frac{a_{n}}{a_{n+k+2}}\right)\\
        &= \left(a_{n+k+1} - a_{n}\right) \cdot \left(
            \frac{1}{a_{n+k+1}} -
        \frac{1}{a_{n+k+2}} \right)\\
        & \geq 0
    \end{align*}
    故命题得证.

    \(\exists \varepsilon = \frac{1}{2}\),\(\forall N \in
    \N\),取\(n = N\),由\(\{a_{n}\}\)无界知\(\exists p >
    N\)使\(a_{n+p+1} > 2a_{n}\)。于是:
    \[
        \left| \left( 1 - \frac{a_{n}}{a_{n+1}} \right) + \left( 1 -
        \frac{a_{n+1}}{a_{n+2}} \right) + \dots + \left( 1 -
        \frac{a_{n+p}}{a_{n+p+1}} \right)  \right| \geq 1 -
        \frac{a_{n}}{a_{n+p+1}} > \frac{1}{2}
    \]
    由Cauchy收敛原理知级数\(\sum_{n=1}^{\infty} \left(1 -
    \frac{a_n}{a_{n+1}}\right)\) 发散.
\end{solution}

\begin{problem}
    求证: \(\sum_{n=2}^{\infty} \frac{1}{(\ln n)^{\ln n}}\) 收敛而
    \(\sum_{n=3}^{\infty} \frac{1}{(\ln n)^{\ln \ln n}}\) 发散。问
    \(\sum_{n=3}^{\infty} \frac{1}{(\ln \ln n)^{\ln n}}\) 是否收敛?
\end{problem}

\begin{solution}
    \begin{itemize}
        \item \(\sum_{n=1}^{\infty} \frac{1}{(\ln n)^{\ln n}}\) 收敛.
            当\(n > \e^{\e^{2}}\),则\(\ln \ln n > 2\),有:
            \[
                \frac{1}{(\ln n)^{\ln n}} = \frac{1}{\exp(\ln n
                \ln\ln n)} < \frac{1}{\exp(2\ln n)} = \frac{1}{n^2}
            \]
            由比较判别法知级数收敛.
        \item \(\sum_{n=3}^{\infty} \frac{1}{(\ln n)^{\ln
            \ln n}}\) 发散.
            易知\(\frac{\ln x}{x} < \frac{1}{\e}\),\(\forall x > 0\).
            则\[
                \frac{(\ln\ln n)^{2}}{\ln n} = \left( \frac{2\ln
                \sqrt{\ln n}}{\sqrt{\ln n}} \right)^{2} < \left(
                \frac{2}{\e} \right)^{2} < 1
            \]
            即\(\ln n > (\ln \ln n)^{2}\),故\[
                \frac{1}{(\ln n)^{\ln \ln n}} = \frac{1}{\exp(\ln\ln
                n)^{2}} > \frac{1}{\exp \ln n} = \frac{1}{n}
            \]
            由比较判别法知级数发散.
        \item \(\sum_{n=3}^{\infty} \frac{1}{(\ln \ln
            n)^{\ln n}}\) 收敛.
            当\(n > \e^{\e^{\e^{2}}}\) 时,
            \(\ln\ln\ln n > 2\),于是: \[
                \frac{1}{(\ln \ln n)^{\ln n}} = \frac{1}{\exp(\ln n
                \ln\ln\ln n)} < \frac{1}{\exp(2\ln n)} = \frac{1}{n^2}
            \]
            由比较判别法知级数收敛.
    \end{itemize}
\end{solution}
\setcounter{problems}{11}
\begin{problem}
    设级数 \(\sum_{n=1}^{\infty} (a_n - a_{n-1})\) 绝对收敛,级数
    \(\sum_{n=1}^{\infty} b_n\) 收敛,求证:级数
    \(\sum_{n=1}^{\infty} a_n b_n\) 也收敛。
\end{problem}

\begin{solution}
    由Abel求和公式知: \[
        \sum_{n=1}^{m} a_{n}b_{n} = a_{m}b_{m} - \sum_{n=1}^{m-1}
        (a_{n+1} - a_{n})b_{n}
    \]

\end{solution}
