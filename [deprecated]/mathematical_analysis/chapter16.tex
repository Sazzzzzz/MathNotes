\chapter{一致收敛}
\section{函数列的一致收敛性}
\begin{problem}
    判定下列各函数列在指定区间上的一致收敛性:
    \begin{enumerate}
        \item \(f_n(x) = \sin \frac{x}{n}, \quad -\infty <
            x < +\infty\);
        \item \(f_n(x) = \frac{x}{n} \ln \frac{x}{n}, \quad
            0 < x < 1\)
        \item \(f_n(x) = \arctan nx, \quad (a) - 1 \leq x
                \leq 1, \quad (b) 0 <
            \delta < x < +\infty\);
        \item \(f_n(x) = nxe^{-nx}, \quad (a) 0 \leq x <
                +\infty, \quad (b) 0 <
            \delta < x < +\infty\);
        \item \(f_n(x) = \frac{x^n}{1 + x^n}, \quad (a) 0
                \leq x \leq 1 -
                \delta, \quad (b) 1 - \delta \leq x \leq 1 + \delta,
            \quad (c) 1 + \delta \leq x < +\infty\),其中 \(0 <
            \delta < 1\)。
    \end{enumerate}
\end{problem}

\begin{solution}
    \begin{enumerate}
        \item 不一致收敛。
            由于\(f(x)=\lim_{n \to \infty} \sin \frac{x}{n} = 0\),
            故\(f_{n}(x)\)逐点收敛到\(0\)。

            但:
            \[
                \lim_{n \to \infty} \left\lVert f_{n}(x) -
                f(x) \right\rVert
                = \lim_{n \to \infty} \sup_{x \in
                \R} \left\lvert \sin
                \frac{x}{n} \right\rvert
                = 1
            \]
            故不一致收敛。

        \item 一致收敛。
            由于\(f(x)=\lim_{n \to \infty} \frac{x}{n} \ln
            \frac{x}{n} = 0\),
            故\(f_{n}(x)\)逐点收敛到\(0\)。

            令\(g(x)=x \ln x\)。则可知\(g(x)\)在\((0,\frac{1}{\e})\)上单调递减。
            故\(\forall x \in (0,1), n > 3\),有\(\frac{x}{n} <
            \frac{1}{\e}\)。故\(f_{n}(x) = \frac{x}{n} \ln
            \frac{x}{n}\)单调递增。由Dini定理知\(f_{n}(x)\)一致收敛到\(0\)。

        \item 一致收敛。
            当\(x \in [0,1-\delta]\),此时\(\lim_{n \to
            \infty}f_{n}(x) = 0\)。且:
            \[
                \lim_{n \to \infty}\left\lVert f_{n}(x) - 0
                \right\rVert
                = \lim_{n \to \infty} \sup_{x \in
                [0,1-\delta]} \left\vert 1 -
                \frac{1}{1+ x^{n}} \right\vert
                = 0
            \]
            故\(f_{n}(x)\)在\([0,1-\delta]\)上一致收敛到\(0\)。

        \item 不一致收敛。

            当\(x \in [1-\delta,1+\delta]\),有:
            \[
                \frac{x^{n}}{1 + x^{n}} < \frac{(1+\delta)^{n}}{1+
                (1+\delta)^{n}} \le \frac{1}{2}
            \]

            考虑函数\(g(x) = \frac{1}{1-x}\)在\([0,\frac{1}{2}]\)的一致连续性。

            \(\forall x_1, x_2 \in [0, \frac{1}{2}]\) 且
            \(\left\vert x_1 - x_2
            \right\vert < \delta\),有:
            \[
                \left\vert g(x_1) - g(x_2) \right\vert
                = \left\vert \frac{x_1 -
                x_2}{(1-x_1)(1-x_2)} \right\vert
                \le 4 \delta
            \]
            故\(g(x)\)在\([0,\frac{1}{2}]\)上一致连续。
            % TODO: What's the right format for proof by
            % contradiction?

        \item 不一致收敛

            \textbf{反证法}:
            假设\(f_{n}(x)\)在\([1-\delta,1+\delta]\)上一致收敛,则
            \(g(f_n(x))=x^{n}\)在\([1-\delta,1+\delta]\)上一致收敛。
            由于\(g(f_n(x))\)在\([1-\delta,1+\delta]\)上一致收敛,
            故\(g(f_n(x)) = x^{n}\)在\([1-\delta,1+\delta]\)上一致连续。
            但\(x^{n}\)在\([1-\delta,1+\delta]\)上不一致连续,矛盾!

            故\(f_{n}(x)\)在\([1-\delta,1+\delta]\)上不一致收敛。

        \item 一致收敛。
            当\(x \in [1+\delta,+\infty)\),
            可知\(f_n(x)\)逐点收敛到\(1\)。%ChkTeX 9
            \[
                \lim_{n \to \infty} \left\lVert f_{n}(x) -
                1 \right\rVert
                = \lim_{n \to \infty} \sup_{x \in
                [1+\delta,+\infty)} \left\vert %ChkTeX 9
                \frac{1}{1 + x^{n}} \right\vert
                = \lim_{n \to \infty} \frac{1}{1+(1+\delta)^{n}} = 0
            \]
            故\(f_{n}(x)\)在\([1+\delta,+\infty)\)上一致收敛到\(1\)。%ChkTeX 9
    \end{enumerate}
\end{solution}

\section{一致收敛与极限换序}
\begin{problem}
    设 \(\{f_n(x)\}\) 是 \([a, b]\) 上的连续函数列,且一致收敛于 \(f(x)\),又设\(\{x_n\}
    \subseteq [a, b] \ \)且\( \lim_{n \to \infty} x_n = x_0\),
    求证:\(\lim_{n \to \infty} f_n(x_n) = f(x_0)\)
\end{problem}

\begin{solution}
    可知,\(f(x)\) 是连续函数。故\(\forall \varepsilon > 0\),
    \(\exists \delta > 0\)
    使得\(\forall x \in B_{\delta}(x_{0})\),有:
    \[
        \left| f(x)- f(x_{0}) \right| < \frac{\varepsilon}{2}
    \]
    由于\(\lim_{n \to \infty} x_{n}=x_{0}\) 对于上述的\(\delta\),
    存在\(N>0\),使得:
    \[
        \left| x_{n} - x_{0} \right| < \delta, \quad \forall n > N
    \]
    故 \(\forall \varepsilon > 0\), \(\exists N_1 > 0\),使得:
    \[
        \left| f(x_{n})- f(x_{0}) \right| <
        \frac{\varepsilon}{2}, \quad
        \forall n > N_1
    \]
    同时\(f_{n}(x) \rightrightarrows f(x)\), 故\(\exists N_2 > 0\),使得:
    \[
        \left| f_{n}(x) - f(x) \right| < \frac{\varepsilon}{2}, \quad
        \forall n > N_2
    \]
    综上: \(\forall \varepsilon > 0\),取\(N = \max\{N_1, N_2\}\),则:
    \[
        \left| f_{n}(x_{n}) - f(x_{0}) \right| \le \left|
        f_{n}(x_{n}) -
        f(x_{n}) \right| + \left| f(x_{n}) - f(x_{0})
        \right| < \varepsilon
    \]
    即 \(\lim_{n \to \infty} f_{n}(x_{n}) = f(x_{0})\)。
\end{solution}
% A normal case of the theorem

\begin{problem}
    设函数列\[
        f_n(x) = nx e^{-nx^2} , \quad n = 1, 2, \dots
    \]
    在 \([0, 1]\) 上逐点收敛于 \(f(x)\),求证:
    \begin{enumerate}
        \item \(\{f_n(x)\}\) 在 \([0, 1]\) 上非一致收敛;
        \item \(\int_{0}^{1} f(x) \mathrm{d}x \neq \lim_{n \to \infty}
            \int_{0}^{1} f_n(x) \mathrm{d}x\)
    \end{enumerate}
\end{problem}

\begin{solution}
    由于\(f(x) = \lim_{n \to \infty} nx e^{-nx^2} = 0\),
    故\(f_{n}(x)\)逐点收敛到\(f(x) = 0\)。
    可知\(f_{n}'(x) = n \e^{-nx^{2}}(1-2nx)\),求得\(f'\left(
    \frac{1}{2n} \right) = 0\)。故\(f_{n}(x)\)
    的最大值在\(\frac{1}{2n}\)处取得。
    故:
    \[
        \lim_{n \to \infty} \left\lVert f_{n}(x) - f(x) \right\rVert
        = \lim_{n \to \infty} f_{n}(1/2n) = \lim_{n \to \infty}
        \frac{1}{2} \e^{-\frac{1}{4n}} = \frac{1}{2} >  0
    \]
    故\(f_{n}(x)\)在\([0,1]\)上非一致收敛。

    \begin{align*}
        & \int_{0}^{1} f(x) \mathrm{d}x = 0                       \\
        & \lim_{n \to \infty} \int_{0}^{1} f_{n}(x) \mathrm{d}x =
        \lim_{n \to \infty} \frac{1}{2}\int_{0}^{1}
        \e^{-nx^{2}} \mathrm{d}nx^{2} = \frac{1}{2}
    \end{align*}
    故\(\int_{0}^{1} f(x) \mathrm{d}x \neq \lim_{n \to
    \infty} \int_{0}^{1}\)

\end{solution}

\begin{problem}
    求证:函数列
    \[
        f_n(x) = nx(1-x)^n, \quad n = 1, 2, \dots
    \]
    在 \([0, 1]\) 上逐点收敛但非一致收敛,可是却有
    \[
        \lim_{n \to \infty} \int_0^1 f_n(x) \mathrm{d}x =
        \int_0^1 \left[
        \lim_{n \to \infty} f_n(x) \right] \mathrm{d}x.
    \]
\end{problem}

\begin{solution}
    可知,\(f(x) = \lim_{n \to \infty} nx(1-x)^n = 0\),
    故\(f_{n}(x)\)逐点收敛到\(f(x) = 0\)。
    由均值不等式知: \[
        \left\lVert f_{n}(x) - f(x) \right\rVert =
        nx(1-x)^{n} \le \left( \frac{n}{n + 1} \right)^{n+1}
    \]
    在\(x = \frac{1}{2}\) 处取得最大值。
    故: \[
        \lim_{n \to \infty} \left\lVert f_{n}(x) - f(x) \right\rVert
        = \lim_{n \to \infty} nx(1-x)^{n} = \lim_{n \to \infty}
        \left( \frac{n}{n + 1} \right)^{n+1} = \frac{1}{\e} > 0
    \]
    故\(f_{n}(x)\)在\([0,1]\)上非一致收敛。

    \begin{align*}
        & \lim_{n \to \infty} \int_0^1 f_n(x) \mathrm{d}x
        = \lim_{n \to \infty} \int_0^1 nx(1-x)^n \mathrm{d}x
        = \lim_{n \to \infty} \frac{n}{n+1} = 1
        \\
        & \int_0^1 \left[ \lim_{n \to \infty} f_n(x)
        \right] \mathrm{d}x
        = \int_0^1 0 \mathrm{d}x = 0
    \end{align*}
    故\(\lim_{n \to \infty} \int_0^1 f_n(x) \mathrm{d}x \neq \int_0^1
    \left[\lim_{n \to \infty} f_n(x) \right] \mathrm{d}x\)。
\end{solution}

\begin{problem}
    问参数\(\alpha\) 分别取什么值时,函数列
    \[
        f_n(x) = n^\alpha x\e^{-n x}, \quad n = 1, 2, \dots
    \] 在\([0,1]\) 上:
    \begin{itemize}
        \item 收敛
        \item 一致收敛
        \item 可以在积分号下取极限,即\[
                \lim_{n \to \infty} \int_0^1 f_n(x)
                \mathrm{d}x = \int_0^1
                \left[ \lim_{n \to \infty} f_n(x) \right] \mathrm{d}x
            \]
    \end{itemize}
\end{problem}

\begin{solution}
    \begin{itemize}
        \item 注意到\(\e^{-nx}\)是\(x\) 的高阶无穷小,故\(\forall \alpha
            \in \R\) 都有: \[
                \lim_{n \to \infty} n^{\alpha} x \e^{-n x} = 0
            \]
        \item 考虑\(f_{n}(x)\)的最大值,令\(f_{n}'(x) = n^{\alpha}
            \e^{-\alpha x} (1 - n x) = 0\)。解得\(x =
            \frac{1}{n}\)。
            则\[
                \lim_{n \to \infty} \left\lVert f_{n}(x) - f(x)
                \right\rVert = \lim_{n \to \infty} \left|
                f_{n}\left(\frac{1}{n}\right) \right| = \lim_{n \to
                \infty} n^{\alpha - 1}
            \]
            故当且仅当\(\alpha <  1\) 时,\(f_{n}\) 一致收敛。
        \item 注意到\(\int_0^1 \left[ \lim_{n \to \infty} f_n(x) \right]
            \mathrm{d}x = \int_0^1 0 \mathrm{d}x = 0\),且有:
            \begin{align*}
                & \lim_{n \to \infty} \int_0^1 f_n(x) \mathrm{d}x \\
                & = \lim_{n \to \infty} \int_{0}^{1} n^{\alpha} x
                \e^{-nx} \mathrm{d}x                       \\
                & = \lim_{n \to \infty} -n^{\alpha - 1} \left(
                    \e^{-n} + \frac{1}{n}\e^{-n}
                -\frac{1}{n} \right)                               \\
                & = \lim_{n \to \infty} -n^{\alpha - 2}
            \end{align*}
            故当且仅当\(\alpha > 2\)时,\(f_{n}\)可以在积分号下取极限。
    \end{itemize}
\end{solution}

\begin{problem}
    求证:函数列
    \[
        f_n(x) = \frac{1}{n} \arctan x^n, \quad n = 1, 2, \dots
    \]
    在 \((-\infty, +\infty)\) 上一致收敛,但是
    \[
        \left. \left[ \lim_{n \to \infty} f_n(x) \right]'
        \right|_{x = 1}
        \neq \lim_{n \to \infty} f_n'(1).
    \]
\end{problem}

\begin{solution}
    由于\(\arctan x^{n}\)在 \(\R\)上有界,则:
    \[
        0 \le  \left\lVert f_{n}(x) - f(x) \right\rVert \le
        \frac{1}{n} \cdot \frac{\pi}{2}
    \]
    即\(\lim_{n \to \infty} \left\lVert f_{n} - f(x) \right\rVert =
    0\),故函数列一致收敛于常函数\(0\)。

    注意到:\(f_{n}'(x) = \frac{x^{n - 1}}{x^{2n} + 1}\),有:
    \begin{align*}
        \lim_{n \to \infty} f_{n}(1) & = \frac{1}{2} \\
        \left. \left[ \lim_{n \to \infty} f_{n}(x) \right]'
        \right|_{x=1}                & = 0           \\
    \end{align*}
    故: \[
        \left. \left[ \lim_{n \to \infty} f_n(x) \right]'
        \right|_{x = 1}
        \neq \lim_{n \to \infty} f_n'(1).
    \]
\end{solution}

\begin{problem}
    求证:函数列
    \[
        f_n(x) = x^2 + \frac{1}{n} \sin n \left( x + \frac{\pi}{2}
        \right), \quad n = 1, 2, \dots
    \]
    在 \((-\infty, +\infty)\) 上一致收敛,但是
    \[
        \left[ \lim_{n \to \infty} f_n(x) \right]' \neq \lim_{n \to
        \infty} f_n'(x).
    \]
\end{problem}

\begin{solution}
    可知函数列一致收敛于\(x^{2}\)。
    但是,
    \begin{align*}
        & \left[ \lim_{n \to \infty} f_{n}(x) \right]' = 2x        \\
        & \lim_{n \to \infty} f'_{n}(x) = \lim_{n \to \infty} 2x +
        \cos n \left( x + \frac{\pi}{2} \right)
    \end{align*}
    故\(\left.\left[ \lim_{n \to \infty} f_n(x) \right]\right|_{x=1}'
        \neq \lim_{n \to
    \infty} f'_{n}(x)\)
\end{solution}

\begin{problem}
    设函数列
    \[
        f_n(x) = \frac{1}{n} e^{-n^2 x^2}, \quad n = 1, 2, \dots
    \]
    在 \((-\infty, +\infty)\) 上逐点收敛于 \(f(x)\),求证:
    \begin{itemize}
        \item \(\{f_n(x)\}\) 在 \((-\infty, +\infty)\) 上一致收敛;
        \item \(\{f_n'(x)\}\) 在任何包含0的区间上都非一致收敛;
        \item \(\lim_{n \to \infty} f_n'(x) = f'(x)\)。
    \end{itemize}
\end{problem}

\begin{solution}
    \begin{itemize}
        \item 可知函数列一致收敛于\(0\)。
        \item 可知,
            \begin{align*}
                f_{n}'(x)  & = -2nx\e^{-n^{2}x^{2}}     \\
                f_{n}''(x) & = 2n\e^{-n^{2}x^{2}}\left(
                2n^{2}x^{2} - 1\right)
            \end{align*}
            故\(f_{n}(x)\)最大值在\(x = -\frac{\sqrt{2}}{2n}\) 处取到。
            则: \[
                \lim_{n \to \infty} \left\lVert f_{n}'(x) - f(x)
                \right\rVert = \lim_{n \to \infty} \left|
                f'_{n}\left(\frac{\sqrt{2}}{2n}\right) \right| =
                -2n\e^{-\frac{1}{2}} \to \infty
            \]
            故函数列不一致收敛
        \item 可知:
            \begin{align*}
                f'(x)                         & = 0
                \\
                \lim_{n \to \infty} f'_{n}(x) & = \lim_{n \to \infty}
                \frac{2x}{n}\e^{-n^{2}x^{2}} = 0
            \end{align*}
    \end{itemize}
\end{solution}
\section{逼近定理}
\section{函数项级数的一致收敛}
\section{利用函数项级数构造特殊函数的例子}
\section{习题}

\begin{problem}
    设 \(f_0(x)\) 在 \([a, b]\) 上连续,求证:函数列
    \begin{align*}
        f_n(x) = \int_a^x f_{n-1}(t) \, \mathrm{d}t, \quad
        n = 1, 2, \dots
    \end{align*}
    在 \([a, b]\) 上一致收敛于0。
\end{problem}

\begin{solution}
    首先证明: \(\forall x \in [a,b], \left\vert f_{n}(x) \right\vert \le
    \frac{M(x-a)^{n}}{k!} \)
    % TODO: What's the right format for mathematical induction?

    当\(n=0\)时,由连续函数的性质可知,\(f_{0}(x)\)在\([a,b]\)上有最大值\(M\)。
    即\(f_{0}(x)\le M\)。
    设当\(n=k\)时,\(\forall x \in [a,b]\),有: \(\left\vert f_{k}(x)
        \right\vert \le
    \frac{M(x-a)^{k}}{k!}\)。则:
    \[
        \left\vert f_{k+1}(x) \right\vert
        = \left\vert \int_{a}^{x} f_{k}(t) \mathrm{d}t \right\vert
        \le \int_{a}^{x} \left\vert f_{k}(t) \right\vert \mathrm{d}t
        \le \int_{a}^{x} \frac{M(t-a)^{k}}{k!} \mathrm{d}t =
        \frac{M(x-a)^{k+1}}{(k+1)!}
    \]
    故原命题成立。

    则:
    \[
        \lim_{n \to \infty} \left\lVert f_{n}(x) - 0 \right\rVert
        =\lim_{n \to \infty} \sup_{x \in [a,b]} \left\vert f_{n}(x)
        \right\vert = \lim_{n \to \infty} \frac{M(b-a)^{n}}{n!} = 0
    \]
    即\(f_{n}(x)\)一致收敛到\(0\)。
\end{solution}