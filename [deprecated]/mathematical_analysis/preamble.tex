% Packages for math
\usepackage{amsmath}
% For theorem environments, which is not used in this document
% \usepackage{amsthm}
% For math symbols, e.g. \mathbb, \mathcal
\usepackage{amssymb}
% Dor a range of integral symbols
\usepackage{esint}
\usepackage{physics}

% ==================================================

% Package for Chinese characters
\usepackage[UTF8, scheme=plain,heading=true]{ctex}

% Packages for writing
% For better control of the layout of lists, which is not
% used in this document
% \usepackage{enumerate}
% \usepackage[shortlabels]{enumitem}
% \usepackage{framed}
% \usepackage{csquotes}

% ==================================================

% Miscellaneous packages
% For environment setting
\usepackage{environ}
% For table setting, which is not used in this document
% \usepackage{float}
% For better table layout, which is not used in this document
% \usepackage{tabularx}
% For color setting, used in hyperlinks temporarily
\usepackage{xcolor}
% For multi-column layout, which might be useful for some problems
\usepackage{multicol}
% For subcaption and caption setting, which is not used in
% this document
% \usepackage{subcaption}
\usepackage{caption}
\captionsetup{format = hang, margin = 10pt, font = small,
labelfont = bf}

% Hyperlinks setup
\usepackage{hyperref}
\definecolor{links}{rgb}{0.36,0.54,0.66}
\hypersetup{
    colorlinks = true,
    linkcolor = black,
    urlcolor = blue,
    citecolor = blue,
    filecolor = blue,
    pdfauthor = {Author},
    pdftitle = {Title},
    pdfsubject = {subject},
    pdfkeywords = {one, two},
    pdfproducer = {LaTeX},
    pdfcreator = {pdfLaTeX},
}
%% Customizes the appearance of sectioning commands like \chapter,
% \section, etc.
\usepackage{titlesec}
% For colorbox setting, which is not used in this document
% \usepackage[many]{tcolorbox}

% Adjust spacing after the chapter title
\titlespacing*{\chapter}{0cm}{-2.0cm}{0.50cm}
\titlespacing*{\section}{0cm}{0.50cm}{0.25cm}

% Indent
\setlength{\parindent}{0pt}
\setlength{\parskip}{1ex}

\newcounter{problems}[section]
\numberwithin{problems}{section}
\NewEnviron{problem}{
    \refstepcounter{problems}
    \noindent\textbf{Problem \theproblems:} \BODY{}
}
\NewEnviron{solution}{
    \noindent\textbf{Solution \theproblems:} \BODY{}
}
\usepackage{breqn}
\everymath{\displaystyle}
% Some may argue that the use of displaystyle in inline formulas may
% break line spacing and make the document look ugly. However, I
% prefer to follow the choice of the textbook setting

% ==================================================
\newcommand{\i}{\mathrm{i}}
\newcommand{\e}{\mathrm{e}}
