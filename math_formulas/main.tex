% This is a mathematical formula memo
\documentclass{article}
\usepackage{amsmath}
\usepackage{amssymb}
\usepackage{amsfonts}
\usepackage{geometry}
\usepackage{ctex}

\setlength{\parindent}{0pt}
\geometry{a4paper,scale=0.8}

\DeclareMathOperator{\arccosh}{arccosh}
\DeclareMathOperator{\arcsinh}{arcsinh}
\DeclareMathOperator{\arctanh}{arctanh}
\usepackage[UTF8, scheme=plain,heading=true,punct=quanjiao]{ctex}
\usepackage{amsthm}
\usepackage{amsmath}
\usepackage{amssymb}
\usepackage{booktabs}  % 提供更专业的表格线条
\usepackage{array}     % 增强的表格功能

\usepackage{breqn}
% Use verb inside other environments
\usepackage{fancyvrb}
\everymath{\displaystyle}
% Some may argue that the use of displaystyle in inline formulas may
% break line spacing and make the document look ugly. However, I
% prefer to follow the choice of the textbook setting
\usepackage{hyperref}
\usepackage{mathrsfs}

\AtBeginDocument{%
    \renewcommand\Re{\operatorname{Re}}
    \renewcommand\Im{\operatorname{Im}}
}

\hypersetup{
    colorlinks=true,
    linkcolor=blue,
    filecolor=magenta,
    urlcolor=cyan,
    pdftitle={Title},
}
\theoremstyle{plain}
\theoremstyle{definition}
\newtheorem{theorem}{Theorem}
\newtheorem{definition}{Definition}[chapter]
\newtheorem{problem}{Problem}[chapter]
\newcommand{\ii}{\mathrm{i}}
\newcommand{\ee}{\mathrm{e}}
% TODO: Consider using `\R' for `\mathbb{R}q`
% TODO: Still configurable
\setlength{\parskip}{5pt}

% TODO: Modify the `proof' environment

\SaveVerb{smile}|:-)|
\SaveVerb{laugh}|:-D|
\SaveVerb{wink}|;-)|
\SaveVerb{sad}|:-(|
\SaveVerb{angry}|:-@|
\SaveVerb{confused}|:-S|
\SaveVerb{cool}|B-)|
\SaveVerb{cry}|:'-|
\SaveVerb{kiss}|:-*|
\SaveVerb{surprised}|:-o|
\SaveVerb{grin}|:-]|
% There is a big conflict between Chinese punctuation and
% English punctuation
% If we use Chinese punctuation only, the formatter won't
% work recognize the break
% If we use Chinese punctuation with space behind, the
% source code will be ugly
% If we use English punctuation, the output will be ugly
\setCJKmainfont[BoldFont=SimHei,ItalicFont=FangSong,BoldItalicFont=LiSu]{SimSun}

\begin{document}

\begin{alignat*}{3}
    \arcsinh'x & =\frac{1}{\sqrt{x^{2}+1}} & \quad &
    \longleftrightarrow \quad & \arcsin'x &
    =\frac{1}{\sqrt{1-x^{2}}}  \\
    \arccosh'x & =\frac{1}{\sqrt{x^{2}-1}} &       &
    \longleftrightarrow       & \arccos'x &
    =-\frac{1}{\sqrt{1-x^{2}}} \\
    \arctanh'x & =\frac{1}{1-x^{2}}        &       &
    \longleftrightarrow       & \arctan'x & =\frac{1}{x^{2}+1}
\end{alignat*}

\begin{align*}
    \sin \theta \cos \varphi & = \frac{\sin(\theta + \varphi) +
    \sin(\theta - \varphi)}{2} \\
    \cos \theta \sin \varphi & = \frac{\sin(\theta + \varphi) -
    \sin(\theta - \varphi)}{2} \\
    \sin \theta \sin \varphi & = \frac{\cos(\theta - \varphi) -
    \cos(\theta + \varphi)}{2} \\
    \cos \theta \cos \varphi & = \frac{\cos(\theta - \varphi) +
    \cos(\theta + \varphi)}{2}
\end{align*}

\begin{align*}
    \sin \theta + \sin \varphi & = 2 \sin \left( \frac{\theta +
    \varphi}{2} \right) \cos \left( \frac{\theta -
    \varphi}{2} \right)  \\
    \sin \theta - \sin \varphi & = 2 \cos \left( \frac{\theta +
    \varphi}{2} \right) \sin \left( \frac{\theta -
    \varphi}{2} \right)  \\
    \cos \theta + \cos \varphi & = 2 \cos \left( \frac{\theta +
    \varphi}{2} \right) \cos \left( \frac{\theta -
    \varphi}{2} \right)  \\
    \cos \theta - \cos \varphi & = -2 \sin \left( \frac{\theta +
    \varphi}{2} \right) \sin \left( \frac{\theta - \varphi}{2} \right)
\end{align*}

\begin{align*}
    \arcsinh x & = \ln \left( x + \sqrt{x^2 + 1} \right)
    \\
    \arccosh x & = \ln \left( x + \sqrt{x^2 - 1} \right)
    \\
    \arccosh x & = \frac{1}{2} \ln \left( \frac{1 + x}{1 - x} \right)
\end{align*}

\begin{align*}
    \sinh x=\sqrt{\frac{\tanh^{2}x}{1-\tanh^{2}x}} \\
    \cosh x=\sqrt{\frac{1}{1-\tanh^{2}x}}          \\
\end{align*}

\begin{align*}
    \sinh' x & = \cosh x                \\
    \cosh' x & = \sinh x                \\
    \tanh' x & = \frac{1}{\cosh^{2}x}   \\
    \coth' x & =- \frac{1}{\sinh^{2} x}
\end{align*}

\begin{align*}
    \arcsinh' x & =\frac{1}{\sqrt{x^{2}+1}} \\
    \arccosh' x & =\frac{1}{\sqrt{x^{2}-1}} \\
    \arctanh' x & =\frac{1}{1-x^{2}}
\end{align*}

\begin{align*}
    \mathrm{d} xy
    & =x\mathrm{d}y+y\mathrm{d}x                     \\
    \mathrm{d} \left( \frac{y}{x} \right)
    & =\frac{x\mathrm{d}y-y\mathrm{d}x}{x^{2}}       \\
    \mathrm{d} \ln \left( \frac{y}{x} \right)
    & =\frac{x\mathrm{d}y-y\mathrm{d}x}{xy}          \\
    \mathrm{d} \arctan \left( \frac{y}{x} \right)
    & =\frac{x\mathrm{d}y-y\mathrm{d}x}{x^{2}+y^{2}} \\
    \mathrm{d} \ln \left| \frac{y - x}{y + x} \right|=\mathrm{d}
    \arctanh \left( \frac{y}{x} \right) &
    =\frac{x\mathrm{d}y-y\mathrm{d}x}{x^{2}-y^{2}} \\
\end{align*}

\[\arcsin x +\arccos x =\frac{\pi}{2} \]
\(\dfrac{\dfrac{\partial M}{\partial y} - \dfrac{\partial N}{\partial
x}}{N}\)只与\(x\)有关 \\
\(\dfrac{\dfrac{\partial N}{\partial x} - \dfrac{\partial M}{\partial
y}}{M}\)只与\(y\)有关 \\

\begin{align*}
    \int \e^{ax} \sin bx \mathrm{d}x & =\frac{a\e^{ax} \sin bx -
    b\e^{ax}\cos bx}{a^{2}+b^{2}} \\
    \int \e^{ax} \cos bx \mathrm{d}x & =\frac{a\e^{ax} \sin bx +
    b\e^{ax}\cos bx}{a^{2}+b^{2}} \\
\end{align*}
\end{document}

Fubini's theorem:

https://en.wikipedia.org/wiki/Fubini%27s_theorem#Proofs
\[
    \int_{X\times Y}  f(x, y) \mathrm{d}x\mathrm{d}y = \int_{X} \left(
    \int_{Y} f(x, y) \mathrm{d}y \right) \mathrm{d}x = \int_{Y} \left(
    \int_{X} f(x, y) \mathrm{d}x \right) \mathrm{d}y
\]

Leibniz integral rule

\begin{aligned}{
    \frac {d}{dx}}\left(\int
    _{a(x)}^{b(x)}f(x,t)\,dt\right)\\=f{\big
    (}x,b(x){\big )}\cdot {\frac
    {d}{dx}}b(x)-f{\big (}x,a(x){\big )}\cdot
    {\frac {d}{dx}}a(x)+\int _{a(x)}^{b(x)}{\frac
    {\partial }{\partial x}}f(x,t)\,dt
\end{aligned}

%https://en.wikipedia.org/wiki/Leibniz_integral_rule#Proofs