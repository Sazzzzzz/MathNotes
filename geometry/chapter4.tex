% !TeX root = main.tex
\chapter{\(n\)维空间的欧式几何}
\section{Finite Subgroups in \(\Isom \R^n\)}

\begin{theorem}{有限子群必有不动点}
    所有\(\Isom \R^{n}\) 的有限子群\(G\)都有不动点。
\end{theorem}

\begin{proof}
    记:
    \begin{align*}
        \mathcal{O}(x) &= G(x) = \left\{ f(x) \mid f \in G \right\} \\
        b &=\frac{1}{\abs{G}} \sum_{f \in G} f(x)
    \end{align*}
    则对于任意的\(g(x) = Bx + c \in G\),有
    \begin{align*}
        g(b) &= g\left(\frac{1}{\abs{G}} \sum_{f \in G} f(x)\right) \\
        &= \frac{1}{\abs{G}} \sum_{f \in G} A(f(x)) + c \\
        &= \frac{1}{\abs{G}} \sum_{f \in G} g \circ f(x) \\
        &= \frac{1}{\abs{G}} \sum_{h \in G} h(x) \\
        &= b
    \end{align*}
\end{proof}

抽象代数里这样求平均值的操作还有
\href{https://math.stackexchange.com/questions/804802/uses-of-averaging-an-operator-over-the-elements-of-a-finite-group}{很多}。