% !TeX root = main.tex
\chapter{等距变换群\(\Isom \R^{2}\)}
% Chapter1: 旋转与狄利克雷逼近和. . . 逼近

交换图: 为什么交换图可以表示可结合的结构?
结合律如何证明?

等距变换群的半直积分解(射影几何的特例)

半直积:
半直积是广泛存在的结构,考虑D2n分解为C2 Cn 以及证明,AutZn=Un
半直积的定义
半直积形状为何如此
半直积的几何意义

等距变换的子群

\(\langle R_{\theta} \rangle\cong \Z\)
\(\langle T_{\theta} \rangle\cong \Z\)
\(\langle T_{a,b} \rangle\cong \Z^{2}\)(a b 线性无关)
\(D_{2n}=C_{2}\ltimes C_{n}\)
3.4 Group action

New notations, new toys!

群作用的定义
when studing groups, we always think of them as acting on some set.

抽象代数里指出所有群都同构于置换群的一个子群,为什么不介绍群作用呢?
% SNCF Train Company?
% R Tree?
Orbit(是一个点集)(固定a看作用) and G-orbit decomposition
\[
    X=\bigsqcup_{i=1}^{N} \mathcal{O}_{i}
\]

Stabilizer(一个子群)(定义在群作用上)(固定a看作用)
\[
    \Stab(a)\coloneq \left\{ f \in G \mid f(a)=a \right\}
\]

Orbit-Stabilizer Theorem
\[
    \left| G \right| = \left| \mathcal{O}_{a} \right| \cdot
    \left| \Stab(a) \right|
\]

\[
    G= \bigsqcup_{b\in \mathcal{O}(a)} G_{a,b}
\]

\(A\cap B\) 是子群?
