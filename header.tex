\usepackage[UTF8, scheme=plain,heading=true,punct=quanjiao]{ctex}
\usepackage{amsthm}
\usepackage{amsmath}
\usepackage{amssymb}
\usepackage{breqn}
% Use verb inside other environments
\usepackage{fancyvrb}
\everymath{\displaystyle}
% Some may argue that the use of displaystyle in inline formulas may
% break line spacing and make the document look ugly. However, I
% prefer to follow the choice of the textbook setting
\usepackage{hyperref}
\usepackage{mathrsfs}

\AtBeginDocument{%
    \renewcommand\Re{\operatorname{Re}}
    \renewcommand\Im{\operatorname{Im}}
}

\hypersetup{
    colorlinks=true,
    linkcolor=blue,
    filecolor=magenta,
    urlcolor=cyan,
    pdftitle={Title},
}
\theoremstyle{plain}
\theoremstyle{definition}
\newtheorem{theorem}{Theorem}
\newtheorem{definition}{Definition}[chapter]
\newtheorem{problem}{Problem}[chapter]
\newcommand{\ii}{\mathrm{i}}
\newcommand{\ee}{\mathrm{e}}
% TODO: Consider using `\R' for `\mathbb{R}q`
% TODO: Still configurable
\setlength{\parskip}{5pt}

% TODO: Modify the `proof' environment

\SaveVerb{smile}|:-)|
\SaveVerb{laugh}|:-D|
\SaveVerb{wink}|;-)|
\SaveVerb{sad}|:-(|
\SaveVerb{angry}|:-@|
\SaveVerb{confused}|:-S|
\SaveVerb{cool}|B-)|
\SaveVerb{cry}|:'-|
\SaveVerb{kiss}|:-*|
\SaveVerb{surprised}|:-o|
\SaveVerb{grin}|:-]|
% There is a big conflict between Chinese punctuation and
% English punctuation
% If we use Chinese punctuation only, the formatter won't
% work recognize the break
% If we use Chinese punctuation with space behind, the
% source code will be ugly
% If we use English punctuation, the output will be ugly
\setCJKmainfont[BoldFont=SimHei,ItalicFont=FangSong,BoldItalicFont=LiSu]{SimSun}
