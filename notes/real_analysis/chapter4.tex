% !TeX root = main.tex
\chapter{可测函数}
\section{Egorov定理}
\begin{theorem}{Egorov定理}
    设\(mE< \infty \),\(\left\{ f_{n} \right\}\) 是\(E\)
    上一列\(\ae\) 收敛于一个\(\ae\) 有限的可测函数,则对于任意的\(\delta > 0\),存在一个可测集\(E_{\delta}
    \subset E\),
    使得\(m(E \setminus E_{\delta}) < \delta\),并且在\(E_{\delta}\)上,\(\left\{ f_{n}
    \right\}\)一致收敛于该函数。
\end{theorem}

\subsection{为什么Egorov定理重要?}
% TODO: 一致收敛和逐点收敛的两次表述
% 积分当中的用途,数分反例的解决
\subsection{Egorov定理的证明}
用\(E\setminus E_{0}\) 代替\(E\) 。
\(f_{n}\to f\)在\(E\setminus E_{0}\)上几乎处处收敛,应有:\[
    \E{\lim_{n \to \infty} \left( f_{n}-f \right) \neq 0}
    = \bigcup_{k=1}^{\infty} \bigcap_{N=1}^{\infty}
    \bigcup_{n=N}^{\infty} \E{\abs{f_{n}-f} \geq \frac{1}{k}}.
\]