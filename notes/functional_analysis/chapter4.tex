% !TeX root = main.tex
\chapter{Continuous Linear Operators}
\subsection{Basic Definitions and Properties}
% TODO: 柯西方程及其反例,不连续的线性算子
% TODO: 复分析怎么解释导数算子不连续?
% TODO: p77例子
本节主要介绍连续线性算子的等价定义和判定条件,以及连续线性算子全体构成赋范线性空间这一性质。

\begin{definition}
    有界:集合\(A\) 有界,如果存在常数 \(M > 0\),使得\(A \subset M U_{E}\)
\end{definition}

\begin{theorem}[连续线性算子的等价条件]
    设 \(X, Y\) 是赋范线性空间,\(T: X \to Y\) 是线性算子。则下列条件等价:
    \begin{enumerate}
        \item \(T\) 是 Lipschitz 连续的;
        \item \(T\) 是连续的;
        \item \(T\) 在 \(0\) 点连续;
        \item \(T(U_{E})\) 有界
        \item \(T\) 将有界集映射为有界集(即 \(T\) \textbf{局部有界});
        \item 存在 \(c \ge 0\),使得 \(\|Tx\| \le c\|x\|,
            \forall x \in X\)(\(T\) 是\textbf{有界算子})。
    \end{enumerate}
\end{theorem}

\begin{proof}
    \begin{itemize}
        \item 由连续性推导有界性:利用连续性定义,在 \(0\) 点找一个小球,其像包含在 \(Y\) 的单位球内,
            开集中总能找到新的开集包含在该像内,从而说明有界性。
        \item 从 (5) 推导 (6):扩大单位球
        \item 从 (6) 推导 (1):利用 \(x-y\) 和线性性。
    \end{itemize}
\end{proof}

同线性代数中矩阵范数类似:

\begin{theorem}[算子范数的等价定义]
    定义:
    \[ \|T\| \coloneqq  \sup_{\|x\| \leq 1} \|Tx\| \]
    设 \(T \in B(X, Y)\),则
    \[ \|T\| = \sup_{x \ne 0} \frac{\|Tx\|}{\|x\|} = \sup_{\|x\| \le
    1} \|Tx\| = \sup_{\|x\| = 1} \|Tx\|. \]
\end{theorem}

\begin{proof}
    \[
        Tu = \norm{u} T \frac{u}{\norm{u}}, \forall u \ne 0.
    \]
    \begin{align*}
        \sup_{\norm{x} < 1} \norm{Tx}&= \sup_{\norm{x}<1} \norm{x} \norm{T
        \frac{x}{\norm{x}}} \leq \norm{x} \sup_{\norm{x}=1} \norm{T
        x} < \sup_{\norm{x} = 1} \norm{T x}\\
        \sup_{\norm{x} =1} \norm{Tx} &= (1+\varepsilon)
        \sup_{\norm{x} =1} \frac{\norm{Tx}}{1+\varepsilon} \leq
        (1+\varepsilon) \sup_{\norm{x}<1} \norm{Tx}, \forall \varepsilon >0\\
    \end{align*}
    因此 \(\sup_{\norm{x} < 1} \norm{Tx} = \sup_{\norm{x} = 1} \norm{T x}\)。
    \[
        \inf \left\{ k \mid k\geq \frac{\norm{Tx}}{\norm{x}} \right\}
        = \sup \frac{\norm{Tx}}{\norm{x}} = \sup_{\norm{x}=1} \norm{Tx} = \|T\|.
    \]
\end{proof}

类比矩阵构成的线性空间\(\mathscr{L}(V,W)\):
\begin{theorem}
    \(\mathbf{B}(X, Y)\) 构成赋范线性空间。
\end{theorem}

\begin{proof}
    \[
        \norm{\alpha Tx} = \sup_{x\in E} \norm{\alpha T x} =
        |\alpha| \sup_{x\in E}
        \norm{T x} = |\alpha| \norm{T}.
    \]
    其他性质类似。
\end{proof}

\begin{theorem}
    \(T: K \mapsto H\) 是 Hilbert 空间之间的有界线性算子,则:
    \[
        \norm{T} = \sup \left\{ \abs{\inner{Th}{k}} \mid k\in U_{k},
        h\in U_{h} \right\}
    \]
\end{theorem}

\begin{proof}
    本质是说自己与自己的内积最大。
    \[
        \abs{\inner{Th}{k}} \leq \norm{Th} \norm{k} \leq \norm{T}
    \]
    \[
        \norm{Th} = \inner{Th}{Th} = \sup_{\norm{k} \leq 1} \abs{\inner{Th}{k}}
    \]
\end{proof}

\begin{example}[积分算子]
    设 \(X = C[0, 1]\) 配备 sup 范数。定义 \(T: X \to \mathbb{F}\) 为
    \[ T(f) = \int_0^1 f(t) dt. \]
    则 \(T\) 是有界线性算子且 \(\|T\| = 1\)。
    \begin{proof}
        求连续线性算子范数的方法一般是给出一个上界,并证明这个上界可达。
        首先,
        \[ |T(f)| = \left| \int_0^1 f(t) dt \right| \le \int_0^1
        |f(t)| dt \le \|f\|_\infty. \]
        这表明 \(\|T\| \le 1\)。
        另一方面,取 \(f(t) \equiv 1\),则 \(\|f\|_\infty = 1\) 且 \(T(f) = 1\)。
        因此 \(\|T\| \ge |T(f)|/\|f\| = 1\)。
        综上,\(\|T\| = 1\)。
    \end{proof}
\end{example}

\begin{example}[导数算子]
    求导算子不是有界算子。

    设 \(X = C^1[0, 1]\), \(Y = C[0, 1]\),均配备 sup 范数。定义 \(T: X \to Y\)
    为 \(T(f) = f'\)。

    考虑 \(f_n(t) = t^n\) (\(n \in \mathbb{N}\))。则 \(\|f_n\|_\infty = 1\)。
    但是 \(T(f_n) = n t^{n-1}\),故 \(\|T(f_n)\|_\infty = n\)。
    这表明 \(T\) 在单位球上无界,因此不是有界算子。
\end{example}
