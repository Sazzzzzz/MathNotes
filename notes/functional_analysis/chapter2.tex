% !TEX root = main.tex
\chapter{Banach Spaces}
\section{Normed Spaces}

% fix notation and $$ to \[ \]
\begin{definition}
    设$X$是一个线性空间,$\|\cdot\|:X\to [0,+\infty)$是一个映射,如果对任意$x,y\in X$和标量$\alpha$,都有
    \begin{enumerate}
        \item $\|x\|=0 \Leftrightarrow x=0$;
        \item $\|\alpha x\|=|\alpha|\|x\|$;
        \item $\|x+y\|\leq \|x\|+\|y\|$;
    \end{enumerate}
    则称$\|\cdot\|$为$X$上的一个范数,$(X,\|\cdot\|)$称为一个赋范空间(normed space)。
\end{definition}

通过范数我们可以定义度量:

\begin{definition}{范数诱导出的度量}
    设$(X,\|\cdot\|)$是一个赋范空间,定义$d:X\times X\to [0,+\infty)$为
    \[
        d(x,y)=\|x-y\|
    \]
    则$d$是$X$上的一个度量,$(X,d)$是一个度量空间。
\end{definition}

故赋范线性空间必然是度量空间。

这样定义出来的度量具有平移不变性:\[
    d(x+z,y+z)=d(x,y)
\]

一般的度量空间并不定义在线性空间上,所以度量空间是比赋范空间更一般的概念\footnote{本书中的度量空间都定义在线性空间上}。
即使是定义在线性空间上的度量,也不一定是由范数诱导出来的。
考虑离散度量\(d(x,y)\) :
\[
    d(x,y)=
    \begin{cases}
        0 & x=y\\
        1 & x\neq y
    \end{cases}
\]

考虑范数\( \|x\| = d(x,0)\):则:
\[
    \|2\| = d(2,0) = 1 \neq 2 d(1,0) = 2 \|1\|=2
\]

一般的,度量空间由范数诱导当且仅当:
\begin{itemize}
    \item 度量具有平移不变性:\(d(x+z,y+z)=d(x,y)\);
    \item 度量具有齐次性:\(d(\alpha x, \alpha y) = |\alpha| d(x,y)\)。
\end{itemize}