% !TEX root = main.tex
\chapter{Banach Spaces}
\section{Normed Spaces}

% fix notation and $$ to \[ \]
\begin{definition}
    设$X$是一个线性空间,$\|\cdot\|:X\to [0,+\infty)$是一个映射,如果对任意$x,y\in X$和标量$\alpha$,都有
    \begin{enumerate}
        \item $\|x\|=0 \Leftrightarrow x=0$;
        \item $\|\alpha x\|=|\alpha|\|x\|$;
        \item $\|x+y\|\leq \|x\|+\|y\|$;
    \end{enumerate}
    则称$\|\cdot\|$为$X$上的一个范数,$(X,\|\cdot\|)$称为一个赋范空间(normed space)。
\end{definition}

通过范数我们可以定义度量:

\begin{definition}{范数诱导出的度量}
    设$(X,\|\cdot\|)$是一个赋范空间,定义$d:X\times X\to [0,+\infty)$为
    \[
        d(x,y)=\|x-y\|
    \]
    则$d$是$X$上的一个度量,$(X,d)$是一个度量空间。
\end{definition}

故赋范线性空间必然是度量空间。

这样定义出来的度量具有平移不变性:\[
    d(x+z,y+z)=d(x,y)
\]

一般的度量空间并不定义在线性空间上,所以度量空间是比赋范空间更一般的概念\footnote{本书中的度量空间都定义在线性空间上}。
即使是定义在线性空间上的度量,也不一定是由范数诱导出来的。
考虑离散度量\(d(x,y)\) :
\[
    d(x,y)=
    \begin{cases}
        0 & x=y\\
        1 & x\neq y
    \end{cases}
\]

考虑范数\( \|x\| = d(x,0)\):则:
\[
    \|2\| = d(2,0) = 1 \neq 2 d(1,0) = 2 \|1\|=2
\]

一般的,度量空间由范数诱导当且仅当:
\begin{itemize}
    \item 度量具有平移不变性:\(d(x+z,y+z)=d(x,y)\);
    \item 度量具有齐次性:\(d(\alpha x, \alpha y) = |\alpha| d(x,y)\)。
\end{itemize}

\section{Banach Spaces}

\begin{definition}{Banach Space}
    设\((X,\|\cdot\|)\)是一个赋范空间,若\(X\) 完备,即任意Cauchy列在\(X\) 中收敛,则称\(X\)
    为一个Banach空间。
\end{definition}

\begin{example}{}
    \begin{itemize}
        \item 连续函数空间\(C[a,b]\) 不是Banach空间:连续函数的极限是可测函数(考虑\(f_{n}(x) =
            x^{n}\) ),不一定是连续函数。
        \item \(\ell_{p}\),\(L_{p}[a,b]\) 是Banach空间。
    \end{itemize}
\end{example}

\begin{theorem}{}
    不存在可数维的Banach空间。
\end{theorem}

\begin{proof}
    设\(E\) 是一个可数维的Banach空间,\(\{e_{i}\}_{i=1}^{\infty}\) 是\(E\) 的一组基。
    记\(E_{n} = \Span\{e_{1},\ldots,e_{n}\}\) ,有:\[
        E = \bigcup_{n=1}^{\infty} E_{n}
    \]

    由于\(E_{n}\) 是闭的子空间,由Baire纲定理,存在某个\(E_{n_{0}}\) ,使得\(E_{n_{0}}\) 有非空的内点:
    \[
        B(x_{0},r) \subset E_{n_{0}}
    \]
    从而:\[
        E = \bigcup_{k=1}^{\infty} k B(x,r) \subset
        \bigcup_{k=1}^{\infty} kE_{n_{0}} = E_{n_{0}}
    \]
    于是\(E = E_{n_{0}}\) ,与\(E\) 可数维矛盾。
\end{proof}

由此立刻可得:
\begin{example}{}
    \begin{itemize}
        \item 多项式空间不是Banach空间。
        \item
            显然\(\left\{ 1, x, x^{2}, \ldots \right\}\) 是一个可数基。故多项式空间不是Banach空间。
            考虑多项式列\[
                P_{n}(x) = \sum_{k=0}^{n} \frac{x^{k}}{k!}
            \]
            有\(\lim_{n \to \infty} P_{n}(x) = \e^{x}\) ,不再是多项式。

        \item \(c_{00}\) 不是Banach空间。

            显然\(\left\{ e_{1}, e_{2}, \ldots \right\}\) 是一个可数基。
            故\(c_{00}\) 不是Banach空间。
            考虑序列列\[
                x_{n} = \left(1, \frac{1}{2}, \ldots, \frac{1}{n}, 0,
                0, \ldots \right)
            \]
            有\(\lim_{n \to \infty} x_{n} = \left( 1, \frac{1}{2},
            \ldots  \right) \notin c_{0}\) ,不再是\(c_{00}\) 中的元素。
    \end{itemize}
\end{example}

\begin{theorem}{}
    Banach空间的闭子空间是Banach空间。
\end{theorem}

\begin{theorem}{}
    有限维内积空间是Banach空间。
\end{theorem}

\begin{proof}
    有限维内积空间同构于\((\R^{n} , \|\cdot\|_{2})\) ,而\((\R^{n} ,
    \|\cdot\|_{2})\) 是Banach空间。
\end{proof}

由此可知,任意内积空间的有限维子空间都是Banach空间,也就是闭子空间。

另外,Banach空间保证所有的Cauchy列都收敛,但不保证所有序列都有收敛子列,也就是并非所有Banach空间都是紧的。
考虑\(\ell_{2}\) 上的一个序列列:
\[
    x_{n} = (\underbrace{0,\ldots,0}_{n-1\text{个}} ,1,0,\ldots )
\]
则  \(\|x_{n}-x_{m}\|_{2} = \sqrt{2}\) ,所以该序列列没有收敛子列。

\section{Separable Banach Spaces}
\begin{definition}{可分空间}
    设\(E\) 是一个度量空间,如果\(E\) 中存在一个可数子集\(A\) ,使得\(A\) 在\(E\) 中稠密,则称\(E\) 是一个可分空间。
\end{definition}

只要Banach空间\(E\) 中任意元素都可以用某个可数子集\(A\) 的元素来逼近,则该Banach空间是可分的。
这部分常用的工具有:可变长的有限长有理数组是可数的。\[
    \abs{\bigcup_{n=1}^{\infty} \mathbb{Q}^{n}} = \abs{\mathbb{N}
    \times \mathbb{Q}} = \abs{\mathbb{N}}
\]

\begin{example}{}
    \begin{itemize}
        \item \(c_{0}\) 是可分的:考虑\(\left\{ (r_{1},\dots r_{N}, r_{0},
            \dots ) \mid r_{0},\dots r_{N} \in \mathbb{Q} \right\}\)
        \item \(\ell_{p}\) 是可分的:注意到\(\ell_{1} \subset \ell_{2}
                \subset \cdots \subset \ell_{p} \subset \dots c_{0}
            \subset \ell_{\infty}\)
        \item \(L^{p}(a,b)\) 是可分的:\(\text{有理系数多项式} \to \text{连续函数}
            \to f_{0} \to f\) 。其中\(f_{0}\) 为:\[
                f_{0}(x) =
                \begin{cases}
                    f(x) & f(x) \leq n_{0}\\
                    0 & f(x) > n_{0}
                \end{cases}
            \]
            其中\(\mu(\left\{ t \in [a,b] \mid \abs{f(t)} > n_{0}
            \right\})< \delta\) 。在这个意义下,\(f_{0}\) 与\(f\) 的距离小于\(\varepsilon\) 。
        \item \(\ell_{\infty}\) 和 \(L^{\infty}(a,b)\) 是不可分的

            证明不可分只需要证明空间中存在不可数子集,且子集中任意两点的距离都大于某个正数。
            例如考虑\(\ell_{\infty}\) 中的子集\[
                A = \{0,1\}^{\mathbb{N}} = \{ (a_{1},a_{2},\ldots)
                \mid a_{i} = 0 \text{或} 1 \}
            \]
    \end{itemize}
\end{example}

\begin{definition}{Schauder Basis}
    设\(X\) 是一个赋范空间,如果存在\(X\) 中的一个可数子集\(\{e_{i}\}_{i=1}^{\infty}\) ,
    使得对于任意\(x \in X\) ,都存在唯一的一组标量\(\{x_{i}\}_{i=1}^{\infty}\) ,使得
    \[
        x = \sum_{i=1}^{\infty} x_{i} e_{i}
    \]
    则称\(\{e_{i}\}_{i=1}^{\infty}\) 为\(X\) 的Schauder基。
\end{definition}

\begin{theorem}{}
    设\(X\) 是一个赋范空间,如果\(X\) 有Schauder基,则\(X\) 是可分的。
\end{theorem}

\begin{proof}
    \[
        A = \left\{ \sum_{i=1}^{N} r_{i} e_{i} \mid r_{i} \in
        \mathbb{Q}, N \in \mathbb{N} \right\}
    \]
    则\(A\) 是\(X\) 的可数稠集。
\end{proof}