% !TEX root = main.tex
\chapter{Hilbert Spaces}

希尔伯特空间(Hilbert Spaces)是有限维欧几里得空间 \(\mathbb{R}^n\) 的自然推广。在
\(\mathbb{R}^n\) 中,向量不仅有“长度”(范数),还有“角度”(由内积定义)。希尔伯特空间就是装备了内积的完备赋范线性空间,其范数由内积诱导。

\section{Basic properties}

关于内积空间的基本定义、柯西-施瓦茨不等式(Cauchy-Schwarz inequality)、勾股定理、三角不等式等性质,
与线性代数中的内容基本一致,不再赘述。

\subsection{内积的连续性}

\begin{theorem}[Continuity of inner product]
    设 \(x_n, y_n\) 是内积空间 \(E\) 中的元素,\(n=1, 2, \dots\),且
    \[
        x_n \to x, \quad y_n \to y,
    \]
    即在诱导范数拓扑下收敛。则
    \[
        \langle x_n, y_n \rangle \to \langle x, y \rangle.
    \]
\end{theorem}

\begin{proof}
    因为 \(y_n \to y\),存在常数 \(C > 0\) 使得 \(\|y_n\| \le C\) 对所有 \(n \in
    \mathbb{N}\) 成立。
    由三角不等式和柯西-施瓦茨不等式:
    \begin{align*}
        |\langle x_n, y_n \rangle - \langle x, y \rangle|
        &= |\langle x_n, y_n \rangle - \langle x_n, y \rangle +
        \langle x_n, y \rangle - \langle x, y \rangle| \\
        &\le |\langle x_n, y_n - y \rangle| + |\langle x_n - x, y \rangle| \\
        &\le \|x_n\| \|y_n - y\| + \|x_n - x\| \|y\| \\
        &\le C \|y_n - y\| + \|x_n - x\| \|y\|.
    \end{align*}
    当 \(n \to \infty\) 时,上式趋于 \(0\)。
\end{proof}

\subsection{平行四边形定则}

平行四边形定则(Parallelogram Law)揭示了范数与内积之间的深刻联系。如果一个范数满足平行四边形定则,
那么它一定是由某个内积诱导的。这是因为平行四边形定则可以推导出极化恒等式(Polarization Identity)。

\begin{theorem}[Parallelogram Law]
    向量空间 \(E\) 上的范数 \(\|\cdot\|\) 是由内积定义的,当且仅当
    \[
        \|x+y\|^2 + \|x-y\|^2 = 2(\|x\|^2 + \|y\|^2), \quad \forall x, y \in E.
    \]
\end{theorem}

如果范数满足平行四边形定则,我们可以通过极化恒等式来定义内积:
\begin{itemize}
    \item 实情形:
        \[
            \langle x, y \rangle = \frac{1}{4} (\|x+y\|^2 - \|x-y\|^2).
        \]
    \item 复情形:
        \[
            \langle x, y \rangle = \frac{1}{4} \sum_{k=0}^3 i^k \|x +
            i^k y\|^2 = \frac{1}{4} (\|x+y\|^2 - \|x-y\|^2 +
            i\|x+iy\|^2 - i\|x-iy\|^2).
        \]
\end{itemize}

\subsection{希尔伯特空间的直和}

我们可以将两个内积空间“拼”成一个更大的内积空间,这称为希尔伯特空间的直和(Direct Sum)。

\begin{example}
    设 \(H\) 和 \(K\) 是内积空间,内积分别为 \(\langle \cdot, \cdot \rangle_H\) 和
    \(\langle \cdot, \cdot \rangle_K\)。则 \(H \times K\) 也是一个内积空间,其内积定义为:
    \[
        \langle (x, \xi), (y, \eta) \rangle = \langle x, y \rangle_H
        + \langle \xi, \eta \rangle_K,
    \]
    其中 \(x, y \in H\) 且 \(\xi, \eta \in K\)。
\end{example}

在这种情况下,我们将 \(H \times K\) 记为 \(H \oplus K\),称为希尔伯特空间 \(H\) 和 \(K\) 的直和。

由上述内积诱导的范数为:
\[
    \|(x, \xi)\|_{H \oplus K} = \sqrt{\|x\|_H^2 + \|\xi\|_K^2}.
\]

显然,如果 \(H\) 和 \(K\) 都是完备的(即希尔伯特空间),那么 \(H \oplus K\) 在此范数下也是完备的。

值得注意的是,高维空间往往可以看作低维空间的直和。例如,\(\mathbb{R}^n\) 实际上就是 \(n\) 个
\(\mathbb{R}\) 作为希尔伯特空间的直和:
\[
    \mathbb{R}^n = \underbrace{\mathbb{R} \oplus \mathbb{R} \oplus
    \cdots \oplus \mathbb{R}}_{n \text{个}}.
\]

\section{Best Approximation}