% !TEX root = main.tex
\chapter{Hilbert Spaces}

希尔伯特空间(Hilbert Spaces)是有限维欧几里得空间 \(\R^n\) 的自然推广。在
\(\R^n\) 中,向量不仅有“长度”(范数),还有“角度”(由内积定义)。希尔伯特空间就是装备了内积的完备赋范线性空间,其范数由内积诱导。

\section{Basic properties}

关于内积空间的基本定义、柯西-施瓦茨不等式(Cauchy-Schwarz inequality)、勾股定理、三角不等式等性质,
与线性代数中的内容基本一致,不再赘述。

\subsection{内积的连续性}

\begin{theorem}[Continuity of inner product]
    设 \(x_n, y_n\) 是内积空间 \(E\) 中的元素,\(n=1, 2, \dots\),且
    \[
        x_n \to x, \quad y_n \to y,
    \]
    即在诱导范数拓扑下收敛。则
    \[
        \inner{x_n}{y_n} \to \inner{x}{y}.
    \]
\end{theorem}

\begin{proof}
    因为 \(y_n \to y\),存在常数 \(C > 0\) 使得 \(\|y_n\| \le C\) 对所有 \(n \in
    \mathbb{N}\) 成立。
    由三角不等式和柯西-施瓦茨不等式:
    \begin{align*}
        |\inner{x_n}{y_n} - \inner{x}{y}|
        &= |\inner{x_n}{y_n} - \inner{x_n}{y} +
        \inner{x_n}{y} - \inner{x}{y}| \\
        &\le |\inner{x_n}{y_n - y}| + |\inner{x_n - x}{y}| \\
        &\le \|x_n\| \|y_n - y\| + \|x_n - x\| \|y\| \\
        &\le C \|y_n - y\| + \|x_n - x\| \|y\|.
    \end{align*}
    当 \(n \to \infty\) 时,上式趋于 \(0\)。
\end{proof}

\subsection{平行四边形定则}

平行四边形定则(Parallelogram Law)揭示了范数与内积之间的深刻联系。如果一个范数满足平行四边形定则,
那么它一定是由某个内积诱导的。这是因为平行四边形定则可以推导出极化恒等式(Polarization Identity)。

\begin{theorem}[Parallelogram Law]
    向量空间 \(E\) 上的范数 \(\|\cdot\|\) 是由内积定义的,当且仅当
    \[
        \|x+y\|^2 + \|x-y\|^2 = 2(\|x\|^2 + \|y\|^2), \quad \forall x, y \in E.
    \]
\end{theorem}

如果范数满足平行四边形定则,可以通过极化恒等式来定义内积:
\begin{itemize}
    \item 实情形:
        \[
            \inner{x}{y} = \frac{1}{4} (\|x+y\|^2 - \|x-y\|^2).
        \]
    \item 复情形:
        \[
            \inner{x}{y} = \frac{1}{4} \sum_{k=0}^3 i^k \|x +
            i^k y\|^2 = \frac{1}{4} (\|x+y\|^2 - \|x-y\|^2 +
            i\|x+iy\|^2 - i\|x-iy\|^2).
        \]
\end{itemize}

\subsection{希尔伯特空间的直和}

我们可以将两个内积空间“拼”成一个更大的内积空间,这称为希尔伯特空间的直和(Direct Sum)。

\begin{example}
    设 \(H\) 和 \(K\) 是内积空间,内积分别为 \(\inner{\cdot}{\cdot}_H\) 和
    \(\inner{\cdot}{\cdot}_K\)。则 \(H \times K\) 也是一个内积空间,其内积定义为:
    \[
        \inner{\langle (x, \xi)}{(y, \eta)}= \inner{x}{y}_H
        + \inner{\xi}{\eta}_K,
    \]
    其中 \(x, y \in H\) 且 \(\xi, \eta \in K\)。
\end{example}

在这种情况下,我们将 \(H \times K\) 记为 \(H \oplus K\),称为希尔伯特空间 \(H\) 和 \(K\) 的直和。

由上述内积诱导的范数为:
\[
    \|(x, \xi)\|_{H \oplus K} = \sqrt{\|x\|_H^2 + \|\xi\|_K^2}.
\]

显然,如果 \(H\) 和 \(K\) 都是完备的(即希尔伯特空间),那么 \(H \oplus K\) 在此范数下也是完备的。

值得注意的是,高维空间往往可以看作低维空间的直和。例如,\(\R^n\) 实际上就是 \(n\) 个
\(\R\) 作为希尔伯特空间的直和:
\[
    \R^n = \underbrace{\R \oplus \R \oplus
    \cdots \oplus \R}_{n \text{个}}.
\]

\section{Best approximation}
% TODO: 独自证明压缩性质、等价性质、存在唯一性

在希尔伯特空间中,最佳逼近问题是寻找一个闭凸集中与给定点距离最近的点。
本章思路:存在性与唯一性\(\to \) 等价性质 \(\to \) 压缩性。
相比线性代数,这一章所讲的投影更加广泛,不再局限于子空间,而是任意闭凸集。

\subsection{存在性与唯一性}

\begin{theorem}[存在性与唯一性]
    设 \(B\) 是希尔伯特空间 \(H\) 中的非空闭凸集。对于任意 \(x \in H\),存在唯一的 \(\hat{x} \in B\) 使得
    \[
        \|x - \hat{x}\| = d(x, B) \coloneqq \inf_{y \in B} \|x - y\|.
    \]
    我们称 \(\hat{x}\) 为 \(x\) 在 \(B\) 中的\textbf{最佳逼近元}(best
    approximation element)。
\end{theorem}

\begin{proof}
    \textbf{存在性}:设 \(d = d(x, B)\)。由下确界定义,存在序列 \(\{y_n\} \subset B\)
    使得 \(\|x - y_n\| \leq  d + \frac{1}{n}\)。
    利用平行四边形定则,我们可以证明 \(\{y_n\}\) 是柯西序列。
    \begin{align*}
        \|y_n - y_m\|^2 &= 2\|x - y_n\|^2 + 2\|x - y_m\|^2 - 4\left\|x
        - \frac{y_n + y_m}{2}\right\|^2 \\
        &\le 2\left(d + \frac{1}{n}\right)^2 + 2\left(d +
        \frac{1}{m}\right)^2 - 4d^2 \\
        &= \frac{2}{n^2} + \frac{2}{m^2} + 4d\left(\frac{1}{n} +
        \frac{1}{m}\right).
    \end{align*}
    由于 \(B\) 是凸集,\(\frac{y_n + y_m}{2} \in B\),故 \(\|x - \frac{y_n +
    y_m}{2}\| \ge d\)。
    由此可得 \(\|y_n - y_m\| \to 0\)(当 \(n, m \to \infty\))。
    由 \(H\) 的完备性,\(y_n \to \hat{x}\)。由 \(B\) 的闭性,\(\hat{x} \in B\)。
    由范数的连续性,\(\|x - \hat{x}\| = d\)。

    \textbf{唯一性}:若有 \(\hat{x}, \hat{y} \in B\) 均为最佳逼近元,再次利用平行四边形定则可证
    \(\|\hat{x} - \hat{y}\| = 0\),即 \(\hat{x} = \hat{y}\)。
\end{proof}

\subsection{等价性质}

最佳逼近元可以通过变分不等式来刻画。这不仅提供了判定准则,也揭示了深刻的几何意义。

\begin{theorem}[Characterization of Best Approximation]
    设 \(B\) 是希尔伯特空间 \(H\) 中的闭凸集,\(\hat{x} \in B\),\(x \in H \setminus
    B\)。下列条件等价:
    \begin{enumerate}
        \item \(\|x - \hat{x}\| = d(x, B)\);
        \item \(\Re\inner{x - \hat{x}}{b - \hat{x}} \le
            0, \quad \forall b \in B\);
        \item \(\Re\inner{x - b}{\hat{x} - b} \ge 0,
            \quad \forall b \in B\)。
    \end{enumerate}
\end{theorem}

这里出现了实部 \(\Re\),这在复希尔伯特空间中尤为重要。

\subsubsection{几何解释:角度与实部}

在实希尔伯特空间中,条件 (2) \(\inner{x - \hat{x}}{b - \hat{x}} \le 0\) 直观地表示误差向量
\(x - \hat{x}\) 与从 \(\hat{x}\) 出发指向 \(B\) 中任意一点的向量 \(b - \hat{x}\)
之间的夹角是\textbf{钝角}(\(\ge 90^\circ\))。这意味着我们无法通过向 \(b\) 的方向移动来进一步减小距离。

在复希尔伯特空间中,距离(范数)仍然是一个实数。当我们对距离函数求导或进行变分分析时,自然会出现内积的实部。
希尔伯特空间中有两种定义角度的方式:
\begin{itemize}
    \item \textbf{欧几里得角度(Euclidean Angle)}:将复空间 \(\mathbb{C}^n\)
        视为实空间 \(\mathbb{R}^{2n}\),此时角度 \(\phi_E\) 定义为
        \[ \cos \phi_E = \frac{\Re\inner{u}{v}}{\|u\|\|v\|}. \]
        这正是最佳逼近定理中使用的角度概念。条件 \(\Re\inner{u}{v} \le 0\)
        意味着在“实化”的几何视图下,角度 \(\ge 90^\circ\)。
    \item \textbf{埃尔米特角度(Hermitian Angle)}:这是复几何中更自然的定义,关注复共线关系,定义为
        \(\cos \phi_H = \frac{|\inner{u}{v}|}{\|u\|\|v\|}\)。但这不适用于距离最小化问题。
\end{itemize}

因此,定理中的条件 (2) 和 (3) 本质上是在说:\(\hat{x}\) 是最佳逼近元,当且仅当误差向量与集合 \(B\)
的“切线”方向(即 \(b - \hat{x}\))在欧几里得意义下成钝角。

\subsection{投影算子}

\begin{definition}[投影算子]
    设 \(B\) 是希尔伯特空间 \(H\) 的非空闭凸子集。定义映射 \(P_B: H \to B\) 为
    \[ P_B(x) = \hat{x}, \]
    其中 \(\hat{x}\) 是 \(x\) 在 \(B\) 中的唯一最佳逼近元。\(P_B\) 称为向 \(B\)
    的投影算子。
\end{definition}

\begin{corollary}[压缩性质]
    投影算子 \(P_B\) 是连续的,且是压缩映射(contraction),即:
    \[
        \|P_B(x) - P_B(y)\| \le \|x - y\|, \quad \forall x, y \in H.
    \]
\end{corollary}

\section{Orthogonal Decomposition}

这一章主要内容是推导正交性的一些基本性质,证明比较巧妙。
\begin{definition}[Orthogonality]
    设 \(H\) 为内积空间。
    \begin{enumerate}
        \item 若 \(x, y \in H\) 满足 \(\inner{x}{y} = 0\),则称 \(x\) 与
            \(y\) \textbf{正交}(orthogonal),记为 \(x \perp y\)。
        \item 若 \(x \in H\) 与集合 \(S \subset H\) 中的任意元素都正交,则称 \(x\) 与
            \(S\) 正交,记为 \(x \perp S\)。
        \item 若集合 \(S_1, S_2 \subset H\) 满足对任意 \(x \in S_1, y \in
            S_2\) 都有 \(x \perp y\),则称 \(S_1\) 与 \(S_2\) 正交,记为 \(S_1 \perp S_2\)。
        \item 定义 \(S\) 的\textbf{正交补}(orthogonal complement)为:
            \[
                S^\perp = \{y \in H : \inner{y}{x} = 0, \forall x \in S\}.
            \]
    \end{enumerate}
\end{definition}
正交条件下勾股定理成立。

\begin{theorem}
    设 \(S, S_1, S_2\) 为内积空间 \(H\) 的子集。
    \begin{enumerate}
        \item \(x^{\perp} \)是闭子空间
        \item \(S^\perp\) 是 \(H\) 的闭子空间。
        \item \(S^\perp = (\overline{S})^\perp\)。
        \item \(S \subseteq  S^{\perp\perp}\)。
        \item \(S_1 \subset S_2 \implies S_2^\perp \subset S_1^\perp\)。
        \item \(S^\perp = S^{\perp\perp\perp}\)。
    \end{enumerate}
\end{theorem}

\begin{proof}
    \begin{itemize}
        \item (1) 由内积的连续性可知
        \item (2)  \(S^\perp = \bigcap_{x
            \in S} \{x\}^\perp\),所以 \(S^\perp\) 也是闭子空间。
        \item (3) 设 \(x \in S\)。对于任意 \(y \in S^\perp\),有
            \(\inner{y}{x} = 0\),即
            \(\inner{x}{y} = 0\)。
            这表明 \(x\) 与 \(S^\perp\) 中所有元素正交,即 \(x \in (S^\perp)^\perp =
            S^{\perp\perp}\)。

            注意:比如\(S\) 只有上半平面时就不能取等。这里 \(S\) 的正交补的正交补
            \(S^{\perp\perp}\) 被称作 \(S\)
            的\textbf{闭线性张成}(closed linear span)。
        \item (5) 由(4) 有\(S \in S^{\perp\perp}
            \)即\(S^{\perp \perp \perp }
            \subset  S^\perp\)
            由(4) 有\(S^{\perp } \subseteq (S^{\perp })^{\perp \perp } = S^{\perp
            \perp \perp } \)
    \end{itemize}
\end{proof}

\begin{theorem}
    设 \(M\) 是希尔伯特空间 \(H\) 的闭子空间,\(x \in H\)。
    \(y \in M\) 是 \(x\) 在 \(M\) 中的最佳逼近元,当且仅当 \(x\) 与其在 \(M\) 上的投影
    \(y\) 的连线垂直于该子空间,即
    \[
        x - y \perp M.
    \]
\end{theorem}

\begin{theorem}[Orthogonal Decomposition Theorem]
    设 \(M\) 是希尔伯特空间 \(H\) 的闭子空间,则 \(H\) 等于 \(M\) 和 \(M\) 的正交补的直和:
    \[
        H = M \oplus M^\perp.
    \]
\end{theorem}

\begin{example}
    设 \(H = L^2[0, 1]\),\(S = \{f \in H : f(t) = 0 \text{ a.e. } t
    \in [0, 1/2]\}\)。
    则
    \[
        S^\perp = \{g \in H : g(t) = 0 \text{ a.e. } t \in [1/2, 1]\}.
    \]
    且 \(S^{\perp\perp} = S\)。
\end{example}

\section{Orthonormal Basis}

% TODO: 规范正交基问题3.4.7的证明中,为什么假定了希尔伯特空间中元素可以被逼近?
% TODO: 等距映射是双射,当且仅当它是单射或者满射吗?
% TODO: 等距映射在泛函分析中和线性代数中有什么区别?
% TODO: 如何对多项式基进行施密特正交化得到勒让德多项式?
% TODO: 关于等距映射更多性质与LADR的对比
% TODO: 补充希尔伯特空间的分类

在有限维欧几里得空间中,正交基是一个非常核心的概念。在希尔伯特空间中,我们将这一概念推广为规范正交基(Orthonormal Basis)。

\subsection{定义与基本性质}

\begin{definition}[Orthonormal Basis]
    设 \(H\) 是一个希尔伯特空间。如果 \(\mathcal{E}\) 是 \(H\)
    中包含关系下的极大规范正交系,则称 \(\mathcal{E}\) 为 \(H\)
    的一个\textbf{规范正交基}(Orthonormal Basis)。
\end{definition}

根据佐恩引理(Zorn's Lemma),任何非零希尔伯特空间都存在规范正交基。

下面的定理是希尔伯特空间理论中最重要的结果之一,它建立了抽象希尔伯特空间与序列空间 \(l^2\) 之间的联系。

\begin{theorem}[Expansion and Parseval's Identity]
    设 \(\{e_\alpha\}_{\alpha \in A}\) 是希尔伯特空间 \(H\) 的一个规范正交基。则对于任意
    \(x \in H\),下列结论成立:
    \begin{enumerate}
        \item \textbf{展开}:
            \[
                x = \sum_{\alpha \in A} \norm{x}{e_\alpha} e_\alpha.
            \]
            这里的求和是基于网(net)的无条件收敛,即对于任意 \(\epsilon > 0\),存在有限集 \(F_0
            \subset A\),使得对任意包含 \(F_0\) 的有限集 \(F \subset A\),都有 \(\|x
                - \sum_{\alpha \in F} \norm{x}{e_\alpha}
            e_\alpha\| < \epsilon\)。
        \item \textbf{帕塞瓦尔恒等式}:
            \[
                \|x\|^2 = \sum_{\alpha \in A} |\norm{x}{e_\alpha}|^2.
            \]
    \end{enumerate}
\end{theorem}

\begin{proof}
    证明主要分为两步:首先证明部分和的有界性,其次证明逼近的收敛性。
    \textbf{第一步:利用网证明收敛性}。
    考虑由 \(A\) 的所有有限子集构成的集合 \(\mathcal{F}\),并通过包含关系 \(\subset\) 定义偏序。
    由于 \(\sum_{\alpha \in A} |\norm{x}{e_\alpha}|^2\) 收敛,
    对于任意 \(\epsilon > 0\),存在 \(F_0 \in \mathcal{F}\) 使得对所有 \(F \supset F_0\),有
    \[
        \| \sum_{\alpha \in F \setminus F_0} \langle x, e_\alpha
        \rangle e_\alpha \|^2 = \sum_{\alpha \in F \setminus F_0}
        |\norm{x}{e_\alpha}|^2 < \epsilon^2.
    \]
    这表明网 \((x_F)_{F \in \mathcal{F}}\) 是柯西网。由于 \(H\) 是完备的,该网收敛于某个元素 \(y \in H\)。
    容易验证 \(x - y\) 与所有 \(e_\alpha\) 正交。由规范正交基的极大性(或完备性),必有 \(x - y =
    0\),即 \(x = y\)。

    \textbf{第二步:贝塞尔不等式}。
    对于 \(A\) 的任意有限子集 \(F\),令 \(x_F = \sum_{\alpha \in F} \langle x,
    e_\alpha \rangle e_\alpha\)。注意到 \(x - x_F\) 与 \(x_F\)
    正交(通过直接计算内积验证),由勾股定理可得:
    \[
        \|x\|^2 = \|x - x_F\|^2 + \|x_F\|^2 = \|x - x_F\|^2 +
        \sum_{\alpha \in F} |\norm{x}{e_\alpha}|^2.
    \]
    由此推出 \(\sum_{\alpha \in F} |\norm{x}{e_\alpha}|^2 \le
    \|x\|^2\)。由于 \(F\) 是任意的,这意味着集合 \(\{\alpha \in A : \langle x,
    e_\alpha \rangle \neq 0\}\) 是可数的,且级数 \(\sum_{\alpha \in A}
    |\norm{x}{e_\alpha}|^2\) 收敛。
\end{proof}

\subsection{希尔伯特空间的维数与同构}

\begin{definition}[Dimension]
    希尔伯特空间 \(H\) 的\textbf{维数}(Dimension)定义为 \(H\)
    中任意规范正交基的基数(Cardinality)。记作 \(\dim H\)。
\end{definition}

为了使上述定义有意义,我们需要如下定理:

\begin{theorem}
    希尔伯特空间 \(H\) 的任意两个规范正交基具有相同的基数。
\end{theorem}

\begin{proof}
    如果基是有限时显然。

    如果基是无限的,设 \(B_1\) 和 \(B_2\) 是两个规范正交基。对于每个 \(e \in B_1\),由傅里叶展开知,
    集合 \(A_e = \{f \in B_2 : \norm{e}{f} \neq 0\}\) 是可数的。
    从而:\[
        \abs{B_2} = \abs{\bigcup_{e \in B_1} A_e} \leq \abs{B_1}
        \cdot \aleph_0 = \abs{B_1}
    \]
    同理可得 \(\abs{B_1} \leq \abs{B_2}\),因此 \(\abs{B_1} = \abs{B_2}\)。
\end{proof}

基于维数我们可以对希尔伯特空间进行分类\UseVerb{smile}:

\begin{theorem}[Isomorphism]
    两个希尔伯特空间 \(H_1\) 和 \(H_2\) 是同构的(即存在保持内积的线性双射),当且仅当它们具有相同的维数。
\end{theorem}

\begin{example}

\end{example}

\subsection{典型例子}

\begin{example}[Legendre Polynomials]
    在空间 \(L^2([-1, 1])\) 中,通过对单项式基 \(\{1, t, t^2, \dots\}\)
    进行格拉姆-施密特(Gram-Schmidt)正交化,可以得到勒让德多项式(Legendre Polynomials)。
    归一化后的勒让德多项式构成 \(L^2([-1, 1])\) 的一个规范正交基:
    \[
        e_n(t) = \sqrt{n + \frac{1}{2}} P_n(t), \quad n = 0, 1, 2, \dots
    \]
    其中 \(P_n(t) = \frac{1}{2^n n!} \odv[n]{}{t}(t^2 - 1)^n\)。
\end{example}

\begin{example}[Fourier Basis]
    \[
        \left\{ \frac{1}{\sqrt{2\pi}}, \frac{1}{\sqrt{\pi}} \sin t,
        \frac{1}{\sqrt{\pi}} \cos t, \dots \right\}
    \]
    是\(L^{2}[-\pi,\pi]\) 配备内积\(\inner{f}{g} = \int_{-\pi}^{\pi}
    f(t)g(t) \d{t}\)的一个规范正交基。
    \[
        f(t) = \frac{a_0}{2} + \sum_{n=1}^\infty \left( a_n \cos nt +
        b_n \sin nt \right)
    \]
    \[
        \norm{f}^2 = \frac{\pi}{2} a_0^2 + \pi \sum_{n=1}^\infty (a_n^2 + b_n^2)
    \]
    其中\(a_{n} = \inner{f}{\cos nt}\),\(b_{n} = \inner{f}{\sin nt}\)。
\end{example}

\subsection{等距映射}

\begin{example}
    设 \(T: H_1 \to H_2\) 是希尔伯特空间之间的线性映射。\(T\) 是等距映射(即 \(\|Tx\| =
    \|x\|, \forall x \in H_1\)),当且仅当 \(T\) 保持内积(即 \(\langle Tx, Ty
    \rangle = \inner{x}{y}, \forall x, y \in H_1\))。
\end{example}

\begin{proof}
    充分性显然,因为范数由内积定义。
    必要性由极化恒等式直接得出。例如在实情形下:
    \[
        \inner{Tx}{Ty} = \frac{1}{4}(\|Tx+Ty\|^2 -
        \|Tx-Ty\|^2) = \frac{1}{4}(\|x+y\|^2 - \|x-y\|^2) = \inner{x}{y}.
    \]
\end{proof}
