\documentclass{../../../articles/mathnotes-art}

% DISABLE_SMART_IME

\begin{document}

\paragraph{1}
Let \(X\) be a topological vector space over \(\mathbb{K}\)
(\(\mathbb{R}\) or \(\mathbb{C}\)) and let \(V\) be a neighborhood of
\(0\) in \(X\). We show \(V\) is absorbing, i.e., for every \(x\in
X\) there exists \(t>0\) such that \(x\in tV\).

Consider the continuous scalar multiplication \(m:\mathbb{K}\times
X\to X\), \((\alpha,y)\mapsto \alpha y\). Fix \(x\in X\). Continuity
of \(m\) at \((0,x)\) yields a neighborhood \(U\) of \(0\) in
\(\mathbb{K}\) such that \(\alpha x\in V\) for all \(\alpha\in U\).
In particular, there exists \(\varepsilon>0\) with
\(|\alpha|<\varepsilon\Rightarrow \alpha x\in V\).

Choose \(t>\varepsilon^{-1}\). Then \((1/t)x\in V\), hence \(x\in
tV\). Since \(x\) was arbitrary, \(V\) absorbs every \(x\in X\);
therefore \(V\) is an absorbing set.

\paragraph{2}
Let \(A = \{x \in \mathbb{R}^2 : \|x\|_2 \leq 1/2 \text{ or } \|x\|_2
= 1\}\). The gauge (Minkowski functional) of \(A\) is defined as
\(p_A(x) = \inf \{ \lambda > 0 : x \in \lambda A \}\).

For any \(x \in \mathbb{R}^2\), let \(r = \|x\|_2\). Then:
\[
    x \in \lambda A \iff \frac{x}{\lambda} \in A \iff \left\|
    \frac{x}{\lambda} \right\|_2 \leq \frac{1}{2} \quad \text{or}
    \quad \left\| \frac{x}{\lambda} \right\|_2 = 1
\]
Substituting \(\|x\|_2 = r\), this condition becomes:
\[
    \frac{r}{\lambda} \leq \frac{1}{2} \quad \text{or} \quad
    \frac{r}{\lambda} = 1 \iff \lambda \geq 2r \quad \text{or}
    \quad \lambda = r
\]
The set of admissible \(\lambda > 0\) is \(\{r\} \cup [2r, \infty)\).
The infimum of this set is clearly \(r\).
Thus, \(p_A(x) = r = \|x\|_2\). The gauge of \(A\) agrees with the
Euclidean norm \(\|\cdot\|_2\).

\end{document}
