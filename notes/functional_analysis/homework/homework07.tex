\documentclass{../../../articles/mathnotes-art}

\DeclarePairedDelimiterX{\inner}[2]{\langle}{\rangle}{#1, #2}
\DeclarePairedDelimiter{\norm}{\lVert}{\rVert}
% DISABLE_SMART_IME

\begin{document}
\paragraph{1.}
\(\forall z(t) \in L^{2}(\Omega)\), we have:
\begin{align*}
    & \phantom{=} \inner{Tx(t)}{z(t)}\\
    &= \int_{\Omega} x(t) y(t) \overline{z(t)} \d{t}\\
    &= \int_{\Omega} x(t) \overline{\overline{y(t)} z(t)} \d{t}\\
    &= \inner{x(t)}{\overline{y(t)} z(t)}\\
\end{align*}
Thus, \(T^* z(t) = \overline{y(t)} z(t)\).

\paragraph{2.}
First we prove \(T\) is self-adjoint, i.e., \(T = T^*\).

\(\forall x(x), y(x) \in L^{2}[0,1]\), by Fubini's theorem, and
\(\varphi(x)\) is real-valued, we have:
\begin{align*}
    &\phantom{=} \inner{Tx(x)}{y(x)}\\
    &= \int_{0}^{1} \left( \varphi(x) \int_{0}^{1} \varphi(t)x(t)
    \d{t} \right) \overline{y(x)} \d{x}\\
    &= \int_{0}^{1} \int_{0}^{1} \varphi(x) \overline{y(x)}
    \varphi(t) x(t) \d{t} \d{x}\\
    &= \int_{0}^{1} x(t) \varphi(t) \left( \int_{0}^{1} \varphi(x)
    \overline{y(x)} \d{x} \right) \d{t}\\
    &= \int_{0}^{1} x(t) \overline{\varphi(t) \left( \int_{0}^{1}
    \varphi(x)y(x) \d{x} \right)} \d{t}\\
    &= \inner{x(x)}{Ty(x)}
\end{align*}
therefore \(T\) is self-adjoint.

Next we prove \(T\) is positive, i.e., \(\inner{Tx}{x} \geq 0\),
\(\forall x \in L^{2}[0,1]\):
\begin{align*}
    \inner{Tx}{x} &= \int_{0}^{1} \left( \varphi(x) \int_{0}^{1}
    \varphi(t)x(t) \d{t} \right) \overline{x(x)} \d{x}\\
    &= \int_{0}^{1} \int_{0}^{1} \varphi(x) \overline{x(x)}
    \varphi(t) x(t) \d{t} \d{x}\\
    &= \left( \int_{0}^{1} \varphi(t) x(t) \d{t} \right)
    \overline{\left( \int_{0}^{1} \varphi(x) x(x) \d{x} \right)}\\
    &= \left| \int_{0}^{1} \varphi(t) x(t) \d{t} \right|^{2} \geq 0
\end{align*}

Therefore \(T\) is a positive operator.

\end{document}
