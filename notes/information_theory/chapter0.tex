% !TEX root = main.tex
\begin{quote}
    What I can not create, I do not understand.

    --- Richard Feynman
\end{quote}

\textbf{信息}: 信息是用来消除不确定性的东西。

概率越小的事件发生,携带的信息量越大。又结合信息的可叠加性,我们可以定义信息量。
\begin{definition}{信息量}
    设事件\(A\)发生的概率为\(P(A)\),则事件\(A\)发生时所包含的信息量定义为:
    \[
        I(A) = -\log P(A)
    \]
\end{definition}

熵是信息量的数学期望,是仅依赖于随机变量的概率分布的泛函。
\begin{definition}{熵}
    设随机变量\(X\)的概率分布为\(P(X=x_i)=p_i\),则随机变量\(X\)的熵定义为:
    \[
        H(X) = -\sum_i p_i \log p_i
    \]
    缺省时\(\log\) 以2为底,单位为bit (binary digit)
\end{definition}

% Make this a graph
离散无机率信源模型:
information source -> source encoder -> channel encoder -> channel ->
channel decoder -> source decoder -> destination

\begin{definition}{分离定理}
    在信道容量大于信源熵率的条件下,信源编码和信道编码可以独立设计,而不影响系统的整体最优性。
\end{definition}

网络通信则会有基于图的信息传输模型。