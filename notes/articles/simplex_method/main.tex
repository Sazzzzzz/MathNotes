\documentclass{ctexart}
\input{../header_art.tex}
\title{无痛入门单纯形法}
\author{}
\begin{document}
\maketitle

2025年3月26日,笔者入门单纯形法未遂,深感苦闷,遂作此篇。

\section{这玩意TM有用吗?}
哥们,高中写多少题了你也没问,现在问有没有用有必要吗?
% TODO: Add delete line

\section{从微积分开始}

说实话,我们已经在微积分里干过不知道多少次求最大最小值的问题了。区区求线性函数的极值,我们上来就是一个求导(梯度):
\[
    c = \nabla f =\left( \frac{\partial f}{\partial x_1},
    \frac{\partial f}{\partial x_2} \right) = (2,1)
\]

我们发现,非常函数线性函数的梯度总是常数值,这意味着求梯度使其等于零的做法是行不通的,
函数在梯度方向上始终保持严格单调递增。所以我们要求最大值,实际上是求函数可行域在线性函数\(f\) 梯度方向最远能延伸到哪里。

用一点投影的语言,可行域内

\end{document}