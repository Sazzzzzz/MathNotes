\documentclass{../mathnotes-art}
\usepackage{amsthm}
\usepackage{amsmath}
\usepackage{amssymb}
\usepackage{amsopn}
\title{无痛入门单纯形法:如果你也被矩阵绕晕……}
\bibliography{../../reference.bib}

\begin{document}

\renewcommand{\c}{\mathbf{c}}
\renewcommand{\b}{\mathbf{b}}
\newcommand{\x}{\mathbf{x}}
% TODO: why?
\providecommand{\nullspace}{\operatorname{null}}
\maketitle
\tableofcontents

大二下,最优化方法课上,笔者入门单纯形法未遂,深感苦闷\UseVerb{sad},遂作此篇。

\section{这题我会}

先来看一道基础题:
\begin{example}{}
    在约束条件
    \[
        \begin{cases}
            0 \leq x \leq 10 \\
            0 \leq y \leq 10
        \end{cases}
    \]
    下,求函数\(f(x,y) = 2x + y\) 的最大值。
\end{example}
% todo: 插入图形

其实我们已经在微积分里干过不知道多少次求最大最小值的问题了。区区求线性函数的极值,我们上来就是一个求导(梯度):
\begin{align*}
    \c &= \nabla f =\left( \pdv{f}{x}, \pdv{f}{y} \right) = (2,1) \neq (0,0)\\
    f &= \c^{\mathrm{T}} \x
\end{align*}

不难发现,非常函数线性函数的梯度总是常数值,这意味着求梯度使其等于零的做法是行不通的。
函数在梯度方向上始终保持单调递增。所以要求最大值,我们看的不是\(f(\x)\)的性质,
我们看的实际上是求约束条件(或者说\textbf{可行域})在线性函数\(f\)
的梯度方向\(\c\) 上最远能延伸到哪里。

不难看出题上所给的约束条件其实是个正方形。我们在约束条件上随便找一点,比如\((0,0)\) ,试着能沿着梯度方向走更远。
\begin{itemize}
    \item \((f(x,y) = 2 x+y\)中\(x\) 前的系数是正的,那就把\(x\) 往大了走,最大能走到10;
        此时\(f(x,y) = 20\) ,\((x,y) = (10,0)\) 。
    \item \(y\) 前的系数也是正的,那就把\(y\) 往大了走,最大也能走到\(10\)。此时\(f(x,y) = 30\)
        ,\((x,y) = (10,10)\) 。
\end{itemize}
所以起始点从\((0,0)\) 移动到了 \((10,10)\) ,此时函数值为\(f(10,10) = 30\) 。\(f(x,y)\)
在约束条件下的最大值也就是30。
% TODO: 加一个把东西调小到0的解法
\hl{Q}uite \hl{E}asy \hl{D}one!

说实话,所谓单纯形法(以及课本上成坨的矩阵计算)其实本质上和这里做的事情是一样的,也是在约束条件上找一个点,然后看着系数,
把能调大的系数调大\UseVerb{grin}

此外,隐隐约约地,我们能感觉到,线性规划中函数极值的取到点,一定是约束条件的某个\textbf{顶点}。这一点其实很好证明:
\footcite[严谨证明可见][p.~27]{ZuiYouHuaFangFa}
怎么可能出现极值呢\UseVerb{smile}

\section{标准化}
一般情况下的线性规划是由若干个等式条件和不等式条件组成的热闹海洋。但课本会把线性规划统一表示成如下形式:
\begin{definition}{线性规划的标准形式}
    线性规划问题的标准形式如下:
    \begin{align*}
        \max \quad c_{1}x_{1}  + \cdots + c_{n}x_{n} & \\
        \text{s.t.} \quad a_{i1}x_{1}  + \cdots +
        a_{in}x_{n} & \leq b_{i}, \quad i = 1, \ldots, m \\
        x_{j} & \geq 0, \quad j = 1, \ldots, n
    \end{align*}

    记\(\c = (c_{1}, \dots, c_{n})^{\mathrm{T}}\) 、\(\b =
    (b_{1},\dots, b_{m})^{\mathrm{T}}\) ,
    \[
        A =
        \begin{pmatrix}
            a_{11} & \cdots & a_{1n} \\
            \vdots & \ddots & \vdots \\
            a_{m1} & \cdots & a_{mn}
        \end{pmatrix} \\
    \]
    则线性规划的矩阵形式可简写为:
    \begin{align*}
        \max \quad & \c^{\mathrm{T}} \x \\
        \text{s.t.} \quad & A \x \leq \b \\
        & \x \geq 0
    \end{align*}
\end{definition}
% todo: add ref
% todo: add min case

化成标准形式的过程往往需要以下的dirty work\sout{(奇技淫巧)}:
\begin{itemize}
    \item[引入松弛变量] 把不等式约束变成等式约束:
        \(x + y \leq 10 \iff x + y + s = 10 \land s \geq 0\)
    \item[拆分变量] 把无非负约束的变量拆成两个非负变量的差:
        \(x\) 无非负约束 \(\iff x = x^{+} - x^{-} \land x^{+}, x^{-} \geq 0\)
    \item[目标函数变负] 把最小化问题转化成最大化问题:
        \(\min f(x,y) \iff \max -f(x,y)\)
    \item[约束变负] 把大于等于约束变成小于等于约束:
        \(x + y \geq 10 \iff -x - y \leq -10\)
\end{itemize}

可标准形式究竟有什么好处,值得我们如此大费周章地去化简呢?

\subsection{线性方程组解的结构}

在回答这个问题之前,我们先来回顾一下线性方程组的解的结构。

先考虑齐次线性方程组\(A\x = \mathbf{0}\) ,其中\(A\)
是一个行满秩\footnote{在线性规划中我们只考虑行满秩的矩阵,对于行不满秩的矩阵其等式约束条件有冗余,总可以由其他等式约束导出}的
\(m \times n\) 矩阵,其秩为\(m\)。

根据线性代数的基础理论可知,该齐次线性方程组的解空间是一个维度为\(n-m\) 的线性子空间。在三维空间中,
可以理解为经过原点的直线/平面。并且该线性方程组的解可以由\(n-m\) 个自由变量完全决定(其他的元根据这些变量唯一确定)。
不妨把该空间记为\(\nullspace A\)。

% TODO: Add figure

再来看非齐次线性方程组\(A\x = \b\) 。该线性方程组的任何一个解都能拆分成一个特解\(\x_{0}\)
和齐次线性方程组\(A\x = \b\) 的和。特解\(x_{0}\) 是任意满足\(A\x = \b\) 的解。也就是说,
非齐次线性方程组的解集是一个平移\(\x_{0}\) 后的线性子空间,在三维空间中,就是任意的平面/直线\footnote{当然了,点也算。
不过点上做线性规划可能就没什么意义了\UseVerb{grin}}。
\[
    \underbrace{\x'}_{非齐次方程的解} = \underbrace{\x_{0}}_{一个满足条件的特解} +
    \underbrace{\x}_{齐次方程的解}
\]

我们可以把解集记为\(\x_{0} + \nullspace{A}\)

\subsection{形式化解线性方程组}

在这里相信大家早就学会了怎么手动解线性方程组,如果你没学会,请回炉重造\UseVerb{relaxed}

很多同学被单纯形法吓住,都是因为课本上成坨的矩阵运算。其实这些矩阵运算最开始的起源非常简单:线性方程组。

如上所述,为了确定自由变量,我们\textbf{不妨假设}\footnote{注意这里只是假设,线性无关的列是可以任意取的。这里取前\(m\)
列只是为了方便矩阵分块} \(A\) 的前\(m\) 列线性无关,也就是\(A\) 的前\(m\) 列构成了一个可逆矩阵\(B\) 。
相对应的,记\(\x\) 的前\(m\) 个分量为基变量,后\(n-m\) 个分量为非基变量。
用分块矩阵的语言:记\(A = (B, N)\) ,\(\x =
    \begin{pmatrix}
        \x_{B} \\
        \x_{N}
\end{pmatrix}\)。

其中\(B\) 代表\textbf{基变量}(\textbf{Basic Variable})所在列组成的矩阵。\(N\)
代表\textbf{非基变量}或\textbf{自由变量}(\textbf{Non-Basic Variable})所在列组成的矩阵。
You know, if notation confuses you \UseVerb{smile}

于是:
\begin{align*}
    A\x &= \b \\
    \begin{pmatrix}
        B & N
    \end{pmatrix}
    \begin{pmatrix}
        \x_{B} \\
        \x_{N}
    \end{pmatrix} &= \b \\
    B\x_{B} + N\x_{N} &= \b \\
    B\x_{B} &= \b - N\x_{N} \\
    \x_{B} &= B^{-1}(\b - N\x_{N})
\end{align*}
这就是线性方程组的形式解。

\subsection{标准形式的几何意义}
终于,我们可以回答最开始的问题了:为什么要把线性规划化成标准形式?

可以看出:方程\(A\x = \b\) 的解,实际上是高维空间的一个超平面,一个仿射集。而约束条件\(\x \geq 0\)
是空间里的\textbf{第一象限};是一个凸锥。一个超平面与一个凸锥的交集,就像是正方体与平面相交得到的截面一样,必然是一个凸多面体,是超平面的一部分。

不仅如此,线性方程组的解集可以由\(n-m\) 个自由变量完全决定。大胆想象,在若干个自由向量的轴形成的坐标平面里,
每一个自由变量的线性组合,都唯一的确定着线性方程的一个解,解空间的一个点。
% todo: add figure

线性规划的极值多少带有些语不惊人死不休的性质:\textbf{在凸集的可行域内,线性函数的极值点必然出现在凸集的某个顶点上}。最美妙的是,
由于这个凸集是超平面与象限的交集,可以大胆想象,所有顶点的形成,一定是坐标平面与超平面相交所得到的;也就是说,这个凸集上的每一个顶点,
肯定有一堆变量为零。具体的说,每一个顶点,都恰好是那些决定解的自由变量\(\x_{N}\)为零。\hl{所有的顶点,就是在划分好基变量/非基变量后,
把所有非基变量设为零得到的解}。代入公式\(\x_{B} = B^{-1}(\b -
N\x_{N})\) ,
其实凸集的所有顶点都长\(
    \begin{pmatrix}
        B^{-1} \b \\
        \mathbf{0}
\end{pmatrix}\)这个样子\UseVerb{smile}
\footnote{很多教科书在此处用\textbf{不妨}二字一笔带过\UseVerb{grin}}

% todo: add figure
\section{当我们知道何时成功}
% todo: 有没有知道何时成功的狗血小鸡汤
OK,了解了顶点,我们就成功了一半。接下来我们需要关注线性函数在这些顶点上的取值。

从之前的例子我们知道,线性函数的取值完全由系数决定。求最大值,对参数为正的变量而言,就是把能调大的变量调大;对参数为负的变量而言,
就是把能调小的变量调小。结合上面我们的分析:顶点都是自由变量取零得到的,\hl{此时的自由变量将小无可小,因为他们身后就是坐标轴}\UseVerb{smile}

所以,如果我们能\hl{把所有自由变量的系数都变成负数(非正数),并且让其他基变量的系数为零,那这些自由变量取零得到的顶点,
就将是线性函数的极值点}\UseVerb{surprised}

不过先别着急高兴,怎么准确的算出换元之后基变量和自由变量的系数还是一个大问题\UseVerb{confused}

为此,我们将\(\c\) 分块成\(\c = (\c_{B}, \c_{N})\) ,对应基变量和非基变量。于是线性函数可以写成:
\begin{align*}
    \c^{\mathrm{T}} \x &= \c_{B}^{\mathrm{T}} \x_{B} +
    \c_{N}^{\mathrm{T}} \x_{N} \\
    &= \c_{B}^{\mathrm{T}} B^{-1}(\b - N\x_{N}) + \c_{N}^{\mathrm{T}} \x_{N} \\
    &= \c_{B}^{\mathrm{T}} B^{-1} \b + (\c_{N}^{\mathrm{T}} -
    \c_{B}^{\mathrm{T}} B^{-1} N) \x_{N}
\end{align*}

线性函数的取值,不过是一个常数项\(\c_{B}^{\mathrm{T}} B^{-1} \b\)
加上自由变量的线性组合\((\c_{N}^{\mathrm{T}} -
\c_{B}^{\mathrm{T}} B^{-1} N) \x_{N}\) 。为使得自由变量的系数都非正,我们只需要让\[
    \Delta = \c_{N}^{\mathrm{T}} - \c_{B}^{\mathrm{T}} B^{-1} N \leq \bm{0}
\] 即可。

没错,我们就这样推导出了单纯形法的判别式\UseVerb{cool}

\section{旋转跳跃我闭着眼}
\printbibliography[heading=bibintoc,title={参考文献}]
\end{document}