\documentclass{../mathnotes-art}
\usepackage{amsthm}
\usepackage{amsmath}
\usepackage{amssymb}
\usepackage{amsopn}
\title{无痛入门单纯形法:如果你也被矩阵绕晕……}
\bibliography{../../reference.bib}

\renewcommand{\c}{\mathbf{c}}
\renewcommand{\b}{\mathbf{b}}
\newcommand{\x}{\mathbf{x}}
\DeclareMathOperator{\nullspace}{null}

\begin{document}
\maketitle
\tableofcontents

大二下,最优化方法课上,笔者入门单纯形法未遂,深感苦闷,遂作此篇\UseVerb{sad}

\section{这题我会}

先来看一道基础题:
\begin{example}{}
    在约束条件
    \[
        \begin{cases}
            0 \leq x \leq 10 \\
            0 \leq y \leq 10
        \end{cases}
    \]
    下,求函数\(f(x,y) = 2x + y\) 的最大值。
\end{example}
% todo: 插入图形

其实我们已经在微积分里干过不知道多少次求最大最小值的问题了。区区求线性函数的极值,我们上来就是一个求导(梯度):
\begin{align*}
    \c &= \nabla f =\left( \pdv{f}{x}, \pdv{f}{y} \right) = (2,1) \neq (0,0)\\
    f &= \c^{\mathrm{T}} \x
\end{align*}

不难发现,非常函数线性函数的梯度总是常数值,这意味着求梯度使其等于零的做法是行不通的。
函数在梯度方向上始终保持单调递增。所以要求最大值,我们看的不是\(f(\x)\)的性质,
我们看的实际上是求约束条件(或者说\textbf{可行域})在线性函数\(f\)
的梯度方向\(\c\) 上最远能延伸到哪里。

不难看出题上所给的约束条件其实是个正方形。我们在约束条件上随便找一点,比如\((0,0)\) ,试着能沿着梯度方向走更远。
\begin{itemize}
    \item \((f(x,y) = 2 x+y\)中\(x\) 前的系数是正的,那就把\(x\) 往大了走,最大能走到10;
        此时\(f(x,y) = 20\) ,\((x,y) = (10,0)\) 。
    \item \(y\) 前的系数也是正的,那就把\(y\) 往大了走,最大也能走到\(10\)。此时\(f(x,y) = 30\)
        ,\((x,y) = (10,10)\) 。
\end{itemize}
所以起始点从\((0,0)\) 移动到了 \((10,10)\) ,此时函数值为\(f(10,10) = 30\) 。\(f(x,y)\)
在约束条件下的最大值也就是30。
% TODO: 加一个把东西调小到0的解法

\hl{Q}uite \hl{E}asy \hl{D}one!

说实话,所谓单纯形法(以及课本上成坨的矩阵计算)其实本质上和这里做的事情是一样的,也是在约束条件上找一个点,然后看着系数,
把能调大的系数调大\UseVerb{grin}

此外,隐隐约约地,我们能感觉到,线性规划中函数极值点,一定是约束条件的某个\textbf{顶点}。这一点其实很好说明:
凸集里所有点都能通过顶点的仿射组合(和为1的线性组合)得到。那中间点的函数取值,也能通过顶点的函数取值仿射组合得到,
也就被限制在顶点的函数取值范围内了\UseVerb{smile}
\footcite[严谨证明可见最优化方法][p.~27]{ZuiYouHuaFangFa}
怎么可能出现极值呢\UseVerb{smile}

\section{标准化}
一般情况下的线性规划是由若干个等式条件和不等式条件组成的热闹海洋。但课本会把线性规划统一表示成如下形式:
\begin{definition}{线性规划的标准形式}
    线性规划问题的标准形式如下:
    \begin{align*}
        \max \quad c_{1}x_{1}  + \cdots + c_{n}x_{n} & \\
        \text{s.t.} \quad a_{i1}x_{1}  + \cdots +
        a_{in}x_{n} & \leq b_{i}, \quad i = 1, \ldots, m \\
        x_{j} & \geq 0, \quad j = 1, \ldots, n
    \end{align*}

    记\(\c = (c_{1}, \dots, c_{n})^{\mathrm{T}}\) 、\(\b =
    (b_{1},\dots, b_{m})^{\mathrm{T}}\) ,
    \[
        A =
        \begin{pmatrix}
            a_{11} & \cdots & a_{1n} \\
            \vdots & \ddots & \vdots \\
            a_{m1} & \cdots & a_{mn}
        \end{pmatrix} \\
    \]
    则线性规划的矩阵形式可简写为:
    \begin{align*}
        \max \quad & \c^{\mathrm{T}} \x \\
        \text{s.t.} \quad & A \x \leq \b \\
        & \x \geq 0
    \end{align*}
\end{definition}
% todo: add ref
% todo: add min case

化成标准形式的过程往往需要以下的dirty work\sout{(奇技淫巧)}:
\begin{itemize}
    \item[引入松弛变量] 把不等式约束变成等式约束:
        \(x + y \leq 10 \iff x + y + s = 10 \land s \geq 0\)
    \item[拆分变量] 把无非负约束的变量拆成两个非负变量的差:
        \(x\) 无非负约束 \(\iff x = x^{+} - x^{-} \land x^{+}, x^{-} \geq 0\)
    \item[目标函数变负] 把最小化问题转化成最大化问题:
        \(\min f(x,y) \iff \max -f(x,y)\)
    \item[约束变负] 把大于等于约束变成小于等于约束:
        \(x + y \geq 10 \iff -x - y \leq -10\)
\end{itemize}

可标准形式究竟有什么好处,值得我们如此大费周章地去化简呢?

\subsection{线性方程组解的结构}

在回答这个问题之前,我们先来回顾一下线性方程组的解的结构。

先考虑齐次线性方程组\(A\x = \mathbf{0}\) ,其中\(A\)
是一个行满秩\footnote{在线性规划中我们只考虑行满秩的矩阵,对于行不满秩的矩阵其等式约束条件有冗余,总可以由其他等式约束导出}的
\(m \times n\) 矩阵,其秩为\(m\)。

根据线性代数的基础理论可知,该齐次线性方程组的解空间是一个维度为\(n-m\) 的线性子空间。在三维空间中,
可以理解为经过原点的直线/平面。并且该线性方程组的解可以由\(n-m\) 个自由变量完全决定(其他的元根据这些变量唯一确定)。
不妨把该空间记为\(\nullspace A\)。

% TODO: Add figure

再来看非齐次线性方程组\(A\x = \b\) 。该线性方程组的任何一个解都能拆分成一个特解\(\x_{0}\)
和齐次线性方程组\(A\x = \b\) 的和。特解\(x_{0}\) 是任意满足\(A\x = \b\) 的解。也就是说,
非齐次线性方程组的解集是一个平移\(\x_{0}\) 后的线性子空间,在三维空间中,就是任意的平面/直线\footnote{当然了,点也算。
不过点上做线性规划可能就没什么意义了\UseVerb{grin}}。
\[
    \underbrace{\x'}_{\text{非齐次方程的解}} = \underbrace{\x_{0}}_{\text{一个满足条件的特解}} +
    \underbrace{\x}_{\text{齐次方程的解}}
\]

我们可以把解集记为\(\x_{0} + \nullspace{A}\)

\subsection{形式化解线性方程组}

在这里相信大家早就学会了怎么手动解线性方程组,如果你没学会,请回炉重造\UseVerb{relaxed}

很多同学被单纯形法吓住,都是因为课本上成坨的矩阵运算。其实这些矩阵运算最开始的起源非常简单:线性方程组。

如上所述,为了确定自由变量,我们\textbf{不妨假设}\footnote{注意这里只是假设,线性无关的列是可以任意取的。这里取前\(m\)
列只是为了方便矩阵分块} \(A\) 的前\(m\) 列线性无关,也就是\(A\) 的前\(m\) 列构成了一个可逆矩阵\(B\) 。
相对应的,记\(\x\) 的前\(m\) 个分量为基变量,后\(n-m\) 个分量为非基变量。
用分块矩阵的语言:记\(A = (B, N)\) ,\(\x =
    \begin{pmatrix}
        \x_{B} \\
        \x_{N}
\end{pmatrix}\)。

其中\(B\) 代表\textbf{基变量}(\textbf{\hl{B}asic Variable})所在列组成的矩阵。\(N\)
代表\textbf{非基变量}或\textbf{自由变量}(\textbf{\hl{N}on-Basic
Variable})所在列组成的矩阵。\footnote{请熟悉基变量/非基变量/自由变量这几个名词,
在后文中会大量的使用\UseVerb{surprised}}
You know, if notation confuses you \UseVerb{smile}

于是:
\begin{align*}
    A\x &= \b \\
    \begin{pmatrix}
        B & N
    \end{pmatrix}
    \begin{pmatrix}
        \x_{B} \\
        \x_{N}
    \end{pmatrix} &= \b \\
    B\x_{B} + N\x_{N} &= \b \\
    B\x_{B} &= \b - N\x_{N} \\
    \x_{B} &= B^{-1}(\b - N\x_{N}) \label{eq:basis-solution}
\end{align*}
\(\x =
    \begin{pmatrix}
        \x_{B} \\
        \x_{N}
    \end{pmatrix} =
    \begin{pmatrix}
        B^{-1}(\b - N\x_{N}) \\
        \x_{N}
\end{pmatrix}\)这就是线性方程组的形式解。

\subsection{标准形式的几何意义}
终于,我们可以回答最开始的问题了:为什么要把线性规划化成标准形式?

可以看出:方程\(A\x = \b\) 的解,实际上是高维空间的一个超平面,一个仿射集。而约束条件\(\x \geq 0\)
是空间里的\textbf{第一象限};是一个凸锥。一个超平面与一个凸锥的交集,就像是正方体与平面相交得到的截面一样,必然是一个凸多面体,是超平面的一部分。

不仅如此,线性方程组的解集可以由\(n-m\) 个自由变量完全决定。大胆想象,在若干个自由向量的轴形成的坐标平面里,
每一个自由变量的线性组合,都唯一的确定着线性方程的一个解,解空间的一个点。
% todo: add figure

线性规划的极值多少带有些语不惊人死不休的性质:\textbf{在凸集的可行域内,线性函数的极值点必然出现在凸集的某个顶点上}。最美妙的是,
由于这个凸集是超平面与象限的交集,可以大胆想象,所有顶点的形成,一定是坐标平面与超平面相交所得到的;也就是说,这个凸集上的每一个顶点,
肯定有一堆变量为零。具体的说,每一个顶点,都恰好是那些决定解的自由变量\(\x_{N}\)为零。\hl{所有的顶点,就是在划分好基变量/非基变量后,
把所有非基变量设为零得到的解}。代入公式\(\x_{B} = B^{-1}(\b -
N\x_{N})\) ,
其实凸集的所有顶点都长\(
    \begin{pmatrix}
        B^{-1} \b \\
        \mathbf{0}
\end{pmatrix}\label{eq:vertex_solution}\)这个样子\UseVerb{smile}
\footnote{很多教科书在此处用\textbf{不妨}二字一笔带过\UseVerb{grin}}

% todo: add figure
\section{当我们知道何时成功}
\subsection{手把手推导判别式}
OK,了解了顶点,我们就成功了一半。接下来我们需要关注线性函数在这些顶点上的取值。

从之前的例子我们知道,线性函数的取值完全由系数决定。求最大值,对参数为正的变量而言,就是把能调大的变量调大;对参数为负的变量而言,
就是把能调小的变量调小。结合上面我们的分析:顶点都是自由变量取零得到的,\hl{此时的自由变量将小无可小,因为他们身后就是坐标轴}\UseVerb{smile}

所以,如果我们能\textbf{把所有自由变量的系数都变成负数(非正数),并且让其他基变量的系数为零,那这些自由变量取零得到的顶点,
就将是线性函数的极值点}\UseVerb{surprised}

不过先别着急高兴,怎么准确的算出换元之后基变量和自由变量的系数还是一个大问题\UseVerb{confused}

为此,我们将\(\c\) 分块成\(\c = (\c_{B}, \c_{N})\) ,对应基变量和非基变量。于是线性函数可以写成:

\begin{align*}
    \c^{\mathrm{T}} \x &= \c_{B}^{\mathrm{T}} \x_{B} +
    \c_{N}^{\mathrm{T}} \x_{N} \\
    &= \c_{B}^{\mathrm{T}} B^{-1}(\b - N\x_{N}) + \c_{N}^{\mathrm{T}} \x_{N} \\
    &= \c_{B}^{\mathrm{T}} B^{-1} \b + (\c_{N}^{\mathrm{T}} -
    \c_{B}^{\mathrm{T}} B^{-1} N) \x_{N}
\end{align*}

线性函数的取值,不过是一个常数项\(\c_{B}^{\mathrm{T}} B^{-1} \b\)
加上自由变量的线性组合\((\c_{N}^{\mathrm{T}} -
\c_{B}^{\mathrm{T}} B^{-1} N) \x_{N}\) 。为使得自由变量的系数都非正,我们只需要让\[
    \Delta = \c_{N}^{\mathrm{T}} - \c_{B}^{\mathrm{T}} B^{-1} N \leq \bm{0}
\] 即可。

没错,我们就这样推导出了单纯形法的判别式\UseVerb{cool}

当然,理论的推导不能替代实际的计算。在实际操作过程当中,我们并不会刻意地去计算\(\Delta = \c_{N}^{\mathrm{T}}
- \c_{B}^{\mathrm{T}} B^{-1} N\),也不会去算\(\x_{B} = B^{-1}(\b -
N\x_{N})\)的那个值。这些矩阵的抽象只是告诉我们为什么这样做是可行的,以及我们背后在究竟算什么。

\begin{quote}
    The power of mathematics rests on its evasion of all unnecessary
    thought and on its wonderful saving of mental operations.

    数学的力量在于它规避了所有不必要的思考,并奇妙地节省了脑力运作。

    -- Alfred North Whitehead
\end{quote}

\subsection{一个例子}

\begin{problem}{}
    用修正单纯形法解下列线性规划问题:
    \[
        \begin{aligned}
            \min\quad & -2x_1 - 3x_2, \\
            \text{s.t.}\quad & -x_1 + x_2 \le 3, \\
            & -2x_1 + x_2 \le 2, \\
            & 4x_1 + x_2 \le 16, \\
            & x_1, x_2 \ge 0.
        \end{aligned}
    \]
\end{problem}
先将问题写成标准形式:
\[
    \begin{aligned}
        \max\quad & 2x_1 + 3x_2 \\
        \text{s.t.}\quad
        & -x_1 + x_2 + x_3 = 3, \\
        & -2x_1 + x_2 + x_4 = 2, \\
        & 4x_1 + x_2 + x_5 = 16, \\
        & x_i \geq 0,\quad i=1,\dots,5.
    \end{aligned}
\]
可见
\[
    A =
    \begin{pmatrix}
        -1 & 1 & 1 & 0 & 0 \\
        -2 & 1 & 0 & 1 & 0 \\
        4 & 1 & 0 & 0 & 1
    \end{pmatrix},
    \quad
    \b =
    \begin{pmatrix}
        3 \\ 2 \\ 16
    \end{pmatrix}.
\]
可见矩阵后三列恰好是单位矩阵,取\(B = (a_{3},a_{4},a_{5})\),基指标\({3,4,5}\),
于是我们列出所谓\textbf{单纯形表}(Simplex Tableau):
\[
    \begin{array}{c|*{5}{>{\hspace{1em}}c<{\hspace{1em}}}|c}
        \hline
        & x_1 & x_2 & x_3 & x_4 & x_5 &    \\ \hline
        \bar{\c} & 2  & 3  & 0   & 0   & 0   &    \\ \hline
        x_{3} & -1  & 1   & 1   & 0   & 0   & 3  \\
        x_{4} & -2  & 1   & 0   & 1   & 0   & 2  \\
        x_{5} & 4   & 1   & 0   & 0   & 1   & 16 \\ \hline
    \end{array}
\]

解释一下:
\begin{itemize}
    \item 这里\(B\) 就是单位阵\footnote{仅靠现有的单纯形法是不能确定初始基变量的,
        一般教科书的例题往往会像这里直接给出初始基变量。完善的方法可参考引入两阶段法或大\(M\)法},
        我们可以理解为这就是已经经过取基变量 \(\to \) 对函数换元之后的结果。
    \item 左侧第一列:所有的基变量
    \item 最后一列是求出的\(B^{-1}\b\)
    \item 结合\cref{eq:vertex_solution},可知此时我们测试的顶点就是\(x_{3} \to 3,
        x_{4} \to 2, x_{5} \to 16\),其他变量为零。
    \item 中间的矩阵部分是\(B^{-1}A\)
    \item 上面的\(\bar{\c}\) 是经过变换后的线性函数系数,对非基变量来说是\(\c_{B}^{\mathrm{T}}
        B^{-1} N\) ,对基变量来说是零。
\end{itemize}

以上所有的矩阵抽象,在计算里非常简单:选定好基变量之后,用高斯消元法把基变量列变成单位阵,
然后对线性函数系数做同样的行变换即可\footnote{如果你愿意在求解开始给\(\c\)行的最后补上\(0\),
那么当你结束后可以直接从这个位置读出线性函数的最小值}。

在这个例子中我们可以读出:非基变量的系数为\((2,3)\) ,均为正,说明我们可以通过调大非基变量来增大线性函数值,单纯形法的任务远没有结束。接下来的任务,
就是单纯形法的核心:
\textbf{换基}(Pivot\footnote{\textit{pivot}有支点、枢轴、使在枢轴上旋转的意思,
这里翻译成旋转也许更有趣\UseVerb{wink}})

\section{旋转跳跃我闭着眼}

这世上的矩阵里,有些列在矩阵里是中流砥柱,例如矩阵\(
    \begin{pmatrix}
        1 & 0 & 0 \\
        0 & 1 & 0 \\
\end{pmatrix}\) 。预想找到该矩阵的一组基变量列,列\(
    \begin{pmatrix}
        1 \\ 0
\end{pmatrix}\)是必不可少的。还有些矩阵则比较随和,例如\(
    \begin{pmatrix}
        2 & 1 & 0 \\
        0 & 1 & 2 \\
\end{pmatrix}\) 。任意两列都可以作为基变量列。

为了找到目标函数的最小值,单纯形法要测试不同的顶点,也就是测试不同的基变量组合,然后让那些自由变量取零,计算出对应的线性函数值。
那考虑最坏情况,对于一个\(m \times n\) 的矩阵,如果任意\(m\) 列都可以作为基变量列,那么总共的顶点数就是:\[
    \binom{n}{m} = \frac{n!}{m!(n-m)!}
\]

所以如果我们暴力的测试所有的顶点,最坏情况下需要测试\(\binom{n}{m}\) 次\UseVerb{cry}。也正因如此,
单纯形法采用贪心的策略:每次调整基变量/非基变量的划分,只调整一个变量,朝着让线性函数值变大的方向前进。它恰好贪心,可行域恰好是凸集。
局部最优恰好是全局最优,单纯形法因而有效\UseVerb{smile}\footnote{也因此,在这个意义下单纯形法的最坏时间复杂度是指数级别的,
但通过微扰输入可以证明单纯形法在实际应用有多项式时间复杂度\cite{ComplexitySimplexAlgorithm2010}}

\subsection{尝试一次换基}

我们尝试完成单纯形法的第一次换基。

在系数行里,我们发现非基变量\(x_{2}\) 的系数最大,那么我们就尝试把它调大。对应单纯形表里,就是让\(x_{2}\) 作为基变量,
这样它就能从零变到正数\UseVerb{grin}。用单纯形表主体第一行第二列做高斯消元的主元,把该列剩余部分消为零:
\[
    \begin{array}{c|*{5}{c}|c}
        \hline
        & x_1 & x_2 & x_3 & x_4 & x_5 &    \\ \hline
        \bar{\c} & 2  & 3  & 0   & 0   & 0   &    \\ \hline
        x_{3} & -1  & [1]   & 1   & 0   & 0   & 3  \\
        x_{4} & -2  & 1   & 0   & 1   & 0   & 2  \\
        x_{5} & 4   & 1   & 0   & 0   & 1   & 16 \\ \hline
    \end{array}
    \quad \to \quad
    \begin{array}{c|*{5}{c}|c}
        \hline
        & x_1 & x_2 & x_3 & x_4 & x_5 &    \\ \hline
        \bar{\c} & 5  & 0  & -3   & 0   & 0   &    \\ \hline
        x_{2} & -1  & 1   & 1   & 0   & 0   & 3  \\
        x_{4} & -1  & 0   & -1   & 1   & 0   & \color{red}{-1} \\
        x_{5} & 5   & 0   & -1   & 0   & 1   & 13 \\ \hline
    \end{array}
\]

一切几乎都很顺利:
\begin{itemize}
    \item \(x_{2}\) 作为基变量引入了单纯形表,相对应的,\(x_{3}\) 作为非基变量被移除,
        这个过程就叫做\textbf{换基}(Pivot)
    \item 此时读出的解是\(x_{2} = 3, x_{5} = 13, x_{4} = -1\),剩余的变量为零,
        零的数目保持不变(\(m-n\)个)。符合我们的预测
    \item 原本两个正系数变成了一个正系数(5)和一个负系数(-3),线性函数值增大了
\end{itemize}
So far so good. 朝着这个方向继续努力,不久我们就能找到最优解\UseVerb{wink} 但是出现了一个问题:
\(x_{4}\) 的值变成了负数!

线性规划的约束条件要求所有变量非负,我们必须阻止这种情况的发生。

\subsection{最小比率检验}

先前我们的讨论指出,顶点是超平面和坐标轴的交点。调整基指标的过程,就是沿着坐标轴移动交点的过程。顶点都具有\(n-m\) 个零值变量\(
    \begin{pmatrix}
        B^{-1} \b \\
        \mathbf{0}
\end{pmatrix}\) 的形式,
但\textbf{并非所有有\(n-m\) 个零值变量且在解空间中的点都是顶点}。我们把所有形如\(
    \begin{pmatrix}
        B^{-1} \b \\
        \mathbf{0}
\end{pmatrix}\) 的点称为\textbf{基本解},如果基本解满足非负约束条件,则称为\textbf{基本可行解}。

在换基的过程中,我们必须保证新的基本解仍然是基本可行解。

\subsubsection{当我们在换基时,我们究竟在干什么?}

当我们移出变量\(x_{k}\),引入\(x_{j}\) 作为基变量时,我们实际上是在让\(x_{k}\)从零开始增大。始终有:
\[
    \x_{B} = B^{-1}(\b - N\x_{N})
\]
只调整\(x_{k}\) ,其他非基变量保持为零,于是:
\[
    \x_{B} = B^{-1}b - B^{-1}a_{k} x_{k}
\]
落在每一个基变量\(x_{B_{i}}\) 上,就是:
\[
    x_{B_i} = \bar{b}_i - \bar{a}_{ik} x_k
\]
其中 \(\bar{b}_i\) 是当前基变量的值(\(\bar{\b}\)也就是单纯形表最右边那一列),\(\bar{a}_{ik}\) 是我们选中的入基变量
\(x_k\) 所在列对应的系数(也就是单纯形表中间的那部分)。

我们要保证新的解仍然在可行域内,也就是说,所有的基变量 \(x_{B_i}\) 必须仍然大于等于零:
\[
    \bar{b}_i - \bar{a}_{ik} x_k \geq 0
\]

这其实就是给 \(x_k\) 的增长设限:

\begin{itemize}
    \item 如果 \(\bar{a}_{ik} \leq 0\),那么不等式变成 \(\bar{b}_i \geq
        \text{负数或0} \times x_k\)。因为 \(\bar{b}_i\)
        本来就是非负的(别忘了我们是从一个可行解出发的),\(x_k\) 也是非负的,这个不等式恒成立。这意味着这个基变量并不限制
        \(x_k\) 的增长。
    \item 如果 \(\bar{a}_{ik} > 0\),情况就不一样了。不等式变成了:
        \[
            x_k \leq \frac{\bar{b}_i}{\bar{a}_{ik}}
        \]
        这意味着,一旦 \(x_k\) 超过了 \(\frac{\bar{b}_i}{\bar{a}_{ik}}\),这个基变量
        \(x_{B_i}\) 就会变成负数,从而跑出可行域。
\end{itemize}

为了照顾所有的基变量,\(x_k\) 的增长不能超过所有这些限制中\textbf{最小}的那个。
这就是\textbf{最小比率检验}(Minimum Ratio Test):
\[
    x_k = \min \left\{ \frac{\bar{b}_i}{\bar{a}_{ik}} \mid
    \bar{a}_{ik} > 0 \right\}
\]

那个限制最紧的基变量(也就是比率最小的那一行对应的变量),会在 \(x_k\) 达到最大值时恰好变成 0。
既然它变成了 0,那它就光荣地完成了作为基变量的使命,退化成非基变量(Leaving Variable)。而 \(x_k\) 则从 0
变成了正数,晋升为基变量(Entering Variable)。

这就是单纯形法中换基的本质:我们在可行域的边界上,沿着目标函数增加的方向(reduced cost > 0),
从一个顶点滑向相邻的另一个顶点,直到被某个约束条件挡住为止。

\subsection{换基直到成功}

接下来完成上面的例子,以下内容基本照搬自\cite[pp.~36-37]{ZuiYouHuaFangFa} :

考虑消去\(x_{2}\) ,\(x_{4}\) 所在行比率最小\footnote{这正是我们错误例子中不满足非负约束的行\UseVerb{grin}}:
\[
    \begin{array}{c|*{5}{c}|c}
        \hline
        & x_1 & x_2 & x_3 & x_4 & x_5 &    \\ \hline
        \bar{\c} & 2  & 3  & 0   & 0   & 0   &    \\ \hline
        x_{3} & -1  & 1   & 1   & 0   & 0   & 3  \\
        x_{4} & -2  & [1]   & 0   & 1   & 0   & 2  \\
        x_{5} & 4   & 1   & 0   & 0   & 1   & 16 \\ \hline
    \end{array}
    \quad \to \quad
    \begin{array}{c|*{5}{c}|c}
        \hline
        & x_1 & x_2 & x_3 & x_4 & x_5 &    \\ \hline
        \bar{\c} & -8  & 0  & 0   & 3   & 0   &    \\ \hline
        x_{3} & 1  & 0   & 1   & -1   & 0   & 1  \\
        x_{2} & -2  & 1   & 0   & 1   & 0   & 2  \\
        x_{5} & 6   & 0   & 0   & -1   & 1   & 14 \\ \hline
    \end{array}
\]
\[
    \begin{array}{c|*{5}{c}|c}
        \hline
        & x_1 & x_2 & x_3 & x_4 & x_5 &    \\ \hline
        \bar{\c} & -8  & 0  & 0   & 3   & 0   &    \\ \hline
        x_{3} & [1]  & 0   & 1   & -1   & 0   & 1  \\
        x_{2} & -2  & 1   & 0   & 1   & 0   & 2  \\
        x_{5} & 6   & 0   & 0   & -1   & 1   & 14 \\ \hline
    \end{array}
    \quad \to \quad
    \begin{array}{c|*{5}{c}|c}
        \hline
        & x_1 & x_2 & x_3 & x_4 & x_5 &    \\ \hline
        \bar{\c} & 0  & 0  & 8   & -5   & 0   &    \\ \hline
        x_{1} & 1  & 0   & 1   & -1   & 0   & 1  \\
        x_{2} & 0  & 1   & 2   & -1   & 0   & 4  \\
        x_{5} & 0   & 0   & -6   & 5   & 1   & 8 \\ \hline
    \end{array}
\]
\[
    \begin{array}{c|*{5}{c}|c}
        \hline
        & x_1 & x_2 & x_3 & x_4 & x_5 &    \\ \hline
        \bar{\c} & 0  & 0  & 8   & -5   & 0   &    \\ \hline
        x_{1} & 1  & 0   & 1   & -1   & 0   & 1  \\
        x_{2} & 0  & 1   & 2   & -1   & 0   & 4  \\
        x_{5} & 0   & 0   & -6   & [5]   & 1   & 8 \\ \hline
    \end{array}
    \quad \to \quad
    \begin{array}{c|*{5}{c}|c}
        \hline
        & x_1 & x_2 & x_3 & x_4 & x_5 &    \\ \hline
        \bar{\c} & 0  & 0  & 2   & 0   & 1   &    \\ \hline
        x_{1} & 1  & 0   & -1/5   & 0   & 1/5   &
        13/5  \\
        x_{2} & 0  & 1   & 4/5   & 0   & 1/5   &
        28/5  \\
        x_{5} & 0   & 0   & -6/5   & 1   & 1/5   &
        8/5 \\ \hline
    \end{array}
\]
% TODO: better reference
% TODO: add ending
\printbibliography[heading=bibintoc,title={参考}]
\end{document}