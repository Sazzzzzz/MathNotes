\documentclass{../mathnotes-art}
\title{无痛入门单纯形法:如果你也被矩阵绕晕……}
\begin{document}

\renewcommand{\c}{\mathbf{c}}
\renewcommand{\b}{\mathbf{b}}
\newcommand{\x}{\mathbf{x}}

\maketitle
\tableofcontents
2025年3月26日,笔者入门单纯形法未遂,深感苦闷,遂作此篇。

\section{这题我会}

先来看一道基础例题:
\begin{example}{}
    在约束条件
    \[
        \begin{cases}
            0 \leq x \leq 10 \\
            0 \leq y \leq 10
        \end{cases}
    \]
    下,求函数\(f(x,y) = 2x + y\) 的最大值。
\end{example}
% todo: 插入图形

其实我们已经在微积分里干过不知道多少次求最大最小值的问题了。区区求线性函数的极值,我们上来就是一个求导(梯度):
\begin{align*}
    \c &= \nabla f =\left( \pdv{f}{x}, \pdv{f}{y} \right) = (2,1) \neq (0,0)\\
    f &= \c^{\mathrm{T}} \x
\end{align*}

不难发现,非常函数线性函数的梯度总是常数值,这意味着求梯度使其等于零的做法是行不通的。
函数在梯度方向上始终保持单调递增。所以我们要求最大值,实际上是求约束条件(或者说\textbf{可行域})在线性函数\(f\)
的梯度方向\(\c\) 上最远能延伸到哪里。

不难看出题上所给的约束条件其实是个正方形。我们在约束条件上随便找一点,比如\((0,0)\) ,看我们是否能沿着梯度方向走更远。
\begin{itemize}
    \item \((f(x,y) = 2 x+y\)中\(x\) 前的系数是正的,那就把\(x\) 往大了走,最大能走到10;
    \item \(y\) 前的系数也是正的,那就把\(y\) 往大了走,最大也能走到10。
\end{itemize}
所以起始点从\((0,0)\) 移动到了 \((10,10)\) ,此时函数值为\(f(10,10) = 30\) 。\(f(x,y)\)
在约束条件下的最大值也就是30。

Quite Easy Done!

说实话,所谓单纯形法(以及课本上成坨的矩阵计算)其实本质上和这里做的事情是一样的,也是在约束条件上找一个点,然后看着系数,
把能调大的系数调大\UseVerb{grin}

\section{标准化}
一般情况下的线性规划是由若干个等式条件和不等式条件构成的。但课本会把线性规划统一表示成如下形式:
\begin{definition}{线性规划的标准形式}
    线性规划问题的标准形式如下:
    \begin{align*}
        \max \quad c_{1}x_{1}  + \cdots + c_{n}x_{n} & \\
        \text{s.t.} \quad a_{i1}x_{1}  + \cdots +
        a_{in}x_{n} & \leq b_{i}, \quad i = 1, \ldots, m \\
        x_{j} & \geq 0, \quad j = 1, \ldots, n
    \end{align*}

    记\(\c = (c_{1}, \dots, c_{n})^{\mathrm{T}}\) 、\(\b =
    (b_{1},\dots, b_{m})^{\mathrm{T}}\) ,
    \[
        A =
        \begin{pmatrix}
            a_{11} & \cdots & a_{1n} \\
            \vdots & \ddots & \vdots \\
            a_{m1} & \cdots & a_{mn}
        \end{pmatrix} \\
    \]
    则线性规划的矩阵形式可简写为:
    \begin{align*}
        \max \quad & \c^{\mathrm{T}} \x \\
        \text{s.t.} \quad & A \x \leq \b \\
        & \x \geq 0
    \end{align*}
\end{definition}
% todo: add ref
% todo: add min case

化成标准形式的过程往往涉及以下的奇技淫巧:
\begin{itemize}
    \item 把不等式约束变成等式约束(引入松弛变量);
    \item 把大于等于约束变成小于等于约束(乘以-1);
    \item 把无非负约束的变量拆成两个非负变量的差;
    \item 把最小化问题转化成最大化问题(目标函数变负)。
\end{itemize}

为什么?

\subsection{线性方程组解的结构}
\subsection{标准形式的几何意义}

于是可以看到:方程\(A\x = \b\) ,实际上是高维空间的一个超平面,一个仿射集。而约束条件\(\x \geq 0\)
只是空间里的一个象限,是一个凸集。一个超平面与一个凸锥的交集,就像是正方体与平面相交得到的截面一样,是一个凸多面体,是超平面的一部分。

不仅如此,线性方程组的解集可以由\(n-m\) 个自由变量完全决定。可以想象,若干个轴形成的空间里,每一个自由变量的组合,都唯一的确定着线性方程的一个解。

线性规划的极值多少带有些语不惊人死不休的性质:在这样一个凸集的可行域内,线性函数的极值点必然出现在凸集的某个顶点上。更美妙的是,
由于该凸集是超平面与象限的交集,可以想象所有顶点的形成,一定是坐标平面与超平面相交所得到的,也就是说,每一个顶点都对应着某些变量取零,
从而使得线性方程组的解唯一确定。

% todo: add figure

\section{不妨?妨!很妨!}
\section{旋转跳跃我闭着眼}

\end{document}