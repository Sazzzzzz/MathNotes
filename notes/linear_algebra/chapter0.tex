% !TeX root = main.tex
\chapter*{写在前面}
% TODO: Change the tone of the preface to be more humorous
% and engaging.
大部分中国的线性代数教材,开篇多以逆序数从天而降的给出行列式的定义
% TODO: Why it involves permutation? See 3b1b's video
% on determinants
(殊不知行列式逆序数概念的定义本身就需要用到抽象代数\(S_{n}\) 的知识),偌大的代数体系建立于行列式之上,
学生们上来就陷入行列式技巧计算的泥潭,课本大量有用的知识还深藏在无头习题里(没有出处与附注的关键习题)。
教材含糊其辞,学生雾里看花,最终一个学期下来,
学生们对线性代数的理解仅仅停留在机械计算上,对着抽象的代数概念望洋兴叹,感叹大学数学的``厉害''。

这是笔者第二次学习线性代数。选用的教材是大名鼎鼎的 \emph{Linear Algebra Done Right}。
相较于国内大部分数学教材,这本书不以集合论开头,不以抽象的行列式概念导入,
在能补充的部分都有详尽的补充(生怕我看不明白)。希望能借以此书,加深自己对线性代数的理解。
