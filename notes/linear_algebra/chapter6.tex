%!TEX root = main.tex
\chapter{内积空间}
\section{内积与范数}

自然的,定义范数时,我们考虑\(\R^{n} \) 中的长度:\[
    \abs{x} = \sqrt{x_1^2 + x_2^2 + \cdots + x_n^2}
\]
为使该定义在复数域上有定义,我们将平方改为模的平方:
\begin{align*}
    \abs{x} &= \sqrt{\abs{x_1}^2 + \abs{x_2}^2 + \cdots + \abs{x_n}^2}
    &= \sqrt{x_1 \conj{x_1} + x_2 \conj{x_2} + \cdots + x_n
    \conj{x_n}}
\end{align*}

于是我们如此定义内积:
\begin{definition}{内积}
    设\(V\)为域\(\F\)上的线性空间,则称映射
    \[
        \inner{\cdot}{\cdot} : V \times V \to \F
    \]
    为\(V\)上的一个\textbf{内积},如果对任意\(x,y,z \in V, a \in \F\),满足:
    \begin{itemize}
        \item (正定性) \(\inner{x}{x} \geq 0\),且当且仅当\(x = 0\)时取等号
        \item (第一个位置的线性性) \(\inner{ax + y}{z} = a\inner{x}{z} + \inner{y}{z}\)
        \item (共轭对称性) \(\inner{x}{y} = \conj{\inner{y}{x}}\)
    \end{itemize}
\end{definition}

这样得到的内积称为厄米特内积(Hermitian inner product),共轭对称性可导出\[
    \inner{x}{\lambda y} = \conj{\lambda} \inner{x}{y}
\]

\begin{definition}{范数}
    \[
        \norm{x} = \sqrt{\inner{x}{x}}
    \]
\end{definition}

\begin{theorem}{勾股定理}
    \(\norm{u}^{2} + \norm{v}^{2} = \norm{u+v}^{2} \iff \inner{u}{v} = 0\)
\end{theorem}

考虑向量\(u\) 在向量\(v\) 上的正交分解:
设\(u = cv + (u-cv)\) ,则由正交应有:
\begin{align*}
    0 &= \inner{u-cv}{v} = \inner{u}{v} - c\norm{v}^{2} \\
    c &= \frac{\inner{u}{v}}{\norm{v}^{2}}
\end{align*}
故\[
    u = \frac{\inner{u}{v}}{\norm{v}^{2}} v + \left(u -
    \frac{\inner{u}{v}}{\norm{v}^{2}} v\right)
\]

对于任意\(u\) ,\(v\) ,由勾股定理:
\begin{align*}
    \norm{u}^{2} &= \norm{u - \frac{\inner{u}{v}}{\norm{v}^{2}} v}^{2} +
    \norm{\frac{\inner{u}{v}}{\norm{v}^{2}} v}^{2} \\
    &\geq \frac{\abs{\inner{u}{v}}^{2}}{\norm{v}^{2}}
\end{align*}

也就是柯西不等式:
\begin{theorem}{柯西不等式}
    \(\abs{\inner{u}{v}} \leq \norm{u} \norm{v}\)
\end{theorem}

\begin{corollary}{}
    \textbf{平行四边形法则}: \(\norm{u + v}^{2} + \norm{u - v}^{2} = 2\norm{u}^{2} +
    2\norm{v}^{2}\)

    实数域中满足极化恒等式:\(\inner{u}{v} = \frac{1}{4}(\norm{u+v}^{2} -
    \norm{u-v}^{2})\)

    但复数域中我们有
    \begin{align*}
        \inner{u}{v} + \conj{\inner{u}{v}} &= \frac{1}{2}(\norm{u+v}^{2} -
        \norm{u-v}^{2}) \\
        \inner{u}{\i v} + \conj{\inner{u}{\i v}} &=
        \frac{1}{2}(\norm{u+\i v}^{2} -
        \norm{u-\i v}^{2})
    \end{align*}
    从而:
    \[
        \inner{u}{v} = \frac{1}{4}(\norm{u+v}^{2} - \norm{u-v}^{2}) -
        \frac{\i}{4}(\norm{u+\i v}^{2} - \norm{u-\i v}^{2})
    \]
\end{corollary}

\section{规范正交基}

\begin{definition}{规范正交组与规范正交基}
    设\(V\)为域\(\F\)上的线性空间,\(\{e_1, e_2, \cdots, e_m\}\)为\(V\)的一组向量,
    如果对任意\(i\),\(j\) ,有:
    \[
        \inner{e_i}{e_j} =
        \begin{cases}
            1 & i = j \\
            0 & i \neq j
        \end{cases}
    \]
    则称该组向量为\(V\)上的一组\textbf{规范正交组},如果该组向量还是\(V\)的一组基,
    则称该组向量为\(V\)上的一组\textbf{规范正交基}。
\end{definition}

由于任意一组线性无关向量都可以扩充为一组基,故任何一个规范向量组也能扩充为一组规范正交基。

\begin{corollary}
    设\(\{e_1, e_2, \cdots, e_m\}\)为\(V\)上的一组规范正交组,则:\[
        \norm{a_1 e_1 + a_2 e_2 + \cdots + a_m e_m}^{2} =
        \abs{a_1}^{2} + \abs{a_2}^{2} + \cdots + \abs{a_m}^{2}
    \]
\end{corollary}

\begin{theorem}{Bessel不等式}
    设\(\{e_1, e_2, \cdots, e_m\}\)为\(V\)上的一组规范正交组,则对任意\(v \in V\),有:
    \[
        \norm{v}^{2} \geq \abs{\inner{v}{e_1}}^{2} +
        \abs{\inner{v}{e_2}}^{2} + \cdots + \abs{\inner{v}{e_m}}^{2}
    \]
    等号成立的充分必要条件是\(v\)在\(e_1, e_2, \cdots, e_m\)张成的子空间中。
\end{theorem}

\begin{proof}
    设\(u = \inner{v}{e_1} e_1 + \inner{v}{e_2} e_2 + \cdots +
    \inner{v}{e_m} e_m\),对于任意\(e_{k}\),有\(\inner{v-u}{e_k} = 0\),
    故\[
        \inner{v-u}{u} = 0
    \]
    则由勾股定理:
    \[
        \norm{v}^{2} = \norm{v-u}^{2} + \norm{u}^{2} \geq
        \norm{u}^{2} = \abs{\inner{v}{e_1}}^{2} +
        \abs{\inner{v}{e_2}}^{2} + \cdots + \abs{\inner{v}{e_m}}^{2}
    \]
    等号成立的充分必要条件是\(v-u = 0\),即\(v\)在\(e_1, e_2, \cdots, e_m\)张成的子空间中。
\end{proof}

\begin{corollary}{向量的规范正交基表示}
    设\(\{e_1, e_2, \cdots, e_n\}\)为\(V\)的一组规范正交基,则对任意\(v \in V\),有
    \begin{itemize}
        \item \(v = \inner{v}{e_1} e_1 + \inner{v}{e_2} e_2 + \cdots +
            \inner{v}{e_n} e_n\)
        \item \(\norm{v}^{2} = \abs{\inner{v}{e_1}}^{2} +
            \abs{\inner{v}{e_2}} ^{2} + \cdots + \abs{\inner{v}{e_n}}^{2}\)
        \item \(\inner{u}{v} = \inner{u}{e_1} \conj{\inner{v}{e_1}} +
                \inner{u}{e_2} \conj{\inner{v}{e_2}} + \cdots + \inner{u}{e_n}
            \conj{\inner{v}{e_n}}\) (Parseval 恒等式)
    \end{itemize}
\end{corollary}

这里出现的Parseval恒等式与Bessel不等式与傅立叶级数中的Parseval恒等式与Bessel不等式是类似的。
\begin{note}{}
    \(f\) 的傅立叶级数展开系数有Bessel不等式:
    \[
        \frac{a_{0}^{2}}{2} + \sum_{n=1}^{\infty} (a_n^2 + b_n^2) \leq
        \frac{1}{\pi} \int_{-\pi}^{\pi} f (x)^{2} \d{x}
    \]
    若\(f\) 的傅立叶级数一致收敛于\(f\),则等号成立。(Parseval恒等式)
\end{note}

\begin{theorem}{Gram-Schimidt 正交化}
    设\(V\)为域\(\F\)上的线性空间,\(\{v_1, v_2, \cdots, v_n\}\)为\(V\)的一组线性无关向量,
    令\(f_{1}= v_{1}\),\(f_{k}\) 依次定义为:
    \[
        f_k = v_k - \sum_{i=1}^{k-1} \frac{\inner{v_k}{f_i}}{\norm{f_i}^{2}} f_i
    \]
    再令\(e_k = \frac{f_k}{\norm{f_k}}\),则\(\{e_1, e_2, \cdots,
    e_n\}\)为\(V\)的一组规范正交组。
    \[
        \Span(v_1, v_2, \cdots, v_k) = \Span(e_1, e_2, \cdots, e_k)
    \]
\end{theorem}

Gram-Schimidt 正交化过程是对新向量反复移除旧向量在其上的分量的过程。

\begin{proof}
    用数学归纳法证明,\(k=1\)时显然成立,假设对\(j\leq k\)时成立,则:
    \begin{align*}
        \inner{e_{k}}{e_{j}} &= \frac{1}{\norm{f_k} \norm{f_j}}
        \inner{f_k}{f_j} \\
        &= \frac{1}{\norm{f_k} \norm{f_j}} \inner{v_k -
        \sum_{i=1}^{k-1} \frac{\inner{v_k}{f_i}}{\norm{f_i}^{2}} f_i}{f_j} \\
        &= \frac{1}{\norm{f_k} \norm{f_j}} \left(
        \inner{v_{k}}{f_{j}} - \inner{v_{k}}{f_{j}} \right) \\
        &= 0
    \end{align*}
    故\(\{e_1, e_2, \cdots, e_n\}\)为\(V\)的一组规范正交组。又由于该组与原向量组维数相同,
    故该组为\(V\)的一组规范正交基。
\end{proof}

% todo: 舒尔定理

\subsection{内积空间上的线性泛函}

\begin{theorem}{Rietz表示定理}
    设\(V\)为域\(\F\)上的内积空间,则对任意线性泛函\(f \in V'\),存在唯一的向量\(y \in V\),
    使得对任意\(x \in V\),有:
    \[
        f(x) = \inner{x}{y}
    \]
    即映射\(y \mapsto f\)为从\(V\)到\(V'\)的共轭线性同构。
\end{theorem}