% !TeX root = main.tex
\chapter{点集(拓扑)}
\subsection{连续函数的拓扑定义}
\begin{definition}
    \(f\) 是连续函数当且仅当开集的原像是开集.
\end{definition}

\begin{proof}
    设\(f: X\to Y\) 是连续的
    \paragraph{充分性}
    \(\forall y \in Y, x \in X\) 且\(f(x)=y\),
    任取\(\varepsilon >0\). 则\(B_{\varepsilon}(y) \subset Y\)
    是开集. 从而其原像\(U=f^{-1}\left( B_{\varepsilon}(y)
    \right)\subset X\) 是开集, 且\(x \in U\). 从而存在\(\delta >0\)
    使得\(B_{\delta}(x) \subset U\). 由\(y\) 的任意性知\(f\) 是连续函数.
    \paragraph{必要性}
    设\(V \subset Y\) 是开集且\(f(U)=V\). \(\forall x \in U\) 令\(y=f(x)\).
    存在\(\varepsilon >0\) 使得\(B_{\varepsilon}(y) \subset
    V\). 由\(f\) 连续性可知存在\(\delta >0\) 使得\(\forall x' \in
    B_{\delta}(x)\), \(y'=f(x')\in B_{\varepsilon}(y)
    \subset V\). 即\(x' \in U\). 由\(x'\)
    的任意性知\(B_{\delta}(x) \subset U\). 故\(x\) 是\(U\) 的内点. 故\(U\) 是开集.
    \paragraph{结论}
    因此,\(f\) 是连续的当且仅当开集的原像是开集.
\end{proof}

% 开集闭集简单性质
% 开集构造
% 康托尔集和康托尔函数