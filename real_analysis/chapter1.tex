% !TeX root = main.tex
\chapter{集合}
\section{Basics}

\subsubsection{集合与量词}

集合的交/并能够表示任意/存在量词,是因为集合交/并本身的定义就涉及任意/存在量词。二者的等价性使利用集合进行推理成为了可能:

\begin{theorem}
    \begin{align*}
        \exists x \in X, \forall y \in Y \quad P(x,y) \\
        \Rightarrow \forall y \in Y, \exists x \in X \quad P(x,y)
    \end{align*}
\end{theorem}

\begin{proof}
    \begin{align*}
        \bigcup_{x \in X} \bigcap_{y \in Y} P(x,y) \subseteq
        \bigcap_{y \in Y} \bigcup_{x \in X} P(x,y)
    \end{align*}
\end{proof}

\section{Infinity}
\subsection{The Art of Stealing from Infinity}

\begin{problem}
    证明\((0,1)\)与 \([0,1)\) 等势。
\end{problem}

\begin{proof}
    考虑\(f:(0,1) \to [0,1)\):\[
        f(x) =
        \begin{cases}
            \frac{1}{n-1} & x =\frac{1}{n}, n=3,4,\dots  \\
            0 & x=\frac{1}{2} \\
            x & \text{otherwise}
        \end{cases}
    \]
\end{proof}
% TODO: reference to
% https://math.stackexchange.com/questions/213391/how-to-construct-a-bijection-from-0-1-to-0-1

\begin{quote}
    The reason why you can map some set into some bigger
    set bijectively is precisely because they are infinite,
    so you must exploit this fact. If you don't, you have no chance.
\end{quote}

\begin{problem}
    证明\(\R\backslash \Q \) 与 \(\R\) 等势。
\end{problem}

\begin{proof}
    令\(P=\left\{ \sqrt{2}r:r \in \Q \right\} \)。
    则\(P \subset \R\backslash \Q\) 且\(P\sim
    \Q\)。故\(P\cup \Q\sim \Q\)。令\(f: P\cup \Q \to \Q\) 是一个双射。
    则映射: \[
        h(x) =
        \begin{cases}
            f(x) & x\in P\cup \Q \\
            x & x \notin P\cup \Q
        \end{cases}
    \]
    是一个从\(\R\) 到\(\R\backslash \Q\) 的双射。故\(\R\backslash
    \Q \sim \R\)。
\end{proof}

这里\(P\) 是用来利用可数集的等势性,把\(\Q\) 藏起来的辅助集合。
同样的方法可证明不可数集除去可数子集后仍不可数,甚至更高的基数。

% 集合运算与加法乘法除法的对应
% 对称差
% 无穷交并
% 上下确界定义性质
% fUa=Ufa
% 存在与任意

% bernstein定理证明
% 为什么A《=B 与A》=B只能成立一个
% 连续函数与解析函数只有c个
% 可测集只有2^c个

% 无穷就是与部分等价 数不完
% 可数集合的表示是一列有序的数
% aa=na=a 对角线法

% c的表示: 一串小数
% c^a=c^n=2^a=c

% 为什么无法定义概率 22题
% 不可数集合必有聚点

% 选择公理
% 维数为什么这么定义

% ???
% 可测集与不可测集等势?
% 构造出一个mA+mB<mA+B的集合
% 示性函数可以用来干嘛?可以证明mA+mB=m\cup B +mA\cap B?