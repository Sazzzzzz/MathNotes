% !TeX root = main.tex
\chapter{特征值与特征向量}
特征值与特征向量,以及接下来的Jordan标准型,都是为了分解算子,本质上是在研究线性空间上自同态的结构。

\section{特征值}
\begin{definition}{算子}
    算子:从一个线性空间到自身的映射
\end{definition}

算子在定义上已经具备了可逆的基础。如果线性映射是不同线性空间之间的同态,那算子就是自同态。
算子比一般的线性映射有趣的地方在于: 算子可以求幂。

预想研究算子的性质,一个很好的思路是把算子分解成一个个不变子空间上算子的组合。
特征向量就是最简单的不变子空间——\textbf{一维不变子空间}里的元素。在这个子空间上,算子被简化为纯粹的拉伸与压缩。
\[
    Tv=\lambda v \Leftrightarrow \nullspace(T-\lambda I)\neq {0}
\]

不同特征值的分解下产生的不变子空间相互``正交''\footnote{这里的正交仅指子空间构成直和}。
以下比较\textit{Linear Algebra Done Right} 与\textit{高等代数}中证明方法的不同:

\begin{theorem}{不同特征值对应特征向量无关}
    设 \(T \in \L(V)\)那么分别对应于\(T\) 的不同特征值的特征向量构成的每个组都线性无关
\end{theorem}

\begin{proof}
    (\cite[p.~114]{LADR})
    假设欲证结论不成立。那么存在最小的正整数 \(m\), 使得\(T\)
    对应于其互异特征值\(\lambda_1, \dots ,\lambda_{m}\)
    的特征向量 \(v_1, \dots, v_m\)构成线性相关向量组。
    于是存在\(\alpha_1, \dots ,\alpha_{m}\),其中各数均非零,使得
    \[\alpha_1 v_1 + \cdots + \alpha_m v_m = 0\]
    将 \(T-\lambda_{m}I\) 作用于上式两侧,得
    \[
        \alpha_1 (\lambda_1 - \lambda_m)v_1 + \cdots +
        \alpha_{m-1} (\lambda_{m-1} - \lambda_m)v_{m-1} = 0
    \]
    因为特征值\(\lambda_1,\dots ,\lambda_{m}\)互异,所以上式中各系数均不等于0。
    于是 \(v_1, \dots , v_{m-1}\) 就是由对应于 \(T\)
    的互异特征值的 \(m- 1\) 个特征向量构成的线性相关向量组,这和 \(m\) 最小的假设相矛盾. 此矛盾
    就说明欲证命题成立.
\end{proof}

\begin{proof}
    (\cite[p.~204 定理~8]{高等代数})
    对特征值的个数做数学归纳法。由于特征向量是不为零的。单个特征向量显然线性无关。先设属于\(k\)
    个不同特征值的特征向量线性无关,我们证明属于\(k+1\) 个不同特征值\(\lambda_1,\dots
    ,\lambda_{k+1}\) 的特征向量\(\xi_1,\dots, \xi_{k+1}\) 也线性无关。
    设关系式
    \begin{equation}
        a_1\xi_1 + a_2\xi_2 + \dots + a_{k+1}\xi_{k+1} = 0 \label{1}
    \end{equation}
    两边同乘\(\lambda_{k+1}\) 有:
    \begin{equation}
        a_1\lambda_{k+1}\xi_1 + a_2\lambda_{k+1}\xi_2 +
        \cdots + a_{k+1}\lambda_{k+1}\xi_{k+1} = 0 \label{eq:2}
    \end{equation}
    式子两端同时施加变换\(\A\),有:
    \begin{equation}
        a_1\lambda_1\xi_1 + a_2\lambda_1\xi_2 + \cdots +
        a_{k+1}\lambda_1\xi_{k+1} = 0 \label{eq:3}
    \end{equation}
    \cref{eq:2} 减去 \cref{eq:3},得:
    \[
        a_1(\lambda_1 - \lambda_{k+1})\xi_1 + a_2(\lambda_1
        - \lambda_{k+1})\xi_2 + \cdots +
        a_{k}(\lambda_{k} - \lambda_{k+1})\xi_{k+1} = 0
    \]
    根据归纳假设,\(\xi_1,\dots \xi_{k}\) 线性无关,于是
    \[
        a_{i}(\lambda_{i} - \lambda_{k+1}) = 0, \quad i=1,\dots ,k
    \]
    但\(\lambda_{i}-\lambda_{k+1}\neq 0\),所以\(a_{i}=0\)。
    从而\(a_{k+1}\) 也为零。这说明\(\lambda_1,\dots ,\lambda_{k+1}\) 线性无关
\end{proof}

短短的几行证明,不知蕴含了作者大一多少不解\UseVerb{cry}。哪个证明更加符合直觉,
想必是高下立判了吧\UseVerb{surprised}

\(T\) 的不变子空间包括可交换的算子的值域与零空间,其自身及其多项式的值域与子空间自然也包含在内。特别的,
自身零空间的子空间与包含值域的子空间也是不变子空间。子空间的交与和也不变。
% TODO:  Is that all the invariant subspaces?

习题当中我们得以看到特征值更多的性质:
\begin{enumerate}
    \item \(T\) 与自身共轭的算子特征值相同,特征向量相差一个平移
    \item 对偶算子与原算子特征值相同,特征向量相差一个转置
    \item \(T\) 的特征值也是\(T_{\C}\) 的特征值
    \item \(T\) 的特征值也是商算子\(T/U\) 的特征值,其中\((T/U)v=Tv+U\)
\end{enumerate}

\section{最小多项式}

% 特征值的存在性(Tv一列必线性相关)
% 最小多项式的存在性(把整个的约化成小的,然后小的必然存在)
% 以上命题也可以直接Hamilton-Cayley定理来证明。

最小多项式的神奇性质:
\begin{enumerate}
    \item 最小多项式的根是特征值
    \item 最小多项式整除所有零化多项式
    \item 商算子的最小多项式是原算子的最小多项式除以不变子空间的最小多项式?
\end{enumerate}
求解方法
% (任给多项式可求最小多项式为多项式的矩阵)

% 习题4: p(T) 的特征值a是p(lambda)