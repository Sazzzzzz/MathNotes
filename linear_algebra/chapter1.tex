% !TeX root = main.tex
\chapter{向量空间}
\section*{小结}
\begin{itemize}
    \item 域的定义
    \item 元组和他们的运算
\end{itemize}
\section{那些``显然''的结论}
\subsection{\texorpdfstring{\(\i\)}{}
    究竟等于\texorpdfstring{\(\sqrt{-1}\)}{} 还是
\texorpdfstring{\(-\sqrt{-1}\)}{}?}

贴吧里曾流传着这么一张图片:
% TODO: Add vertical dashed line
\begin{flalign*}
    &
    \begin{aligned}
        \frac{1}{\i} &= \frac{\sqrt{-1} \cdot \sqrt{-1}}{\i}\\
        &= \frac{\i \cdot \i}{\i} = \i
    \end{aligned}
    &&
    \begin{aligned}
        \frac{1}{\i}&=\frac{1\cdot \i}{\i \cdot \i}\\
        &=\frac{\i}{-1}=-\i
    \end{aligned}&
\end{flalign*}
如果我们说的再直接一些:
\[
    1 = \sqrt{-1}\cdot \sqrt{-1} = \i \cdot \i = \i^{2} = -1
\]

悖论的关键在于, 虚数 \(\i\) 的定义从来不是 \(\sqrt{-1}\) 或者\(-\sqrt{-1}\),
而是\(\i^{2}=-1\). \(\i\) 只是满足这个性质的一个数. 实数轴从不在乎单位虚数在它的左边或是右边.
从某种意义上看, \(\i\) 可以看作是\(\sqrt{-1}\)和\(-\sqrt{-1}\)的任选一
\footnote{事实上, 在抽象代数中广泛使用\(\sqrt{-1}\) 记号而非\(\i\), 原因就在于此}.
如果我们``假定''\(\i=\sqrt{-1}\),
那么我们将所有的\(\i\)再替换成\(j=-\i=-\sqrt{-1}\), 所有的式子仍将成立,
欧拉公式\(e^{j\pi}+1=0\)
照样对偶.
\footnote{至于\(\sqrt{ab}=\sqrt{a}\cdot\sqrt{b}\)在复数域中为何不再成立,
读者可以参见复变函数的教材.}

在本书中, 复数被定义为满足复数乘法和加法关系的二元数组. \(\i^{2}=-1\)则更是结论而不是定义了.
\begin{definition}
    一个复数是一个有序对\((a,b)\) , 其中\(a,b\in\R\) ,
    不过我们会把这写成\(a+b\i\)
    全体复数所构成的集合用\(\C\) 表示:
    \[\C=\{a+b\i:a,b\in \C\}\]
    \(\C\) 上的加法(addition)和乘法(multiplication)定义为
    \[(a+b\i)+(c+d\i)=(a+c)+(b+d)\i,\]
    \[(a+b\i)(c+d\i)=(ac-bd)+(ad+bc)\i;\]
    其中\(a,b,c,d\in\R\)
\end{definition}

\subsection{域是什么?}

% TODO: Add reference
抽象代数中我们与域第一次见面时, 域的定义是这样一个诘屈聱牙的形式:
\begin{definition}
    无零因子的的交换幺环称为\emph{整环}; 若一个幺环\(R\) 满足条件
    \(R^{*}=U\), 即\(R^{*}\) 对于乘法构成群, 则称\(R\)为\emph{除环}.
    交换的除环称为\emph{域}.
\end{definition}

\begin{figure}[H]
    \centering
    \includegraphics[width=0.9\textwidth]{compiled/Jesse.pdf}
\end{figure}
事实上域的定义没有那么复杂, 上面域的定义完全可以化简为以下通俗易懂且好记的等价形式:
\begin{definition}
    设\(F\) 是一个集合, 其中定义了加法\((+)\) 和乘法\((\cdot)\) 两种运算, 满足以下条件:
    \begin{itemize}
        \item \((F, +)\) 构成一个交换群
        \item \((F^{*}, \cdot)\) 构成一个交换群
        \item 加法和乘法满足(左/右)分配律
    \end{itemize}
    则称\((F, +, \cdot)\) 是一个\emph{域}.
\end{definition}

这个定义乍看是非常优雅的, 满足了我们对域美好性质的一切幻想. 但细想一下,
凭什么乘法只能在\(F^{*}\)上构成群呢?难道乘法天然比加法低人一等吗?

事实上我们可以从定义出发, 证明若乘法在\(F\)上构成群, 则\(F\)最多只包含零元素. 这样的域是平凡的.

\begin{theorem}
    若集合\((F, +,
    \cdot)\)且\((F,\cdot)\)构成群, 则\(F\)只能包含零元素.
    \footnote{这是抽象代数中一道关于域的课后习题, 见}
\end{theorem}
\begin{proof}
    % TODO: Add reference
    先证明\(0 \cdot a=0\), \(\forall a \in F\).
    \begin{align*}
        0 \cdot a &= (0+0) \cdot a\\
        &= 0 \cdot a + 0 \cdot a
    \end{align*}
    故\(0 \cdot a=0\). 从而 \(0\cdot 0^{-1}=0\).
    但由乘法逆元的定义,  \(0 \cdot 0^{-1}=1\), 从而\(1=0\).
    于是\(\forall a \in F\), \(a=1 \cdot a=0 \cdot a=0\).
    故\(F\)只包含零元素.
\end{proof}

所以我们现在见到的域已经是我们能做到的性质最好的定义了.

此外还有一些特殊的域:
\begin{itemize}
    \item \emph{代数闭域}:一个域\(F\) 被称作代数闭域, 当且仅当任何系数属于\(F\)
        且次数大于零的单变量多项式在\(F\) 里至少有一个根. \footnote{性质可见维基百科}
        % TODO: Add reference
    \item \emph{数域}:元素全是数的域.
\end{itemize}
实际上, 非数域上的线性代数也非常精彩, 下面给出一个例子:

\subsection{2023四省联考数学16题深度解析}

% TODO:  Add fig
\begin{problem}
    右图为一个开关阵列, 每个开关只有“开”和“关两种状态, 按其中一个开关1次,
    将导致自身和所有相邻的开关改变状态.例如,按\((2,2)\)将导致\((1,2)\), \((2,1)\),
    \((2,2)\), \((2,3)\),
    \((3,2)\)改变状态, 如果要求只改变\((1,1)\)的状态, 求需按开关的最少次数.
\end{problem}

这道题中学范围内的解法可以在任何高考备考的专业教育机构里找到, 大概是利用对称性解决. 而一般的解法在中科院物理所的
\href{https://mp.weixin.qq.com/s/k15Xlib2k9JOVrvBifQoiQ}{这篇文章}
已经解释的非常清楚了. 这里给出问题解决的大概思路.

\begin{itemize}
    \item 一个格子点击两次(偶数次)不改变状态.\footnote{这样的代数结构相当常见: 奇数偶数的乘除、逻辑代数等等}
    \item 一个格子的状态只取决于它被点击偶数次还是奇数次, 因此\emph{与按键的顺序无关}.
\end{itemize}

如果我们把灯的开记为1, 灭记为0, 于是表格中所有灯的状态可以用一个9维向量表示,
例如\((1,0,0,0,0,0,0,0,0)^{\mathrm{T}}\)表示.

我们记开关初始状态向量为\(B=(1,0,\dots,0)^{\mathrm{T}}\), 在第\(i\)
个位置按下开关的状态转移向量为\(T_{i}\), 完全灭灯过程需要按下开关\(x_{i}\) 次, 我们可以得到以下方程组:
\[
    B+ x_{1}T_{1}+x_{2}T_{2}+\dots+x_{9}T_{9}=0
\]
把 \(x_{1}T_{1}+x_{2}T_{2}+\dots+x_{9}T_{9}\)
这一线性组合写成矩阵乘法的形式\(Ax\), 其中\(A=(T_{1}, T_{2}, \dots,
T_{9})\) 称作状态转移矩阵, \(x=(x_{1}, x_{2}, \dots, x_{9})^{\mathrm{T}}\)
于是求解灭灯问题就完全转化成了在有限域
\(\Z_{2}\) 上求解线性方程组\[
    Ax+B=0
\]

\subsubsection{把灭灯问题写成看不懂的样子}

理论上来说, 有矩阵的地方, 就应该有线性变换. 但我们之前的推导过程中矩阵是靠无理由的多个向量拼凑在一起,
用矩阵乘法的定义给捏在一起的, 线性变换究竟在哪里?

实际上灭灯问题的线性空间与线性变换与几何学中向量与点的关系十分相似, 这里给出一个推导.

记\(\F\) 为\(3 \times 3\) 网格中所有灯光状态的集合,
其中每个灯光状态都可以用一个\(3 \times 3=9\) 维的行向量表示, 其中每个元素为0或1. 容易发现,
\(\F=\Z_{2}^{9}\).

同时我们记在\(\F\)
上灯光状态全部变换的集合为\(\overrightarrow{\F}=\overrightarrow{\Z_{2}^{9}}\),
读者不难验证 \(\overrightarrow{\Z_{2}^{9}}\)
对\(\Z_{2}\)上的加法构成群, 对\(\Z_{2}\) 上的加法与数乘构成一个线性空间.

注意到灯光状态变换实际上是通过加法实现\(\Z^{9}_{2}\) 上的变换的, 定义群作用
\begin{align*}
    T: \overrightarrow{\Z_{2}^{9}} \times
    \Z_{2}^{9} &\to \Z_{2}^{9}\\
    (\vec{p},y) &\mapsto \vec{p}+y
\end{align*}

于是\((\Z^{9}_{2},
\overrightarrow{\Z^{9}_{2}}, T)\) 构成一个仿射空间.

不难注意到线性空间\(\overrightarrow{\Z^{9}_{2}}\) 上自然诱导出一组标准基
\(\overrightarrow{e_{n}}=[0,\dots,1,\dots,0]^{\mathrm{T}}\),
表示仅改变第\(n\) 个灯光的状态, 其余灯光不变.  那么原先解法中矩阵\(A\) 就有了实际意义, 他表示从规则构造的向量组
\({T_{1}, \dots, T_{n}}\) 到标准基 \({e_{1}, \dots, e_{n}}\) 的过渡矩阵.
\footnote{当然这里的说法可能并不严谨, 因为规则构造的向量组并不一定是满秩的,
我个人最初的理解是矩阵从``操作空间''到\(\overrightarrow{\Z^{9}_{2}}\) 的线性变换.}
% TODO: More Understanding
% TODO: Solve this with computational algebra

于是求解灭灯问题也自然的等价于在\(\Z_{2}\) 上求解方程组:
\[
    Ax+B=0
\]
% TODO: There might be some misunderstanding on affine
% space and group action

\section{向量空间的定义}
向量空间\footnote{这本书中的向量空间就是高等代数中的线性空间}是一个完全超脱于\(\R^{n}\)的一个概念,
是一个独立而抽象的代数结构. 只有从公理出发, 才能真正理解它.

向量空间看上去有点像交换幺环, 但乘法被数乘代替.
向量空间是广泛存在的, (但不是一些人想的那样无处不在\UseVerb{surprised}),
本章中最有趣的向量空间是\(\F^{S}\):定义域为\(S\),
值域为\(\F\)的函数全体所构成的向量空间
% TODO:  Change another position
% TODO: give it a better look
\begin{quote}
    线性映射在数学中普遍存在, 然而, 他们并不如有些人想象的那样无处不在.
    当这些人错误的写出\(\cos(x+y)\) 等于\(\cos x+\cos y\) 和\(\cos 2x\)
    等于 \(2\cos x\) 时, 似乎连\(\cos\)
    都能看成从\(\R\)到\(\R\)的线性映射.
\end{quote}
如果我们定义:
\[
    \begin{aligned}
        f: \F^{\{0,1,\dots,n\}} &\to \F^{n}\\
        f(x_{0}, x_{1}, \dots, x_{n}) &\mapsto (x_{0},
        x_{1}, \dots, x_{n})
    \end{aligned}
\]

就能找到\(\F^{\{0,1,\dots,n\}}\)和\(\F^{n}\)之间的一个双射,
这事实上证明了\(R^{n}\)
是一个向量空间.
同理, \(\R^{\R}\)等一系列奇形怪状的向量空间也能被定义出来.
\begin{definition}
    设\(V\) 是一个集合, 其中定义了加法\((+)\) 和数乘\((\cdot)\) 两种运算, 满足以下条件:
    \begin{itemize}
        \item \((V, +)\) 构成一个交换群
        \item 数乘满足分配律
    \end{itemize}
    则称\((V, +, \cdot)\) 是一个\emph{向量空间}.
\end{definition}

通过定义可以得到线性空间的一些性质:
\begin{itemize}
    \item 数乘与加法的部分关系:\(a0=0\quad 0v=0\quad (-1)v=-v\)
    \item 加法上的消去律:\(v+w=v+u \Rightarrow w=u \)
    \item (习题2)无零因子性:\(a\cdot v=0 \Rightarrow a=0
        \text{或} v=0\)
\end{itemize}

实向量空间的复化:
\begin{definition}
    设 \(V\) 是实向量空间.
    \begin{itemize}
        \item \(V\) 的\emph{复化(complexification)}记为
            \(V_{\C}\), 等于 \(V \times
            V\). \(V_{\C}\)
            中的元素为有序对 \((u, v)\), 其中
            \(u, v \in V\), 不过我们把它写作 \(u + \i \).
        \item \(V_{\C}\) 上的加法定义为
            \((u_1 + \i v_1) + (u_2 + \i v_2) = (u_1 +
            u_2) + \i (v_1 + v_2)\)
            对所有 \(u_1, v_1, u_2, v_2 \in V\) 成立.
        \item \(V_{\C}\) 上的复标量乘法定义为
            \((a + b\i) (u + \i v) =
                (au - bv) +
            \i (av + bu)\)
            对所有 \(a, b \in \R\) 和所有 \(u, v \in V\) 成立
    \end{itemize}
    % TODO: How to define determinant with Sn?
    % TODO: What does complexification adds?
\end{definition}
这样的形成的代数结构\(V_{\C}\)也是向量空间.

\section{子空间}
子空间的判定法则和环与子环的判定方法颇为相似:
\begin{theorem}[子空间的条件]
    当且仅当 \(V\) 的子集 \(U\) 满足以下三个条件时, \(U\) 是 \(V\) 的子空间.
    \begin{enumerate}
        \item 加法恒等元(additive identity):\(0 \in U\).
        \item 对于加法封闭(closed under addition):\(u, w \in U\)
            意味着 \(u + w \in U\) .
        \item 对于标量乘法封闭(closed under scalar
            multiplication):\(a \in F\)  且 \(u \in U\) 意味着
            \(au \in U\).
    \end{enumerate}
    % TODO: Add subring theorem in parallel
\end{theorem}
子空间的一些有趣例子包括:
\begin{itemize}
    \item 线性方程组的解集是子空间, 更进一步, 满足某齐次线性方程式的点集是子空间.
    \item \(\R\)
        上的连续函数、可微函数构成\(\R^{\R}\)的子空间
    \item \(\{0\}\), 过原点的直线与过原点的平面是 \(\R^{2}\) 上的全部子空间
\end{itemize}
什么不是子空间:
\begin{itemize}\label{item:subspace counterexample}
    \item 加法恒等元不存在:平凡反例\(\emptyset\)
    \item 对加法不封闭:利用孤立分支构造的反例\(\left\{ (a,b)\in
        \C^{2}: a^{3}=b^{3}\right\}\)
    \item 对标量乘法不封闭:利用不连续性构造的反例\(\left\{(a,0):a\in
        \Z\right\}\)
\end{itemize}

如何把子空间变大/变小:
