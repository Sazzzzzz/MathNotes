\chapter{向量空间}
\subsubsection{小结}
\begin{itemize}
    \item 域的定义
    \item 元组和他们的运算
\end{itemize}
\subsection{那些``显然''的结论}
\paragraph{\(\ii\) 究竟等于\(\sqrt{-1}\)还是\(-\sqrt{-1}\)?}

贴吧里曾流传着这么一张图片:
\begin{flalign*}
    &
    \begin{aligned}
        \frac{1}{\ii} &= \frac{\sqrt{-1} \cdot \sqrt{-1}}{\ii}\\
        &= \frac{\ii \cdot \ii}{\ii} = \ii
    \end{aligned}
    &&
    \begin{aligned}
        \frac{1}{\ii}&=\frac{1\cdot \ii}{\ii \cdot \ii}\\
        &=\frac{\ii}{-1}=-\ii
    \end{aligned}&
\end{flalign*}
如果我们说的再直接一些:
\[
    1 = \sqrt{-1}\cdot \sqrt{-1} = \ii \cdot \ii = \ii^{2} = -1
\]

悖论的关键在于,虚数 \(\ii\) 的定义从来不是 \(\sqrt{-1}\) 或者\(-\sqrt{-1}\),
而是\(\ii^{2}=-1\). \(\ii\) 只是满足这个性质的一个数。实数轴从不在乎单位虚数在它的左边或是右边.
从某种意义上看,\(\ii\) 可以看作是\(\sqrt{-1}\)和\(-\sqrt{-1}\)的任选一
\footnote{事实上, 在抽象代数中广泛使用\(\sqrt{-1}\) 记号而非\(\ii\), 原因就在于此}。
如果我们``假定''\(\ii=\sqrt{-1}\),
那么我们将所有的\(\ii\)再替换成\(j=-\ii=-\sqrt{-1}\),所有的式子仍将成立,欧拉公式\(e^{j\pi}+1=0\)
照样对偶.
\footnote{至于\(\sqrt{ab}=\sqrt{a}\cdot\sqrt{b}\)在复数域中为何不再成立,读者可以参见复变函数的教材。}

在本书中,复数被定义为满足复数乘法和加法关系的二元数组。\(\ii^{2}=-1\)则更是结论而不是定义了。
\begin{definition}
    一个复数是一个有序对\((a,b)\) ,其中\(a,b\in\mathbb{R}\) ,不过我们会把这写成\(a+b\ii\) .
    全体复数所构成的集合用\(\mathbb{C}\) 表示:
    \[\mathbb{C}=\{a+b\ii:a,b\in \mathbb{C}\}.\]
    \(\mathbb{C}\) 上的加法(addition)和乘法(multiplication)定义为
    \[(a+b\ii)+(c+d\ii)=(a+c)+(b+d)\ii,\]
    \[(a+b\ii)(c+d\ii)=(ac-bd)+(ad+bc)\ii;\]
    其中\(a,b,c,d\in\mathbb{R}\)
\end{definition}

\paragraph{域是什么?}

% TODO: Add reference
抽象代数中我们与域第一次见面时,域的定义是这样一个诘屈聱牙的形式:
\begin{definition}
    无零因子的的交换幺环称为\emph{整环}; 若一个幺环\(R\) 满足条件
    \(R^{*}=U\),即\(R^{*}\) 对于乘法构成群,则称\(R\)为\emph{除环}. 交换的除环称为\emph{域}.
\end{definition}
% TODO: Add Jesse image
% TODO: Add reference to abstract algebra
事实上域的定义没有那么复杂,上面域的定义完全可以化简为以下的通俗易懂且好记的等价形式:
\begin{definition}
    设\(F\) 是一个集合,其中定义了加法\((+)\) 和乘法\((\cdot)\) 两种运算,满足以下条件:
    \begin{itemize}
        \item \((F, +)\) 构成一个交换群
        \item \((F^{*}, \cdot)\) 构成一个交换群
        \item 加法和乘法满足(左/右)分配律
    \end{itemize}
    则称\((F, +, \cdot)\) 是一个\emph{域}.
\end{definition}

这个定义乍看是非常优雅的,满足了我们对域美好性质的一切幻想。但细想一下,凭什么乘法只能在\(F^{*}\)上构成群呢?难道乘法天然比加法低人一等吗?

事实上我们可以从定义出发,证明若乘法在\(F\)上构成群,则\(F\)最多只包含零元素。这样的域是平凡的。

\begin{theorem}
    若集合\((F, +,
    \cdot)\)且\((F,\cdot)\)构成群,则\(F\)只能包含零元素.
    \footnote{这是抽象代数中一道关于域的课后习题,见}
\end{theorem}
\begin{proof}
    % TODO: Add reference
    先证明\(0 \cdot a=0\), \(\forall a \in F\).
    \begin{align*}
        0 \cdot a &= (0+0) \cdot a\\
        &= 0 \cdot a + 0 \cdot a
    \end{align*}
    故\(0 \cdot a=0\). 从而 \(0\cdot 0^{-1}=0\).
    但由乘法逆元的定义, \(0 \cdot 0^{-1}=1\), 从而\(1=0\).
    于是\(\forall a \in F\), \(a=1 \cdot a=0 \cdot a=0\).
    故\(F\)只包含零元素。
\end{proof}

所以我们现在见到的域已经是我们能做到的性质最好的定义了。

此外还有一些特殊的域:
\begin{itemize}
    \item \emph{代数闭域}:一个域\(F\) 被称作代数闭域,当且仅当任何系数属于\(F\)
        且次数大于零的单变量多项式在\(F\) 里至少有一个根。\footnote{性质可见维基百科}
        % TODO: Add reference
    \item \emph{数域}:元素全是数的域。
\end{itemize}

\subsection{向量空间的定义}
向量空间\footnote{这本书中的向量空间就是高等代数中的线性空间}是一个完全超脱于\(R^{n}\)的一个概念,是一个独立而抽象的代数结构,只有从公理出发,才能真正理解它。

向量空间看上去有点像交换幺环,但乘法被数乘代替。
向量空间是广泛存在的,(但不是一些人想的那样无处不在),本章中最有趣的向量空间是\(F^{S}\):定义域为\(S\),值域为\(F\)的函数全体所构成的向量空间
% TODO:  Change another position
% TODO: give it a better look
\begin{quote}
    线性映射在数学中普遍存在, 然而, 他们并不如有些人想象的那样无处不在.
    当这些人错误的写出\(\cos(x+y)\) 等于\(\cos x+\cos y\) 和\(\cos 2x\)
    等于 \(2\cos x\) 时, 似乎连\(\cos\)
    都能看成从\(\mathbb{R}\)到\(\mathbb{R}\)的线性映射.
\end{quote}
如果我们定义:
\[
    \begin{aligned}
        f: F^{\{0,1,\ldots,n\}} &\to F^{n}\\
        f(x_{0}, x_{1}, \ldots, x_{n}) &\mapsto (x_{0},
        x_{1}, \ldots, x_{n})
    \end{aligned}
\]

就能找到\(F^{\{0,1,\ldots,n\}}\)和\(F^{n}\)之间的一个双射,这事实上证明了\(R^{n}\)
是一个向量空间。
同理,\(\mathbb{R}^{\mathbb{R}}\)等一系列奇形怪状的向量空间也能被定义出来。
\begin{definition}
    设\(V\) 是一个集合,其中定义了加法\((+)\) 和数乘\((\cdot)\) 两种运算,满足以下条件:
    \begin{itemize}
        \item \((V, +)\) 构成一个交换群
        \item 数乘满足分配律
    \end{itemize}
    则称\((V, +, \cdot)\) 是一个\emph{向量空间}.
\end{definition}

通过定义可以得到线性空间的一些性质:
\begin{itemize}
    \item 数乘与加法的部分关系:\(a0=0\quad 0v=0\quad (-1)v=-v\)
    \item 加法上的消去律:\(v+w=v+u \Rightarrow w=u \)
    \item (习题2)无零因子性:\(a\cdot v=0 \Rightarrow a=0
        \text{或} v=0\)
\end{itemize}

实向量空间的复化:
\begin{definition}
    设 \(V\) 是实向量空间.
    \begin{itemize}
        \item \(V\) 的\emph{复化(complexification)}记为
            \(V_{\mathbb{C}}\),等于 \(V \times
            V\)。\(V_{\mathbb{C}}\)
            中的元素为有序对 \((u, v)\),其中
            \(u, v \in V\),不过我们把它写作 \(u + \mathrm{i}v\)。
        \item \(V_{\mathbb{C}}\) 上的加法定义为
            \((u_1 + \mathrm{i}v_1) + (u_2 +
                \mathrm{i}v_2) = (u_1 +
            u_2) + \mathrm{i} (v_1 + v_2)\)
            对所有 \(u_1, v_1, u_2, v_2 \in V\) 成立.
        \item \(V_{\mathbb{C}}\) 上的复标量乘法定义为
            \((a + b\mathrm{i}) (u + \mathrm{i}v) =
                (au - bv) +
            \mathrm{i} (av + bu)\)
            对所有 \(a, b \in \mathbb{R}\) 和所有 \(u, v \in V\) 成立
    \end{itemize}
    % TODO: How to define determinant with Sn?
    % TODO: What does complexification adds?
\end{definition}
这样的形成的代数结构\(V_{\mathbb{C}}\)也是向量空间。

\subsection{子空间}
子空间的判定法则和环与子环的判定方法颇为相似:
\begin{theorem}[子空间的条件]
    当且仅当 \(V\) 的子集 \(U\) 满足以下三个条件时,\(U\) 是 \(V\) 的子空间.
    \begin{enumerate}
        \item 加法恒等元(additive identity):\(0 \in U\).
        \item 对于加法封闭(closed under addition):\(u, w \in U\)
            意味着 \(u + w \in U\) .
        \item 对于标量乘法封闭(closed under scalar
            multiplication):\(a \in F\)  且 \(u \in U\) 意味着
            \(au \in U\).
    \end{enumerate}
    % TODO: Add subring theorem in parallel
\end{theorem}
子空间的一些有趣例子包括:
\begin{itemize}
    \item 线性方程组的解集是子空间,更进一步,满足某齐次线性方程式的点集是子空间。
    \item \(\mathbb{R}\) 上的连续函数、可微函数构成\(\mathbb{R}^{\mathbb{R}}\)的子空间
    \item \(\{0\}\), 过原点的直线与过原点的平面是 \(\mathbb{R}^{2}\) 上的全部子空间
\end{itemize}
什么不是子空间:
\begin{itemize}\label{item:subspace counterexample}
    \item 加法恒等元不存在:平凡反例\(\emptyset\)
    \item 对加法不封闭:利用孤立分支构造的反例\(\left\{ (a,b)\in
        \mathbb{R}^{2}: a^{3}=b^{3}\right\}\)
    \item 对标量乘法不封闭:利用不连续性构造的反例\(\left\{(a,0):a\in \mathbb{Z}\right\}\)
\end{itemize}

如何把子空间变大/变小:
